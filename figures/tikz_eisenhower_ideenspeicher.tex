\begin{tikzpicture}


%% - Draw grid
%% - --------------------------------------

\draw[step=4cm,thick] (0,0) grid (12,12);

%% - Draw Arrows
%% - --------------------------------------

\draw [decoration={markings,mark=at position 1 with
    {\arrow[scale=3,>=stealth]{>}}},postaction={decorate}] (0,0) -- (12,0);

\draw [decoration={markings,mark=at position 1 with
    {\arrow[scale=3,>=stealth]{>}}},postaction={decorate}] (0,0) -- (0,12);
    
%% - label x- and y-Axis
%% - --------------------------------------

\node[scale=1.0, align=center] at (13,0) {Kosten\\$(K)$};
\node[scale=1.0, align=center] at (0,12.8) {Zeitaufwand\\$(T)$};

%% - Draw x- and y- Axis scale
%% - --------------------------------------

\node[draw,scale=1,shape=rectangle,draw=none, rotate=90,anchor=east] at (2,-0.2) {tief};
\node[draw,scale=1,shape=rectangle,draw=none, rotate=90,anchor=east] at (6,-0.2) {mittel};
\node[draw,scale=1,shape=rectangle,draw=none, rotate=90,anchor=east] at (10,-0.2) {hoch};

\node[draw,scale=1,shape=rectangle,draw=none,anchor=east] at (-0.2,2) {tief};
\node[draw,scale=1,shape=rectangle,draw=none,anchor=east] at (-0.2,6) {mittel};
\node[draw,scale=1,shape=rectangle,draw=none,anchor=east] at (-0.2,10) {hoch};

%% - Draw visible nodes in grid
%% - --------------------------------------

\node[draw,scale=1.2,shape=rectangle,draw=none,gray, align=center] at (2,2) {zur Umsetzung \\ empfohlen};
\node[draw,scale=1.2,shape=rectangle,draw=none,gray, align=center] at (6,2) {zur Umsetzung \\ empfohlen};
\node[draw,scale=1.2,shape=rectangle,draw=none,gray, align=center] at (2,6) {zur Umsetzung \\ empfohlen};
\node[draw,scale=1.2,shape=rectangle,draw=none,gray, align=center] at (6,6) {zur Umsetzung \\ empfohlen};

\node[draw,scale=1.2,shape=rectangle,draw=none,gray, align=center] at (2,10) {in Ressourcen-\\planung \\aufnehmen };
\node[draw,scale=1.2,shape=rectangle,draw=none,gray, align=center] at (6,10) {in Ressourcen-\\planung \\aufnehmen };

\node[draw,scale=1.2,shape=rectangle,draw=none,gray, align=center] at (10,2) {in Kosten- \\ planung \\ aufnehmen };
\node[draw,scale=1.2,shape=rectangle,draw=none,gray, align=center] at (10,6) {in Kosten- \\ planung \\ aufnehmen };

\node[draw,scale=1.2,shape=rectangle,draw=none,gray, align=center] at (10,10) {Idee \\ langfristig \\ einplanen oder \\ verwerfen};


%% - Draw selection line
%% - --------------------------------------

\filldraw [color=blue, opacity=0.05] (0,0) -- (12,0) -- (12,8) -- (8,8) -- (8,12) -- (0,12) -- cycle;

\node[draw,scale=1,shape=rectangle,draw=none, blue,align=center] at (-2.0,3) {Selektions-\\fläche};
\draw [blue] (-1,3) -- (0.5,3);

\end{tikzpicture}