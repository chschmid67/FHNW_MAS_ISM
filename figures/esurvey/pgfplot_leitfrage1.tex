\pgfplotstableread[col sep=comma, header=false]{
% -- <percent value>, <startpoint from above> , <label>
      51,0,  Persönliches Verhalten
      43,0.7,  Datenschutz/sicherheit/klassifizierung/zugriff
      39,1.4,  Risikobewusstsein
      35,2.1,  Wissen über Bedrohungen
      34,2.8,  Kenntnisse Securitypolicies
      29,3.5,  Betrifft die physische Sicherheit
      25,4.2,  Umgang mit Passwörtern
      19,4.9,  Umgang mit Firmendaten
      18,5.6,  Security Schulung
      16,6.3,  Umgang mit Email
      9,7.0,  Verhalten in der Öffentlichkeit
      9,7.7,  Weltanschauung/Kultur/Soziales
      8,8.4,  Produktesicherheit
      6,9.1,  Umgang mit persönlichen Daten
      6,9.8,  Umgang mit Kundendaten
      6,10.5,  Security Incidents
      6,11.2,  Technische Hilfsmittel
      4,11.9,  Verantwortung Management
      2,12.6,  Erkennung verdächtige Aktivitäten
}\datatable

\begin{tikzpicture}

  \begin{axis}[
%    height = 5.5cm,
    xbar,
    y=-.7cm,
    enlarge y limits={abs=0.45cm},
    axis x line       = none,
    tickwidth         = 0pt,
    y axis line style = { opacity = 0 },
   yticklabels from table={\datatable}{2},
    ytick=data,
%    yticklabel style={text width=9cm,align=right},
    nodes near coords,
    nodes near coords align={horizontal},
%    nodes near coords={\pgfmathprintnumber\pgfplotspointmeta\%},
    ]
    \addplot table [y=1, x=0] {\datatable};
  \end{axis}
  
  % --
% -- draw surrounding box
% --
  \node[
      draw=black, very thin,
      minimum width=\textwidth,
      fit=(current bounding box.north west) (current bounding box.south east),
    ]at (current bounding box.center){};
    
\end{tikzpicture}