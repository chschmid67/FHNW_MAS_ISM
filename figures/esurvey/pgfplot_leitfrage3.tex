\pgfplotstableread[col sep=comma, header=false]{
% -- <percent value>, <startpoint from above> , <label>
      55,0,  Die anderen Arbeitskollegen
      30,0.7,  Unklare Richtlinien
      24,1.4,  Menschliches Fehlverhalten
      21,2.1,  Fehlende Zutrittskontrolle
      20,2.8,  Umgang mit Email
      19,3.5,  Mangelnde Kenntnisse IT Technologien
      18,4.2,  Sabotage/Spionage/Hacker/Social Engineering
      16,4.9,  Umgang vertrauliche Dokumente (drucken/kopieren/entsorgen)
      11,5.6,  Ungepatchte Software
      10,6.3,  Eingesetzte Technologien
      10,7.0,  Mobilgeräte (Handy/Tablets/Datenträger)
      10,7.7,  Externe Mitarbeiter + Partner/Besucher
      10,8.4,  Infrastruktur (WLAN/Netz/VoIP)
      9,9.1,  Umgang mit Passwörtern
      9,9.8,  Benutzung von Cloud Services
}\datatable

\begin{tikzpicture}

  \begin{axis}[
%    height = 5.5cm,
    xbar,
    y=-.7cm,
    enlarge y limits={abs=0.45cm},
    axis x line       = none,
    tickwidth         = 0pt,
    y axis line style = { opacity = 0 },
   yticklabels from table={\datatable}{2},
    ytick=data,
%    yticklabel style={text width=9cm,align=right},
    nodes near coords,
    nodes near coords align={horizontal},
%    nodes near coords={\pgfmathprintnumber\pgfplotspointmeta\%},
    ]
    \addplot table [y=1, x=0] {\datatable};
  \end{axis}
  
  % --
% -- draw surrounding box
% --
  \node[
      draw=black, very thin,
      minimum width=\textwidth,
      fit=(current bounding box.north west) (current bounding box.south east),
    ]at (current bounding box.center){};
    
\end{tikzpicture}