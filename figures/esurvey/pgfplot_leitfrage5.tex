%\documentclass[../../main.tex]{subfiles}

%\begin{document}

\pgfplotstableread[col sep=comma, header=false]{
% -- <percent value>, <startpoint from above> , <label>
      45,0,  Aktuelle/lesbare Informationen zum Cybercrimegeschehen
      45,1.2,  Integration in Firmenveranstaltungen
      44,2.4,  Verständliche Empfehlungen zum Einsatz von IT für den Alltag (Firma/Heim)
      30,3.6,  Wirkungsvolle Sicherheitsschulungen und Trainings
      26,4.8,  Live Hacking Event
      24,6.0,  Praktizierbare Verhaltensanweisungen (Sicherheitsvorfall/Datenaustausch/etc.)
      13,7.2,  Informationen zu physischer Sicherheit
      13,8.4,  Vereinfachung/Verbesserung der Prozesse/Systemlandschaften
      12,9.8,  Praktisches Know-How Datenklassifizierung/-schutz/-sicherheit
      11,11.2,  Offene+transparente Kommunikation von Sicherheitsvorfällen
      4,12.4,  Verantwortungsübernahme des Managements
}\datatable

\begin{tikzpicture}

  \begin{axis}[
%    height = 5.5cm,
    xbar,
    y=-.7cm,
    enlarge y limits={abs=0.45cm},
    axis x line       = none,
    tickwidth         = 0pt,
    y axis line style = { opacity = 0 },
   yticklabels from table={\datatable}{2},
    ytick=data,
    yticklabel style={text width=9cm,align=right},
    nodes near coords,
    nodes near coords align={horizontal},
%    nodes near coords={\pgfmathprintnumber\pgfplotspointmeta\%},
    ]
    \addplot table [y=1, x=0] {\datatable};
  \end{axis}
  
% --
% -- draw surrounding box
% --
  \node[
      draw=black, very thin,
      minimum width=\textwidth,
      fit=(current bounding box.north west) (current bounding box.south east),
    ]at (current bounding box.center){};
    
\end{tikzpicture}

%\end{document}
