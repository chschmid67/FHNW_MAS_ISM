\pgfplotstableread[col sep=comma, header=false]{
% -- <percent value>, <startpoint from above> , <label>
      79,0,  Einfache Arbeitsanweisungen/Klare Richtlinien/Verhaltenskodex
      63,0.7,  Security Awareness Kampagne
      55,1.4,  Regelmässige Informationsveranstaltungen
      49,2.1,  Sicherheitsschulungen und Trainings
      22,2.8,  Persönliches Bewusstsein fördern
      20,3.5,  Technische Massnahmen
      19,4.2,  Bekannte Meldestellen: Security Officer/CISO
      15,4.9,  Vorbildverhalten des Managements
      15,5.6,  Informationen/Warnungen zu aktuellen Cyberbedrohungen
      12,6.3,  Es fehlt nichts/keine Idee
      6,7.0,  Risiko- und Datenklassifikation
}\datatable

\begin{tikzpicture}

  \begin{axis}[
%    height = 5.5cm,
    xbar,
    y=-.7cm,
    enlarge y limits={abs=0.45cm},
    axis x line       = none,
    tickwidth         = 0pt,
    y axis line style = { opacity = 0 },
   yticklabels from table={\datatable}{2},
    ytick=data,
%    yticklabel style={text width=9cm,align=right},
    nodes near coords,
    nodes near coords align={horizontal},
%    nodes near coords={\pgfmathprintnumber\pgfplotspointmeta\%},
    ]
    \addplot table [y=1, x=0] {\datatable};
  \end{axis}
  
  % --
% -- draw surrounding box
% --
  \node[
      draw=black, very thin,
      minimum width=\textwidth,
      fit=(current bounding box.north west) (current bounding box.south east),
    ]at (current bounding box.center){};
    
\end{tikzpicture}