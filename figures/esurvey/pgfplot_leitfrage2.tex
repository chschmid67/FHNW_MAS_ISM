\pgfplotstableread[col sep=comma, header=false]{
% -- <percent value>, <startpoint from above> , <label>
      46,0,  Spam Email und Malware
      38,0.7,  Familienmitglieder
      28,1.4,  Mangelnde Kenntnisse IT Technologien
      25,2.1,  WLAN
      21,2.8,  Persönliches Verhalten
      21,3.5,  Physische Sicherheit
      21,4.2,  Ungepatchte Software/Hardware
      17,4.9,  Installierte Software
      14,5.6,  Internetanschluss
      13,6.3,  Download von Inhalten
      12,7.0,  Anzahl Geräte mit Internetzugriff
      11,7.7,  Umgang mit Passworten
      10,8.4,  Nichts
      7,9.1,  Mobile Geräte
      6,9.8,  Keine Antwort/unklar
      6,10.5,  Persönliche Daten (Kreditkarten/Ausweise)
      5,11.2,  Manipulation Infrastruktur (Router/Switch/PC)
      4,11.9,  Verwendung Social Networks
      3,12.6,  Datendiebstahl/Datenverlust
      1,13.3,  Umgang mit Firmendokumenten
}\datatable

\begin{tikzpicture}

  \begin{axis}[
%    height = 5.5cm,
    xbar,
    y=-.7cm,
    enlarge y limits={abs=0.45cm},
    axis x line       = none,
    tickwidth         = 0pt,
    y axis line style = { opacity = 0 },
   yticklabels from table={\datatable}{2},
    ytick=data,
%    yticklabel style={text width=9cm,align=right},
    nodes near coords,
    nodes near coords align={horizontal},
%    nodes near coords={\pgfmathprintnumber\pgfplotspointmeta\%},
    ]
    \addplot table [y=1, x=0] {\datatable};
  \end{axis}
  
  % --
% -- draw surrounding box
% --
  \node[
      draw=black, very thin,
      minimum width=\textwidth,
      fit=(current bounding box.north west) (current bounding box.south east),
    ]at (current bounding box.center){};
    
\end{tikzpicture}