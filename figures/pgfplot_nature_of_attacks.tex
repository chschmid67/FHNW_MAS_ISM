% --
% -- datatable 2015
% --

\pgfplotstableread[col sep=comma, header=false]{
% -- <percent value>, <startpoint from above> , <label>
      68,0,    Phishing
      38,1,    Social Engineering
      59,2,    Malware
      14,3,    Advanced Persistent Threat APT
}\datatableA

% --
% -- datatable 2016
% --

\pgfplotstableread[col sep=comma, header=false]{
% -- <percent value>, <startpoint from above> , <label>
      84,0,    Phishing
      64,1,    Social Engineering
      80,2,    Malware
      24,3,    Advanced Persistent Threat APT
}\datatableB

\begin{tikzpicture}
  \begin{axis}[
%    height = 5.5cm,
    xbar,
    y=-1.1cm,
    enlarge y limits={abs=0.45cm},
    axis x line       = none,
    tickwidth         = 0pt,
    y axis line style = { opacity = 0 },
    yticklabels from table={\datatableA}{2},
    ytick=data,
    yticklabel style={text width=9cm,align=right},
    nodes near coords,
    nodes near coords align={horizontal},
    nodes near coords={\pgfmathprintnumber\pgfplotspointmeta\%},
    reverse legend,
    legend style={
        at={(-1.0,0.6)},
        font=\footnotesize \selectfont,
        anchor=north east,
        legend columns=1
        },
    ]
    \addplot table [y=1, x=0] {\datatableB};
    \addplot table [y=1, x=0] {\datatableA};
    
    \legend {2016, 2015}

  \end{axis}
  
% --
% -- draw surrounding box
% --
  \node[
      draw=black, very thin,
      minimum width=\textwidth,
      fit=(current bounding box.north west) (current bounding box.south east),
    ]at (current bounding box.center){};
    
\end{tikzpicture}


