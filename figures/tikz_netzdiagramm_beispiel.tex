%%
%% - Der verwwendete Code für die Darstellung der Grafik
%% - ist geistiges Eigentum von Dominik Renzel
%% - © 2009 
%% - http://dbis.rwth-aachen.de/cms/staff/renzel
%%
%%

\begin{tikzpicture}[scale=0.7]
  \path (0:0cm) coordinate (O); % define coordinate for origin

  % draw the spiderweb
  \foreach \X in {1,...,\D}{
    \draw (\X*\A:0) -- (\X*\A:\R);
  }

  \foreach \Y in {0,...,\U}{
    \foreach \X in {1,...,\D}{
      \path (\X*\A:\Y*\R/\U) coordinate (D\X-\Y);
      \fill (D\X-\Y) circle (1pt);
    }
    \draw [opacity=0.3] (0:\Y*\R/\U) \foreach \X in {1,...,\D}{
        -- (\X*\A:\Y*\R/\U)
    } -- cycle;
  }


  % for each sample case draw a path around the web along concrete values
  % for the individual dimensions. Each node along the path is labeled
  % with an identifier using the following scheme:
  %
  %   D<d>-<v>, dimension <d> a number between 1 and \D (#dimensions) and
  %             value <v> a number between 0 and \U (#scale units)
  %
  % The paths will be drawn half-opaque, so that overlapping parts will be
  % rendered in a composite color.

  % Example Case 2 (green)
  %
  D1 (Security): 4/9; D2 (Content Quality): 2/7; D3 (Performance): 5/7;
  D4 (Stability): 1/7; D5 (Usability): 4/7; D6 (Generality): 1/7;
  D7 (Popularity): 7/7
  \filldraw [color=blue, opacity=0.1,line width=1.5pt]
   (D1-7) --
    (D2-8) --
    (D3-9.5) --
    (D4-4) --
    (D5-8) --
    (D6-7) --
    (D7-9) -- cycle;

\end{tikzpicture}

