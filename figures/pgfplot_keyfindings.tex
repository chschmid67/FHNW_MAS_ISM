\pgfplotstableread[col sep=comma, header=false]{
% -- <percent value>, <startpoint from above> , <label>
      75,0,    Zustimmung dass die Geschäftsleitung Cyber-Gefahren als ein operationelles Risiko sieht
      64,1.5,  Das Unternehmen war schon Ziel von Social Engineering Attacken
      48,3,    Cyber Security ist mehrheitlich ein technisches Problem
      36,4,    Es gibt keinen Notfallplan für Cyber-Attacken
      19,5.5,  Die Mitarbeitenden sind sich der Cyber-Risiken genügend bewusst
}\datatable

\begin{tikzpicture}
  \begin{axis}[
%    height = 5.5cm,
    xbar,
    y=-.7cm,
    enlarge y limits={abs=0.45cm},
    axis x line       = none,
    tickwidth         = 0pt,
    y axis line style = { opacity = 0 },
    yticklabels from table={\datatable}{2},
    ytick=data,
    yticklabel style={text width=9cm,align=right},
    nodes near coords,
    nodes near coords align={horizontal},
    nodes near coords={\pgfmathprintnumber\pgfplotspointmeta\%},
    ]
    \addplot table [y=1, x=0] {\datatable};
  \end{axis}

% --
% -- draw surrounding box
% --
  \node[
      draw=black, very thin,
      minimum width=\textwidth,
      fit=(current bounding box.north west) (current bounding box.south east),
    ]at (current bounding box.center){};
    
\end{tikzpicture}

