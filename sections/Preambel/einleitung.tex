\documentclass[../../main.tex]{subfiles}

\begin{document}

%% - do not number the subsections, but put them in the TOC
%% - --------------------------------------

\subsubsection*{Sinn und Zweck dieser Arbeit}
\addcontentsline{toc}{subsubsection}{Sinn und Zweck dieser Arbeit}

\subsubsection*{Zielpublikum}
\addcontentsline{toc}{subsubsection}{Zielpublikum}

\subsubsection*{Abgrenzung}
\addcontentsline{toc}{subsubsection}{Abgrenzung}

\subsubsection*{Überblick Aufbau des Dokumentes}
\addcontentsline{toc}{subsubsection}{Überblick Aufbau des Dokumentes}

Die vorliegende Arbeit gliedert sich in die folgenden Teile:

\begin{sloppypar}
Der \textbf{erste Teil} zeigt auf, welche theoretischen Grundlagen für die "`Security Awareness NEXT GENERATION'' zu berücksichtigen sind...
\end{sloppypar}

\begin{sloppypar}
Mit dem \textbf{zweiten Teil} wird das Unternehmens selbst beschrieben...
\end{sloppypar}

\begin{sloppypar}
Der \textbf{dritte Teil} beinhaltet die Aufnahme der Ist-Situation bezüglich Security Awareness im Unternehmen...
\end{sloppypar}

\begin{sloppypar}
Der \textbf{vierte Teil} beinhaltet ...
\end{sloppypar}

\begin{sloppypar}
Der \textbf{fünfte Teil} beinhaltet ...
\end{sloppypar}


\begin{sloppypar}
Im \textbf{Sechsten und letzten Teil} wird...

\end{sloppypar}

\end{document}

