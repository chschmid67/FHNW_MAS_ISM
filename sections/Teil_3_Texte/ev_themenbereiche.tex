\documentclass[../../main.tex]{subfiles}

\begin{document}

\begin{sloppypar}
Die Auswahl der für die Untersuchung der Security Awareness relevanten Themenbereiche fokussiert sich primär auf das private (soziale) sowie das berufliche Umfeld. Sie werden für die spätere Datenauswertung zum Clustering der Antworten und zur visuellen Darstellung des Abdeckungsgrades / Ermittlung des Handlungsbedarfes des jeweiligen Themenbereiches verwendet.
\end{sloppypar}


%% - table metadata
%% - --------------------------------------

\sloppy 

\begin{table}[H]
\tablefontsize	
\centering
\caption{Themenbereiche}
\label{themenbereiche}

\begin{tabular}{ |p{4cm}|p{9.5cm}|p{2.5cm}|}

\hline
\tableheaderbgcolor
\textbf{Themenbereich} & \textbf{Fokus} & \textbf{Kürzel}\\ 
\hline
Private und \newline soziale Umfelder & Erhebung der Präventionsbereitschaft (d.h. die Bereitschaft, Aufklärungsarbeit zu leisten) in privaten sozialen Umfeldern wie der heimischen Umgebung, Familie, Partnerschaft, Verwandtschaft sowie Freundes- bzw. Bekanntenkreis. & SOCIAL\\
\hline
Privater Einsatz von\newline Tools, IT Kenntnisse & Erfassung über den Einsatz / Unterhalt von technischen Hilfsmitteln im privaten Umfeld zum Schutz der eigenen Daten und der eigenen IT-Infrastruktur. & KNOWHOW\\
\hline
Datenschutz & Erhebung des Bewusstseins gegenüber Datenschutz- und Verschlüsselungsthematiken. & PROTECT\\
\hline
Sicherheit am \newline Arbeitsplatz & Feststellung des allgemeinen Verständnisses im Umgang mit physischen oder logischen Autorisierungsmitteln des Arbeitsumfeldes. & WPLACE\\
\hline
Kenntnisse Angriffs- \newline und Bedrohungformen & Den Bewusstseins- / Bekanntheitsgrad für die derzeit gängigsten Angriffs- / Bedrohungsformen abfragen. Aufzeigen, ob aufklärende Informationen über diese Angriffsformen nötig sind. & ATTACKS\\
\hline
Sicherheit im \newline Unternehmen & Feststellung von persönlicher Verhaltensmustern, welche Auswirkungen auf die Sicherheit im Unternehmen haben können. Erkennung, ob mit für vom Unternehmen zur Verfügung gestellten persönlichen IT-Sachmitteln verantwortungsbewusst umgegangen wird. & COMPANY\\
\hline
Gerätesicherheit & Erfassung des Bewusstseins / der Sensibilität gegenüber Gerätesicherheit, bzw. der Bekanntheitsgrad der zur Zeit als "Best Practice" angesehenen (Gegen-) Massnahmen und Verhaltensweisen. & DEVICESEC\\
\hline

\end{tabular}
\end{table}

\end{document}
