\documentclass[../../main.tex]{subfiles}

\begin{document}

\paragraph*{Angewandte Methode}\mbox{}

\begin{sloppypar}
Für die quantitative Darstellung der Antwortauswertung wird auf das Netzdiagramm FN: auch bekannt unter Spinnennetzdiagramm, Radardiagramm oder Kiviat-Diagramm verwendet. Besonders gut eignet sich dieses Diagramm zum Visualisieren von Evaluationen für zuvor festgelegte Kriterien zweier (oder mehrerer) Serien \cite{wikipedia_netzdiagramm_2014}.
\end{sloppypar}

\paragraph*{Definition des Netzdiagrammes}

\subparagraph*{Skala und Semantik der Achsen}\mbox{}

\begin{sloppypar}
Für die Auswertung der Online-Fragen werden sieben Achsen verwendet. Jede dieser Achsen entspricht einem der zuvor   festgelegten Themenbereiche\footnote{Siehe Kapitel \ref{beschreibung_themenbereiche}, Seite \pageref{beschreibung_themenbereiche}. }. Die Skala einer einzelnen Achse reicht von 0\% bis 100\%, wobei der Schnittpunkt aller sieben Achsen als 0\% definiert wird.
\end{sloppypar}

\begin{figure}[H]
 \centering
    %%
%% - Der verwwendete Code für die Darstellung der Grafik
%% - ist geistiges Eigentum von Dominik Renzel
%% - © 2009 
%% - http://dbis.rwth-aachen.de/cms/staff/renzel
%%
%%

\begin{tikzpicture}[scale=1]
  \path (0:0cm) coordinate (O); % define coordinate for origin

  % draw the spiderweb
  \foreach \X in {1,...,\D}{
    \draw (\X*\A:0) -- (\X*\A:\R);
  }

  \foreach \Y in {0,...,\U}{
    \foreach \X in {1,...,\D}{
      \path (\X*\A:\Y*\R/\U) coordinate (D\X-\Y);
      \fill (D\X-\Y) circle (1pt);
    }
    \draw [opacity=0.3] (0:\Y*\R/\U) \foreach \X in {1,...,\D}{
        -- (\X*\A:\Y*\R/\U)
    } -- cycle;
  }

  % define labels for each dimension axis (names config option)
  \path (1*\A:\L+2mm) node (L1) {\scriptsize KNOWHOW};
  \path (2*\A:\L) node (L2) {\scriptsize SOCIAL};
  \path (3*\A:\L+5mm) node (L3) {\scriptsize DEVICE};
  \path (4*\A:\L+5mm) node (L4) {\scriptsize COMPANY};
  \path (5*\A:\L) node (L5) {\scriptsize ATTACK};
  \path (6*\A:\L+2mm) node (L6) {\scriptsize WPLACE};
  \path (7*\A:\L+5mm) node (L7) {\scriptsize DATASEC};

\end{tikzpicture}


 \caption{Netzdiagramm mit Themenbereichen als Achsen}
 \label{Netzdiagramm Schema}
\end{figure}

\subparagraph*{Zuordnung der Fragen zum Netzdiagramm}\mbox{}

\begin{sloppypar}
Die Fragen werden anhand des ihnen zugewiesenen Themenbereiches mit ihrer \acrshort{id}(\acrlong{id}) in das Diagramm übertragen.
\end{sloppypar}

\begin{figure}[H]
 \centering
    %%
%% - Der verwwendete Code für die Darstellung der Grafik
%% - ist geistiges Eigentum von Dominik Renzel
%% - © 2009 
%% - http://dbis.rwth-aachen.de/cms/staff/renzel
%%
%%

\begin{tikzpicture}[scale=1]
  \path (0:0cm) coordinate (O); % define coordinate for origin

  % draw the spiderweb
  \foreach \X in {1,...,\D}{
    \draw (\X*\A:0) -- (\X*\A:\R);
  }

  \foreach \Y in {0,...,\U}{
    \foreach \X in {1,...,\D}{
      \path (\X*\A:\Y*\R/\U) coordinate (D\X-\Y);
      \fill (D\X-\Y) circle (1pt);
    }
    \draw [opacity=0.3] (0:\Y*\R/\U) \foreach \X in {1,...,\D}{
        -- (\X*\A:\Y*\R/\U)
    } -- cycle;
  }

  % define labels for each dimension axis (names config option)
  \path (1*\A:\L+2mm) node (L1) {\scriptsize KNOWHOW};
  \path (2*\A:\L) node (L2) {\scriptsize SOCIAL};
  \path (3*\A:\L+5mm) node (L3) {\scriptsize DEVICE};
  \path (4*\A:\L+5mm) node (L4) {\scriptsize COMPANY};
  \path (5*\A:\L) node (L5) {\scriptsize ATTACK};
  \path (6*\A:\L+2mm) node (L6) {\scriptsize WPLACE};
  \path (7*\A:\L+5mm) node (L7) {\scriptsize DATASEC};

%% - Achse KNOWHOW
%% - --------------------------------------

\matrix [nodes={minimum size=5mm},column sep=1mm, row sep = 1mm] at (1*\A:\L+20mm)
{
\node [draw,scale=0.7,shape=circle, minimum width=3em] {S8}; & \node [draw,scale=0.7,shape=circle, cyan, minimum width=3em] {S9}; & \node [draw,scale=0.7,shape=circle, cyan, minimum width=3em] {S11};\\
\node [draw,scale=0.7,shape=circle, cyan, minimum width=3em] {S12}; & \node [draw,scale=0.7,shape=circle, minimum width=3em] {V31};\\
};

%% - Achse SOCIAL
%% - --------------------------------------

\matrix [nodes={minimum size=5mm},column sep=1mm, row sep = 1mm] at (2*\A:\L+15mm)
{
\node [draw,scale=0.7,shape=circle, minimum width=3em] {S13}; & \node [draw,scale=0.7,shape=circle, minimum width=3em] {S14}; & \node [draw,scale=0.7,shape=circle, minimum width=3em] {S15}; & \node [draw,scale=0.7,shape=circle, cyan, minimum width=3em] {U19}; \\
\node [draw,scale=0.7,shape=circle, cyan, minimum width=3em] {U24}; & \node [draw,scale=0.7,shape=circle, cyan, minimum width=3em] {U26}; & \node [draw,scale=0.7,shape=circle, cyan, minimum width=3em] {V32}; & \node [draw,scale=0.7,shape=circle, cyan, minimum width=3em] {V34};  \\
\\
};

%% - Achse DEVICE
%% - --------------------------------------

\matrix [nodes={minimum size=5mm},column sep=1mm, row sep = 1mm] at (3*\A-8:\L+20mm)
{
\node [draw,scale=0.7,shape=circle, cyan, minimum width=3em] {S9}; & \node [draw,scale=0.7,shape=circle, minimum width=3em] {S10}; & \node [draw,scale=0.7,shape=circle, cyan, minimum width=3em] {U18}; \\
\node [draw,scale=0.7,shape=circle, cyan, minimum width=3em] {U19}; & \node [draw,scale=0.7,shape=circle, cyan, minimum width=3em] {U20}; & \node [draw,scale=0.7,shape=circle, minimum width=3em] {V30};  \\
};

%% - Achse COMPANY
%% - --------------------------------------

\matrix [nodes={minimum size=5mm},column sep=1mm, row sep = 1mm] at (4*\A-18:\L+15mm)
{
\node [draw,scale=0.7,shape=circle, cyan, minimum width=3em] {U17}; & \node [draw,scale=0.7,shape=circle, cyan, minimum width=3em] {U18}; & \node [draw,scale=0.7,shape=circle, cyan, minimum width=3em] {U23};  \\
\node [draw,scale=0.7,shape=circle, cyan, minimum width=3em] {U24}; & \node [draw,scale=0.7,shape=circle, minimum width=3em] {U25}; & \node [draw,scale=0.7,shape=circle, cyan, minimum width=3em] {U26};  \\
};

%% - Achse ATTACK
%% - --------------------------------------

\matrix [nodes={minimum size=5mm},column sep=1mm, row sep = 1mm] at (5*\A:\L+15mm)
{
\node [draw,scale=0.7,shape=circle, cyan, minimum width=3em] {S9}; & \node [draw,scale=0.7,shape=circle, cyan, minimum width=3em] {S12}; & \node [draw,scale=0.7,shape=circle,cyan, minimum width=3em] {U19}; & \node [draw,scale=0.7,shape=circle, cyan, minimum width=3em] {U22};  \\
\node [draw,scale=0.7,shape=circle, cyan, minimum width=3em] {V28}; & \node [draw,scale=0.7,shape=circle, cyan, minimum width=3em] {V32}; & \node [draw,scale=0.7,shape=circle, minimum width=3em] {V33}; & \node [draw,scale=0.7,shape=circle, cyan, minimum width=3em] {V34};  \\
};

%% - Achse WPLACE
%% - --------------------------------------

\matrix [nodes={minimum size=5mm},column sep=1mm, row sep = 1mm] at (6*\A:\L+20mm)
{
\node [draw,scale=0.7,shape=circle, cyan, minimum width=3em] {S11}; & \node [draw,scale=0.7,shape=circle, cyan, minimum width=3em] {U17}; & \node [draw,scale=0.7,shape=circle, cyan, minimum width=3em] {U21};  \\
\node [draw,scale=0.7,shape=circle, cyan, minimum width=3em] {U22}; & \node [draw,scale=0.7,shape=circle, cyan, minimum width=3em] {U23}; & \node [draw,scale=0.7,shape=circle, cyan, minimum width=3em] {U24};  \\
};

%% - Achse DATASEC
%% - --------------------------------------

\matrix [nodes={minimum size=5mm},column sep=1mm, row sep = 1mm] at (7*\A+7:\L+15mm)
{
\node [draw,scale=0.7,shape=circle, cyan, minimum width=3em] {S11}; & \node [draw,scale=0.7,shape=circle, cyan, minimum width=3em] {U18}; & \node [draw,scale=0.7,shape=circle, cyan, minimum width=3em] {U19}; \\
};

%% - Achse DATASEC
%% - --------------------------------------

\matrix [nodes={minimum size=5mm},column sep=1mm, row sep = 1mm] at (7*\A-11:\L+15mm)
{
\node [draw,scale=0.7,shape=circle, cyan, minimum width=3em] {U20}; & \node [draw,scale=0.7,shape=circle, cyan, minimum width=3em] {U21}; & \node [draw,scale=0.7,shape=circle, cyan, minimum width=3em] {V28};  \\
\node [draw,scale=0.7,shape=circle, minimum width=3em] {V29}; \\
};

%% - Add Legend and draw box around it
%% - --------------------------------------

\matrix [column sep=1mm] at (-5.7,-4.3)
{
\node [draw,scale=0.3,shape=circle, cyan, minimum width=3em] {}; & \node [anchor=west] {\footnotesize Mehrfachverwendung}; \\
\node [draw,scale=0.3,shape=circle, minimum width=3em] {}; & \node [anchor=west] {\footnotesize Einmalverwendung}; \\
\node [draw,scale=0.3,fill=blue!10, minimum width=3em,minimum height=2em] {}; & \node [anchor=west] {\footnotesize Beispiel Auswertung}; \\
};
% - draw enclosing box
\draw (-8,-5.2) rectangle (-3.5,-3.3);

%% - draw sample data area
%% - --------------------------------------

%\filldraw [fill=blue, draw=black, opacity=0.05 ] 
\filldraw [opacity=0.3, fill=blue!20!, draw=black ] 
% - KNOWHOW        SOCIAL
(1*\A:\L-50) -- (2*\A:\L-90) --
% - DEVICE        COMPANY
(3*\A:\L-30) -- (4*\A:\L-85) -- 
% - ATTACK        WPLACE
(5*\A:\L-45) -- (6*\A:\L-25) -- 
% - DATASEC
(7*\A:\L-95) -- cycle;

\end{tikzpicture}


 \caption{Netzdiagramm mit den zugeordneten Fragen}
 \label{Netzdiagramm Schema fragepositionen}
\end{figure}

\paragraph*{Beispielauswertung}\mbox{}

\begin{sloppypar}
\blindtext
\end{sloppypar}

\begin{figure}[H]
 \centering
    %%
%% - Der verwwendete Code für die Darstellung der Grafik
%% - ist geistiges Eigentum von Dominik Renzel
%% - © 2009 
%% - http://dbis.rwth-aachen.de/cms/staff/renzel
%%
%%

\begin{tikzpicture}[scale=1]
  \path (0:0cm) coordinate (O); % define coordinate for origin

  % draw the spiderweb
  \foreach \X in {1,...,\D}{
    \draw (\X*\A:0) -- (\X*\A:\R);
  }

  \foreach \Y in {0,...,\U}{
    \foreach \X in {1,...,\D}{
      \path (\X*\A:\Y*\R/\U) coordinate (D\X-\Y);
      \fill (D\X-\Y) circle (1pt);
    }
    \draw [opacity=0.3] (0:\Y*\R/\U) \foreach \X in {1,...,\D}{
        -- (\X*\A:\Y*\R/\U)
    } -- cycle;
  }

  % define labels for each dimension axis (names config option)
  \path (1*\A:\L+2mm) node (L1) {\scriptsize KNOWHOW};
  \path (2*\A:\L) node (L2) {\scriptsize SOCIAL};
  \path (3*\A:\L+5mm) node (L3) {\scriptsize DEVICE};
  \path (4*\A:\L+5mm) node (L4) {\scriptsize COMPANY};
  \path (5*\A:\L) node (L5) {\scriptsize ATTACK};
  \path (6*\A:\L+2mm) node (L6) {\scriptsize WPLACE};
  \path (7*\A:\L+5mm) node (L7) {\scriptsize DATASEC};

  % for each sample case draw a path around the web along concrete values
  % for the individual dimensions. Each node along the path is labeled
  % with an identifier using the following scheme:
  %
  %   D<d>-<v>, dimension <d> a number between 1 and \D (#dimensions) and
  %             value <v> a number between 0 and \U (#scale units)
  %
  % The paths will be drawn half-opaque, so that overlapping parts will be
  % rendered in a composite color.

  % Example Case 1 (red)
  %
  % D1 (Security): 0/7; D2 (Content Quality): 5/7; D3 (Performance): 0/7;
  % D4 (Stability): 6/7; D5 (Usability): 0/7; D6 (Generality): 5/7;
  % D7 (Popularity): 0/7
  % \draw [color=red,line width=1.5pt,opacity=0.5]
  %  (D1-0) --
  %  (D2-5) --
  %  (D3-0) --
  %  (D4-6) --
  %  (D5-0) --
  %  (D6-5) --
  %  (D7-0) -- cycle;

  % Example Case 2 (green)
  %
  % D1 (Security): 2/7; D2 (Content Quality): 2/7; D3 (Performance): 5/7;
  % D4 (Stability): 1/7; D5 (Usability): 4/7; D6 (Generality): 1/7;
  % D7 (Popularity): 7/7
  % \draw [color=green,line width=1.5pt,opacity=0.5]
  % (D1-2) --
  %  (D2-2) --
  %  (D3-5) --
  %  (D4-1) --
  %  (D5-4) --
  %  (D6-1) --
  %  (D7-7) -- cycle;

  % Example Case 3 (blue)
  %
  % D1 (Security): 1/7; D2 (Content Quality): 7/7; D3 (Performance): 4/7;
  % D4 (Stability): 4/7; D5 (Usability): 3/7; D6 (Generality): 5/7;
  % D7 (Popularity): 2/7
  % \draw [color=blue,line width=1.5pt,opacity=0.5]
  %  (D1-1) --
  %  (D2-7) --
  %  (D3-4) --
  %  (D4-4) --
  %  (D5-3) --
  %  (D6-5) --
  %  (D7-2) -- cycle;

\end{tikzpicture}


 \caption{Beispiel eines Netzdiagrammes}
 \label{Netzdiagramm Schema beispiel}
\end{figure}

\end{document}