\documentclass[../../main.tex]{subfiles}

\begin{document}

\paragraph*{Grundstruktur Interviews}\mbox{}

\begin{sloppypar}
Die Grundproblematik bei solchen komplett freien Interview besteht in der Nichtvorhersehbarkeit der Gesprächsentwicklung und der dementsprechend schwierigen Auswertung der Antworten. Aus diesem Grund werden die Interviews als halbstrukturierte, "'Freie Interviews"` durchgeführt. Dies ermöglicht es, dass sich die Themenwahl während des Gesprächsverlaufs entwickelt und so weitere, aus der persönlichen Wahrnehmung des Probanden wichtige Aspekte von Security Awareness zur Sprache kommen können.

Der halbstrukturierte Teil besteht darin, die Gesprächsentwicklung an den ebenfalls in der Online-Umfrage vorkommenden offen formulierten Kernfragen zu orientieren:

\begin{itemize}
  \item Was bedeutet Security Awareness für Dich?
  \item Was fehlt / was braucht es Deiner Meinung nach bezüglich Security Awareness?
  \item Was würdest Du von einer Security Awareness Kampagne erwarten?
\end{itemize}


Die Interviews werden (nach vorheriger Einverständniserklärung) akkustisch aufgezeichnet. Die Aufzeichnung wird jedoch für die spätere Auswertung nicht transkribiert, sondern sinngemäss vom Interviewer zusammengefasst und elektronisch festgehalten.

An persönlichen Daten werden analog der Online-Umfrage die Rolle, Anzahl Jahre in der Firma, Anzahl Jahre in der Branche, höchster Schulabschluss und die Herkunftsregion erhoben.

Die Grundstruktur der Interviews besteht aus drei Frageblöcken. Pro Frageblock gibt es eine auf den Frageblock abgestimmte Stichwortliste, eine Liste für zusätzliche, aus dem Interview entnommenen Stichworten und einem Ideenspeicher. Diese Elemente werden nun nachstehend kurz beschrieben.
\end{sloppypar}

\paragraph*{Frageblock}\mbox{}

\begin{sloppypar}
Die im Frageblock gestellte Kernfrage entspricht exakt derjenigen aus der Online-Umfrage.
\end{sloppypar}

\paragraph*{Stichworteliste}\mbox{}

\begin{sloppypar}
Bei der Auswertung wird pro Kernfrage die Antwort anhand einer vordefinierten Stichwortliste qualifiziert. (Qualifikationskriterium = wurde genannt / wurde nicht genannt).
\end{sloppypar}

\paragraph*{Zusätzliche Stichworteliste}\mbox{}

\begin{sloppypar}
Um nicht vordefinierte (aber trotzdem genannte) Stichwörter abzudecken, werden diese bei der Auswertung aus der Antwort extrahiert und in einer zusätzlichen, offenen Stichworteliste hinterlegt.
\end{sloppypar}

\paragraph*{Ideenspeicher}\mbox{}

\begin{sloppypar}
Da bei einem freien Interview immer die Möglichkeit besteht, dass sehr konkrete Vorschläge oder Ideen formuliert werden, können diese im Ideenspeicher als zusammengefasster Freitext hinterlegt werden
\end{sloppypar}


\end{document}