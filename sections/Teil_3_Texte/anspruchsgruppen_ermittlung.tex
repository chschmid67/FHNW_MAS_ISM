\paragraph*{Definition der Schutzbedarfskategorien}\mbox{}

\begin{sloppypar}
Die nachfolgend beschriebenen Schutzbedarfskategorien enstammen dem BSI.

%% abkürzungsverzeichnis, BibliograFie

\end{sloppypar}

%% - table metadata
%% - --------------------------------------

\begin{table}[H]
\centering
\caption{Schutzbedarfsklassen}
\label{schutzbedarfsklassen}

%% - set width of columns
%% - ---------------------------------------

\begin{tabular}{ |p{4cm}|p{6cm}|{c}| }

%% - Header row with shadowing
%% - --------------------------------------

\hline
\rowcolor[HTML]{C0C0C0} 
\textbf{Schutzbedarfsklasse} & \textbf{Ausprägung} & \textbf{Schutzbedarf-Faktor}\\ 
\hline

%% - table data
%% - --------------------------------------

\textbf{Tief} & Normaler Schutzbedarf & 1 \\ \hline
\textbf{Mittel} & Erhöhter Schutzbedarf & 2  \\ \hline
\textbf{Hoch} & Höchster Schutzbedarf & 3 \\ \hline

\end{tabular}
\end{table}

\paragraph*{Berechnung des Schutzbedarfes}\mbox{}

\begin{sloppypar}
Für die festgelegten zugreifbaren Datenbestände der Anspruchsgruppen wird deren Schutzbedarf als Produkt von Vertraulichkeit und Integrität berechnet. Dieses Produkt wird zur Selektion der Anspruchsgruppen verwendet. Die Festlegung der Vertraulichkeits-, respektive der Integritätsklasse für die Datenbestände basiert auf einer Erst-Einschätzung welche vom Autor vorgenommen wurde; diese wurde aber im Verlauf dieser Arbeit anhand der durchgeführten Interviews kontinuierlich nachkalibriert.
\end{sloppypar}

%% - table metadata
%% - --------------------------------------

\begin{table}[H]
\centering
\caption{Schutzbedarfsberechnung Datenbestände }
\label{schutzbedarf_datenbestände}

%% - set width of columns
%% - ---------------------------------------

\begin{tabular}{ |l|l|{c}|{c}|{c}| }

%% - Header row with shadowing
%% - --------------------------------------

\hline
\rowcolor[HTML]{C0C0C0} 
\textbf{Anspruchsgruppe} & \textbf{Zugreifbare Datenbestände} & \textbf{Vertraulichkeit} & \textbf{Integrität} & \textbf{Schutzbedarf}\\ 
\hline

%% - table data
%% - --------------------------------------

\textbf{Produktentwicklung} & Quellcode & Mittel & Hoch & 6\\ \hline
\textbf{Gebäudemanagement} & Zutrittskontrolle & Mittel & Mittel & 4  \\ \hline
\textbf{Marketing} & Firmenidentifikation & Mittel & Mittel & 4\\ \hline
\textbf{Einkauf} & Lieferantendaten + Offerten & Tief & Tief & 1\\ \hline
\textbf{Kommunkation} & Interne Mitarbeiterinformationen & Mittel & Mittel & 4\\ \hline
\textbf{Personaldienst} & Personaldossiers & Hoch & Hoch & 9 \\ \hline
\textbf{Verkauf} & Produkteofferten & Hoch & Mittel & 6\\ \hline
\textbf{Verkaufsanbahnung} & Anonymisierte Produktdaten  & Mittel & Tief & 2\\ \hline
\textbf{Finanzen} & Buchhaltungsdaten & Hoch & Hoch & 9\\ \hline
\textbf{Qualitätssicherung} & Verfahrensbeschriebe  & Tief & Mittel & 2\\ \hline
\textbf{Rechtsabteilung} & Verträge & Hoch & Hoch & 9\\ \hline
\textbf{IT} &  Systempasswörter & Hoch & Hoch & 9\\ \hline
\textbf{Schulung} & Prüfungsfragen & Mittel & Hoch & 6\\ \hline

\end{tabular}
\end{table}

\newpage

\paragraph*{Definition des Rastermodells}\mbox{}

\begin{sloppypar}
Nachfolgend wird der berechnete Schutzbedarf für die Anspruchsgruppen anhand eines Rastermodells dargestellt. Es wird die Selektionslinie bestimmt, anhand welcher die definitiven Anspruchsgruppen für die Datenerhebung festgelegt werden. 
\end{sloppypar}

\begin{sloppypar}
Das Rastermodell für die Bestimmung der Anspruchsgruppen wird als eine Matrix mit 3 x 3 Elementen aufgebaut. Die horizontale Dimension der Matrix ist aus der Integritätsklasse, die vertikale Dimension der Matrix aus der Vertraulichkeitsklasse abgeleitet.
\end{sloppypar}

\subparagraph*{Festlegung der Selektionslinie}\mbox{}

\begin{sloppypar}
Die Selektionslinie wir im Modell wie folgt definiert: 2 < x, wobei x = berechneter Schutzbedarf
\end{sloppypar}

\begin{figure}[H]
    \centering
    \begin{tikzpicture}

%% - Draw grid
%% - --------------------------------------

\draw[step=2cm,gray,very thin] (0,0) grid (6,6);

%% - Draw Grid Ticks
%% - --------------------------------------

\foreach \x in {2,4}
    \draw (\x cm,3pt) -- (\x cm,-3pt);

\foreach \y in  {2,4}
    \draw (3pt,\y cm) -- (-3pt,\y cm);

%% - Draw Arrows
%% - --------------------------------------

\draw [decoration={markings,mark=at position 1 with
    {\arrow[scale=3,>=stealth]{>}}},postaction={decorate}] (0,0) -- (6,0);

\draw [decoration={markings,mark=at position 1 with
    {\arrow[scale=3,>=stealth]{>}}},postaction={decorate}] (0,0) -- (0,6);
    
%% - label x- and y-Axis
%% - --------------------------------------

\node[scale=1.2] at (8,0) {Vertraulichkeit};
\node[scale=1.2] at (0,6.5) {Integrität};

%% - Draw x- and y- Axis scale
%% - --------------------------------------

\node[draw,scale=1,shape=rectangle,draw=none, ] at (1,-0.5) {Tief (1)};
\node[draw,scale=1,shape=rectangle,draw=none] at (3,-0.5) {Mittel (2)};
\node[draw,scale=1,shape=rectangle,draw=none] at (5,-0.5) {Hoch (3)};

\node[draw,scale=1,shape=rectangle,draw=none, rotate=90] at (-0.5,1) {Tief (1)};
\node[draw,scale=1,shape=rectangle,draw=none, rotate=90] at (-0.5,3) {Mittel (2)};
\node[draw,scale=1,shape=rectangle,draw=none, rotate=90] at (-0.5,5) {Hoch (3)};

%% - Draw visible nodes in grid
%% - --------------------------------------

\node[draw,scale=2,shape=rectangle,draw=none,gray] at (1,1) {1};
\node[draw,scale=2,shape=rectangle,draw=none,gray] at (3,1) {2};
\node[draw,scale=2,shape=rectangle,draw=none,gray] at (5,1) {3};

\node[draw,scale=2,shape=rectangle,draw=none,gray] at (1,3) {2};
\node[draw,scale=2,shape=rectangle,draw=none,gray] at (3,3) {4};
\node[draw,scale=2,shape=rectangle,draw=none,gray] at (5,3) {6};

\node[draw,scale=2,shape=rectangle,draw=none,gray] at (1,5) {3};
\node[draw,scale=2,shape=rectangle,draw=none,gray] at (3,5) {6};
\node[draw,scale=2,shape=rectangle,draw=none,gray] at (5,5) {9};

%% - Draw selection line and label
%% - --------------------------------------

\draw [red, ultra thick, dashed] (-1,4) -- (2,4) -- (2,2) -- (4,2) -- (4,0);

\node[draw,scale=1,shape=rectangle,draw=none, red] at (-2.5,4) {Selektionslinie};


\end{tikzpicture}
    \caption{Raster für Bestimmung der Anspruchsgruppen}
    \label{fig:raster1}
\end{figure}

\newpage

\paragraph*{Anwendung des Rastermodells}\mbox{}

\blindtext

\begin{figure}[H]
    \begin{tikzpicture}

%% - Draw grid
%% - --------------------------------------

\draw[step=2cm,gray,very thin] (0,0) grid (6,6);

%% - Draw Grid Ticks
%% - --------------------------------------

\foreach \x in {2,4}
    \draw (\x cm,3pt) -- (\x cm,-3pt);

\foreach \y in  {2,4}
    \draw (3pt,\y cm) -- (-3pt,\y cm);

%% - Draw Arrows
%% - --------------------------------------

\draw [decoration={markings,mark=at position 1 with
    {\arrow[scale=3,>=stealth]{>}}},postaction={decorate}] (0,0) -- (6,0);

\draw [decoration={markings,mark=at position 1 with
    {\arrow[scale=3,>=stealth]{>}}},postaction={decorate}] (0,0) -- (0,6);
    
%% - label x- and y-Axis
%% - --------------------------------------

\node[scale=1.0] at (7.5,0) {Vertraulichkeit};
\node[scale=1.0] at (0,6.5) {Integrität};

%% - Draw x- and y- Axis scale
%% - --------------------------------------

\node[draw,scale=1,shape=rectangle,draw=none, ] at (1,-0.5) {Tief};
\node[draw,scale=1,shape=rectangle,draw=none] at (3,-0.5) {Mittel};
\node[draw,scale=1,shape=rectangle,draw=none] at (5,-0.5) {Hoch};

\node[draw,scale=1,shape=rectangle,draw=none, rotate=90] at (-0.5,1) {Tief};
\node[draw,scale=1,shape=rectangle,draw=none, rotate=90] at (-0.5,3) {Mittel};
\node[draw,scale=1,shape=rectangle,draw=none, rotate=90] at (-0.5,5) {Hoch};

%% - Draw visible nodes in grid
%% - --------------------------------------

%\node[draw,scale=2,shape=rectangle,draw=none,gray] at (1,1) {1};
%\node[draw,scale=2,shape=rectangle,draw=none,gray] at (3,1) {2};
%\node[draw,scale=2,shape=rectangle,draw=none,gray] at (5,1) {3};

%\node[draw,scale=2,shape=rectangle,draw=none,gray] at (1,3) {2};
%\node[draw,scale=2,shape=rectangle,draw=none,gray] at (3,3) {4};
%\node[draw,scale=2,shape=rectangle,draw=none,gray] at (5,3) {6};

%\node[draw,scale=2,shape=rectangle,draw=none,gray] at (1,5) {3};
%\node[draw,scale=2,shape=rectangle,draw=none,gray] at (3,5) {6};
%\node[draw,scale=2,shape=rectangle,draw=none,gray] at (5,5) {9};

%% - Draw selection line and label
%% - --------------------------------------

\draw [red, ultra thick, dashed] (0,4) -- (2,4) -- (2,2) -- (4,2) -- (4,0);
%\node[draw,scale=1,shape=rectangle,draw=none, red] at (-2.5,4) {Selektionslinie};

%% - Add datapoints
%% - --------------------------------------

\node[draw,shape=circle,cyan,very thick,minimum width=2em] at (3.5,4.5) {A};
\node[draw,shape=circle,cyan,very thick,minimum width=2em] at (2.5,3.5) {B};
\node[draw,shape=circle,cyan,very thick,minimum width=2em] at (3.5,3.0) {C};
\node[draw,shape=circle,red,very  thick,minimum width=2em] at (1.0,1.0) {D};
\node[draw,shape=circle,cyan,very thick,minimum width=2em] at (2.5,2.5) {E};
\node[draw,shape=circle,cyan,very thick,minimum width=2em] at (5.5,5.5) {F};
\node[draw,shape=circle,cyan,very thick,minimum width=2em] at (5.0,3.0) {G};
\node[draw,shape=circle,red,very  thick,minimum width=2em] at (3.0,1.0) {H};
\node[draw,shape=circle,cyan,very thick,minimum width=2em] at (4.5,4.5) {I};
\node[draw,shape=circle,red,very  thick,minimum width=2em] at (1.0,3.0) {J};
\node[draw,shape=circle,cyan,very thick,minimum width=2em] at (4.5,5.5) {K};
\node[draw,shape=circle,cyan,very thick,minimum width=2em] at (5.5,4.5) {L};
\node[draw,shape=circle,cyan,very thick,minimum width=2em] at (2.5,5.5) {M};

%% - legend
%% - --------------------------------------

\matrix  [matrix of nodes,row sep=1mm,column 1/.style={nodes={circle,scale=0.5,draw,minimum width=1em}},column 2/.style={right,font=\footnotesize}] at (13,3)
{
A   & Produktentwicklung \\
B   & Gebäudemanagement \\
C   & Marketing \\
D   & Einkauf \\
E   & Kommunkation \\
F   & Personaldienst \\
G   & Verkauf \\
H   & Verkaufsanbahnung \\
I   & Finanzen \\
J   & Qualitätssicherung \\
K   & Rechtsabteilung \\
L   & IT \\
M   & Schulung \\
};


\end{tikzpicture}

    \caption{Anwendung des Rasters auf Anspruchsgruppen}
    \label{fig:raster2}
\end{figure}

\newpage

\paragraph*{Selektion der definitiven Anspruchsgruppen}\mbox{}

\begin{sloppypar}
Nach der Anwendung des Rasters stehen die nachfolgend aufgeführten Anspruchsgruppen als Basis für die Erhebung der Zielpopulation.
\end{sloppypar}

%% - table metadata
%% - --------------------------------------

\begin{table}[H]
\centering
\caption{Definitive Anspruchsgruppen}
\label{definitive_anspruchsgruppen}

%% - set width of columns
%% - ---------------------------------------

\begin{tabular}{ |p{8cm}| }

%% - Header row with shadowing
%% - --------------------------------------

\hline
\rowcolor[HTML]{C0C0C0} 
\textbf{Definitive Anspruchsgruppe} \\ 
\hline

%% - table data
%% - --------------------------------------

\textbf{Produktentwicklung}             \\ \hline
\textbf{Gebäudemanagement}              \\ \hline
\textbf{Marketing}                      \\ \hline
\textbf{Kommunkation}                   \\ \hline
\textbf{Personaldienst}                 \\ \hline
\textbf{Verkauf}                        \\ \hline
\textbf{Finanzen}                       \\ \hline
\textbf{Rechtsabteilung}                \\ \hline
\textbf{IT}                             \\ \hline
\textbf{Schulung}                       \\ \hline

\end{tabular}
\end{table}

\begin{sloppypar}
Die Anspruchsgruppen "`Einkauf"', "`Verkaufsanbahnung"' und "`Qualitätssicherung"' wurden aufgrund ihrer Position unterhalb der Selektionslinie vom weiteren Verfahren ausgeschlossen.
\end{sloppypar}