\documentclass[../../main.tex]{subfiles}

\begin{document}

\begin{sloppypar}
Für das empirische Vorgehen wird eine Doppelstrategie bestehend aus einer Kombination von Online-Umfrage und Interviews / Diskussionen angewandt.
\end{sloppypar}

\subparagraph*{Online-Umfrage}\mbox{}

\begin{sloppypar}
Um eine standardisierte Erfassung von Antworten zu ermöglichen, wird eine Online-Umfrage mit einem auf die zentralen Fragestellungen ausgerichteten Fragenkatalog durchgeführt. Jede Frage ist einem bestimmten Themenbereichen zugeordnet. Durch die Auswertung soll der Abdeckungsgrad für den jeweiligen Themenbereich bestimmt und dargestellt werden.

Der Katalog wird im Rahmen dieser Arbeit entwickelt. Die Beantwortung der Online-Umfrage sollte nicht mehr als 10 Minuten in Anspruch nehmen.

Das Ziel der Online-Umfrage ist es, anhand 
\end{sloppypar}

\subparagraph*{Interview / Diskussion}\mbox{}

\begin{sloppypar}
Als Ergänzung zu den Onlinefragen finden zusätzliche Interviews mit einigen Probanden statt. Diese Interviews sind als Ergänzung zu der Online-Umfrage zu verstehen. Als Leitlinie werden diejenigen Fragen aus der Online-Umfrage als Diskussionsgrundlage verwendet, zu denen eine persönliche Meinung abgefragt wird, z.B. "Was bedeutet für Dich Security Awareness?".

Die Interviews werden als offene Diskussion geführt bei welcher das Ergebnis wesentlich durch die Gesprächsentwicklung beeinflusst wird. Die Interviews werden nicht transkribiert, sondern jeweils die Kernaussagen in einem Kurzprotokoll festgehalten. Diese Kernaussagen fliessen ebenfalls in die Arbeit mit ein. Eine solche Diskussion / Interview sollte nicht länger als 30 Minuten dauern.
\end{sloppypar}

\end{document}
