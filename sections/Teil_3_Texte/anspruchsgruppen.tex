\begin{sloppypar}
Für die Datenerhebung werden die folgenden potentiellen Anspruchsgruppen mit den von ihnen hauptsächlich zugreifbaren Datenbeständen definiert. Für diese Datenbestände wird anhand ihres Integritäts- und Vertraulichkeitsanspruches der jeweilige Schutzbedarf eingeschätzt um damit diejenigen Anspruchsgruppen zu bestimmen, die bei der nachfolgenden Datenerhebung berücksichtigt werden müssen.
\end{sloppypar}

%% - table metadata
%% - --------------------------------------

\begin{table}[H]
\centering
\caption{Potentielle Anspruchsgruppen}
\label{potentielle_anspruchsgruppen}

%% - set width of columns
%% - ---------------------------------------

\begin{tabular}{ |l|l| }

%% - Header row with shadowing
%% - --------------------------------------

\hline
\rowcolor[HTML]{C0C0C0} 
\textbf{Anspruchsgruppe} & \textbf{Zugreifbare Datenbestände}\\ 
\hline

%% - table data
%% - --------------------------------------

\textbf{Produktentwicklung} & Quellcode                     \\ \hline
\textbf{Gebäudemanagement} & Zutrittskontrolle              \\ \hline
\textbf{Marketing} & Firmenidentifikation                   \\ \hline
\textbf{Einkauf} & Offerten                                 \\ \hline
\textbf{Kommunkation} & Interne Mitarbeiterinformationen    \\ \hline
\textbf{Personaldienst} & Personaldossiers                  \\ \hline
\textbf{Verkauf} & Produkteofferten                         \\ \hline
\textbf{Verkaufsanbahnung} & Anonymisierte Produktdaten     \\ \hline
\textbf{Finanzen} & Buchhaltungsdaten                       \\ \hline
\textbf{Qualitätssicherung} & Verfahrensbeschriebe          \\ \hline
\textbf{Rechtsabteilung} & Verträge                         \\ \hline
\textbf{IT} &  Systempasswörter                             \\ \hline
\textbf{Schulung} & Prüfungsfragen                          \\ \hline

\end{tabular}
\end{table}

