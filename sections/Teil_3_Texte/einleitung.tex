\documentclass[../../main.tex]{subfiles}

\begin{document}

\begin{sloppypar}
In diesem dritten Teil wird beschrieben, an welchen Anspruchsgruppen die Security Awareness Ist-Situation im Unternehmen mittels einer Online-Umfrage und persönlichen Interviews untersucht wird. Neben der allgemeinen Vorgehensbeschreibung wird die Zielpopulation und die Stichprobe definiert. Ein einem weiteren Schritt werden die zur Erhebung der Ist-Situation relevanten Themenbereiche aufgeführt und begründet.

Anschliessend werden die Bausteine der Umfrage beschrieben (Grundstruktur, Fragebogenabschnitte und Antworttypen). Die detaillierten Fragen des Fragenkataloges werden definiert, die Absicht der Frage wird deklariert und mit den zuvor definierten Antworttypen und Themenbereichen verknüpft.

Für die persönlichen Befragungen wird die Interviewform und deren Auswertungsmethode beschrieben. 
\end{sloppypar}

\end{document}

