\paragraph*{Allgemeine Vorgehensbeschreibung}\mbox{}

Für das empirische Vorgehen wird eine Doppelstrategie angewendet:

\subparagraph*{Online-Umfrage}\mbox{}

\begin{sloppypar}
Um eine standardisierte Erfassung von Antworten zu ermöglichen, wird eine Online-Umfrage mit einem auf die zentralen Fragestellungen ausgerichteten Fragenkatalog durchgeführt. Der Fragenkatalog wird im Rahmen dieser Arbeit entwickelt. Die Beantwortung der Online-Umfrage sollte nicht mehr als 10 Minuten in Anspruch nehmen.
\end{sloppypar}

\subparagraph*{Interview / Diskussion}\mbox{}

\begin{sloppypar}
Als Ergänzung zu den Onlinefragen finden zusätzliche Interviews mit einigen Probanden statt. Diese Interviews sind als Ergänzung zu der Online-Umfrage zu verstehen. Als Leitlinie werden diejenigen Fragen aus der Online-Umfrage als Diskussionsgrundlage verwendet, zu denen eine persönliche Meinung abgefragt wird, z.B. "Was bedeutet für Dich Security Awareness?".

Die Interviews werden als offene Diskussion geführt bei welcher das Ergebnis wesentlich durch die Gesprächsentwicklung beeinflusst wird. Die Interviews werden nicht transkribiert, sondern jeweils die Kernaussagen in einem Kurzprotokoll festgehalten. Diese Kernaussagen fliessen ebenfalls in die Arbeit mit ein. Eine solche Diskussion / Interview sollte nicht länger als 30 Minuten dauern.
\end{sloppypar}

\paragraph*{Beschreibung der Zielpopulation (Grundgesamtheit) und der Stichprobe}\mbox{}

\subparagraph*{Zielpopulation}\mbox{}

\begin{sloppypar}
Als Zielpopulation der Empirie werden potentiell alle internen Mitarbeitenden an allen Standorten der Aquina AG genannt, welche in den selektierten Anspruchsgruppen beschäftigt sind.
\end{sloppypar}

\subparagraph*{Teilerhebung}\mbox{}

\begin{sloppypar}
Die Empirie kann aufgrund von firmeninternen Jahresziel-Verschiebungen nicht an der ursprünglich geplanten Zielpopulation (n ~ 1200) durchgeführt werden, sondern an nur an einer wesentlich kleineren Stichprobe (n <= 50). Die in dieser Teilerhebung befragten Probanden wurden vorselektioniert. Die Stichprobe ist aufgrund der Auswahlkriterien (Funktionsausübung, kultureller Herkunftshintergrund) jedoch als repräsentativ anzusehen.
\end{sloppypar}

\subparagraph*{Stichprobenselektivität}\mbox{}

\begin{sloppypar}
Stichprobenausfälle (d.h. das Nichtreagieren auf den Fragebogen bei der Erhebung der Daten) wird bewusst in Kauf genommen. Dieser partielle Antwortausfall kann zu einer Schweigeverzerrung der Auswertung führen, welche bewusst nicht verringert wird. 
\end{sloppypar}

\paragraph*{Beschreibung und Zweck der Themenbereiche}\mbox{}

\begin{sloppypar}

\end{sloppypar}


\paragraph*{Bausteine der Online-Umfrage}\mbox{}

\begin{sloppypar}

\end{sloppypar}


\paragraph*{Allgemeine Vorgehensbeschreibung}\mbox{}

\begin{sloppypar}

\end{sloppypar}
