\documentclass[../../main.tex]{subfiles}

\begin{document}

\paragraph*{Zielpopulation}\mbox{}

\begin{sloppypar}
Als Zielpopulation der Empirie werden potentiell alle internen Mitarbeitenden an allen Standorten der Aquina AG genannt, welche in den selektierten Anspruchsgruppen beschäftigt sind.
\end{sloppypar}

\paragraph*{Teilerhebung}\mbox{}

\begin{sloppypar}
Die Empirie kann aufgrund von firmeninternen Jahresziel-Verschiebungen nicht an der ursprünglich geplanten Zielpopulation ($n \approx 1200$) durchgeführt werden, sondern an nur an einer wesentlich kleineren Stichprobe ($n \leq 50$). Die in dieser Teilerhebung befragten Probanden wurden vorselektioniert. Die Stichprobe ist aufgrund der Auswahlkriterien (Funktionsausübung, kultureller Herkunftshintergrund) jedoch als repräsentativ anzusehen.
\end{sloppypar}

\paragraph*{Stichprobenselektivität}\mbox{}

\begin{sloppypar}
Stichprobenausfälle (d.h. das Nichtreagieren auf den Fragebogen bei der Erhebung der Daten) kann zu einer Schweigeverzerrung bei der Auswertung führen. Dieser partielle Antwortausfall , wird bewusst in Kauf genommen. 
\end{sloppypar}

\end{document}
