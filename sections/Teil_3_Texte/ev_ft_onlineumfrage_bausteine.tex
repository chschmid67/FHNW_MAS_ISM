\documentclass[../../main.tex]{subfiles}

\begin{document}
\paragraph*{Grundstruktur Onlinebefragung}\mbox{}

\begin{sloppypar}
Die Grundstruktur der Onlinebefragung setzt sich aus mehreren Umfrageabschnitten zusammen. Jeder dieser Umfrageabschnitte beinhaltet einen Fragensatz. Um die Mehrsprachigkeit und somit auch die Interkulturalität abzudecken, wird die Online-Umfrage in Deutsch und Englisch zur Verfügung gestellt. 

\begin{itemize}
  \item Allgemeine Angaben
  \item Eigene Meinung zu Security Awareness
  \item Security Awareness im privaten Umfeld
  \item Security Awareness im Unternehmen
  \item Kenntnisse zu Sicherheitsthemen
  \item Erwartungshaltung Security Awareness
\end{itemize}

\end{sloppypar}


\paragraph*{Beschreibung der Umfrageabschnitte}\mbox{}

%% - table metadata
%% - --------------------------------------

\sloppy 

\begin{table}[H]
\tablefontsize	
\centering
\caption{Beschreibung der Umfrageabschnitte}
\label{Beschreibung der Umfrageabschnitte}
\begin{tabular}{ |p{3cm}|p{12.5cm}|}

\hline
\tableheaderbgcolor
\textbf{Umfragebereiche} & \textbf{Inhaltsbeschreibung} \\ 

\hline
\textbf{Allgemeine \newline Angaben}                &  Mit dem ersten Abschnitt (Untertitel: "`Allgemeine Angaben zu Dir selber"') sollen bei der späteren Auswertung der Fragen mögliche Korrelationen zwischen Rollen, Abteilungen, Ausbildungen, kulturelle Hintergründe aufgezeigt werden können. \\

\hline
\textbf{Eigene Meinung zu \newline Security Awareness}                &  Im zweiten Abschnitt (Untertitel: "`Deine Meinung zu Security Awareness"') soll die befragte Person den Begriff "Security Awareness" sowohl aus ihrer persönlichen Perspektive wie auch aus ihrer eigenen Erfahrung heraus beschreiben.  \\

\hline
\textbf{Security Awareness \newline im privaten Umfeld}                &  Der dritte Abschnitt (Untertitel: "`Security bei Dir zu Hause und in Deinem privaten Umfeld"') dient dazu, das Sicherheitsverhalten der befragten Person in ihrer heimischen / familiären Umgebung und in ihrem Verwandten- / Freundeskreis zu erfragen. Die Absicht ist, damit sicherndes Verhalten vom sozialen Umfeld in die Firmenumgebung zu transportieren. \\

\hline
\textbf{Security Awareness \newline im Unternehmen}                &  In diesem Abschnitt (Untertitel: "`Deine Beurteilung von Security Awareness im Unternehmen"') geht es darum, die aktuelle Wahrnehmung der sichernden Elemente in der Firma (sicherheitsbeauftragte Personen, Verfahren bei Sicherheitsvorfällen oder Stand des Bewusstseins im Umgang mit Firmendaten) abzufragen. \\

\hline
\textbf{Kenntnisse zu \newline Sicherheitsthemen}                &  Dieser Abschnitt (Untertitel: "`Wie schätzst Du Deine Kenntinsse zu den folgenden Sicherheitsthemen ein?"') soll nicht als Prüfungsblock für Wissen / nicht Wissen wahrgenommen werden, sondern es soll der befragten Person die Möglichkeit geboten werden, zu aktuellen Sicherheitsthemen / Bedrohungsformen ein eventuelles Informationsbedürfnis anzuzeigen. \\

\hline
\textbf{Erwartungshaltung \newline Security Awareness}                &  Der letzte Abschnitt (Untertitel: "`Beschreibe Deine Erwartungshaltung bezüglich Security Awareness"') ist als Sammelgefäss zu verstehen, in welches die befragten Personen ihre Anregungen, Ideen, Wünsche oder durch ihre Berufstätigkeit an anderen Orten bereits gesehene Security Awareness Massnahmen anregen können. \\
\hline

\end{tabular}
\end{table}

\begin{sloppypar}
Grundsätzlich wird zwischen offenen und geschlossenen Fragen unterschieden. Offene Fragen geben für ihre Beantwortung keine Antwortstruktur vor; geschlossene Fragen haben eine vordefinierte Antwortstruktur (Bereichsangabe, Ja / Nein Entscheidung, etc.)

Die Fragen lassen sich einem (oder in einigen Fällen auch weiteren) der zuvor definierten Themenbereichen zuordnen. Die offen formulierten Fragen stellen hierbei einen Sonderfall dar. Da sich die Antworten auf diese Fragen naturgemäss nicht direkt Auswerten lassen, ist ein Zwischenschritt notwendig, welcher die Antworten zuerst analysiert. Aus diesem Grund werden die Antworten auf die offenen zusammen mit den Interviews ausgewertet\footnotemark.
 
Hinter jeder Frage der Umfrage-Bereiche steht eine Absicht, d.h. eine Erklärung dafür, was mit der Frage erreicht / im tieferen Sinne abgefragt werden soll. Diese Absichtserklärung ist bei der jeweils entsprechenden Frage hinterlegt. 

Zwecks besserer Umsetzbarkeit und Lesbarkeit der Umfrage werden nachfolgend die Antworttypen definiert und diese dann jeweils einer Frage zugeordnet.

\end{sloppypar}

\footnotetext {Für eine genauere Beschreibung des Vorgehens siehe Kapitel \ref{Auswertungsmethode}, Seite \pageref{Auswertungsmethode}.}

\paragraph*{Definition der Antworttypen}\mbox{}

\begin{sloppypar}
Die Antworttypen dienen der Standardisierung des Fragebogens und sollen die spätere Auswertung erleichtern.

\end{sloppypar}

%% - table metadata
%% - --------------------------------------

\sloppy 

\begin{table}[H]
\tablefontsize	
\centering
\caption{Definition der Antworttypen}
\label{Definition der Antworttypen}
\begin{tabular}{ |p{2cm}|p{2.5cm}|p{6cm}|p{4.5cm}|}

\hline
\tableheaderbgcolor
\textbf{Antworttyp} & \textbf{Verwendet bei} & \textbf{Beschreibung} & \textbf{Kardinalität}\\ 

\hline
\textbf{Typ A} & Geschlossene Frage &  Einschätzungsfrage mit unscharfer \newline Bereichsangabe & "`Trifft überhaupt nicht zu"' \newline bis \newline "`Trifft voll zu"' \\

\hline
\textbf{Typ B} & Geschlossene Frage & Dieser Antworttyp wird bei Fragen verwendet, die mit einer klaren Auswahl (Beispielsweise Ja oder Nein) beantwortet werden können & Ja \newline Nein \newline unbekannt oder sagt mir nichts \\

\hline
\textbf{Typ C} & Offene Frage &  Dieser Antworttyp ermöglicht der befragten Person ihre Antwort frei zu formulieren. & Unstrukturierte, Freie Textantwort \\
\hline

\end{tabular}
\end{table}

%% - read questionnaire
%% - --------------------------------------

\subfile{sections/Teil_3_Texte/ev_ft_online_fragenkatalog}


\end{document}







