\documentclass[../../main.tex]{subfiles}

\begin{document}


%% - table metadata
%% - --------------------------------------

\sloppy 

\begin{table}[H]
\tablefontsize	
\centering
\caption{Stichwortliste Leitfragen für Experteninterview}
\label{Stichwortliste Leitfragen Experteninterview}
\begin{tabular}{ |p{4.0cm}|p{5.7cm}|p{5.7cm}|c|}

\hline
\tableheaderbgcolor
\textbf{Leitfrage} & \textbf{Vordefinierte Stichworte} & \textbf{Weitere Stichworte} & \textbf{ID} \\ 
\hline
Was bedeutet der Begriff "`Security Awareness"' für Dich persönlich? & \framebox(5,5){} hat mit mir zu tun \newline  \framebox(5,5){} ist eine Zeitverschwendung\newline \framebox(5,5){} Es gibt verbindliche Richtlinien\newline \framebox(5,5){} Kenntnisse von Angriffsformen\newline \framebox(5,5){} Betrifft die physische Sicherheit\newline \framebox(5,5){} ist wichtig\newline \framebox(5,5){} betrifft persönliche Verhaltensweisen\newline \framebox(5,5){} Anwendung von Best Practices\newline \framebox(5,5){} Kentnisse im Umgang mit Passwörtern\newline \framebox(5,5){} Clean Desk Policy & & F7 \\
\hline

Was würdest Du bei Dir zu Hause bezüglich Security als die grösste Risikoquelle identifizieren? & \framebox(5,5){} Lebenspartner \newline  \framebox(5,5){} WLAN\newline \framebox(5,5){} Umgang mit Email\newline \framebox(5,5){} Ungepatchte Geräte\newline \framebox(5,5){} WLAN Gastnetzwerk\newline \framebox(5,5){} Meine Kinder\newline \framebox(5,5){} Meine persönlichen Verhaltensweisen\newline \framebox(5,5){} Ungepatchte Software\newline \framebox(5,5){} Mangelnde Kenntnisse IT Technologien\newline \framebox(5,5){} Anzahl Geräte mit Internetzugriff & & S16 \\
\hline

Was würdest Du im Unternehmen bezüglich Security als die grösste Risikoquelle identifizieren? & \framebox(5,5){} Fremde Geräte im Netz \newline  \framebox(5,5){} WLAN\newline \framebox(5,5){} Umgang mit Email\newline \framebox(5,5){} Die anderen Arbeitskollegen\newline \framebox(5,5){} Unklare (Sicherheits-) Richtlinien\newline \framebox(5,5){} USB-Geräte (Sticks,Harddisks, etc.)\newline \framebox(5,5){} Mobile persönliche Geräte (Handy, etc.)\newline \framebox(5,5){} Ungepatchte Software\newline \framebox(5,5){} Mangelnde Kenntnisse IT Technologien\newline \framebox(5,5){}  Fehlende Clean Desk Policy & & U27\\
\hline

Was fehlt / was braucht es Deiner Meinung nach bezüglich Security Awareness? & \framebox(5,5){} Security Awareness Kampagne \newline  \framebox(5,5){} Sicherheitsbeauftrager als Person bekannt\newline \framebox(5,5){} Persönliches Bewusstsein fördern\newline \framebox(5,5){} (Regelmässige) Kommunikation\newline \framebox(5,5){} Notfallübung durchführen\newline \framebox(5,5){} Klare Richtlinien\newline \framebox(5,5){} Vorbildverhalten des Managements\newline \framebox(5,5){} Praxisnahe Integration in Arbeitsalltag\newline \framebox(5,5){} Informationsveranstaltungen & & F35\\
\hline

Was  erwartest  Du (inhaltlich, resultatmässig) von  einer  Security Awareness Kampagne? & \framebox(5,5){} Wiederkehrende Veranstaltung \newline  \framebox(5,5){} Nichts, ist eine Zeitverschwendung\newline \framebox(5,5){} Poster / Plakate\newline \framebox(5,5){} Live Hacking Event\newline \framebox(5,5){} Physiche Sicherheit diskutieren\newline \framebox(5,5){} Security Blog / Twitter\newline \framebox(5,5){} Giveaways\newline \framebox(5,5){} (Regelmässiger) Newsletter\newline \framebox(5,5){} Erzeugung von sicheren Passwörtern\newline \framebox(5,5){}  Integration in Firmenveranstaltungen & & F36\\
\hline

\end{tabular}
\end{table}

\end{document}
