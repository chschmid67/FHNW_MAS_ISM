\documentclass[../../main.tex]{subfiles}

\begin{document}

\subsubsection*{Begleitschreiben Online-Umfrage}
\addcontentsline{toc}{subsubsection}{Begleitschreiben Online-Umfrage}
\label{begleitschreiben_online_umfrage}

\begin{sloppypar}

{\fontfamily{cmtt}\selectfont
\footnotesize

(German translation see below)

Welcome to the Survey on ``Security Awareness''!

What is it all about? Nowadays, Security Awareness is an absolute necessity in both private as well as in business environments. In times of blackmail trojans that encrypt your data completely and perfectly faked spam emails which you encourage you to disclose your account information, it is more than just important that " security awareness" becomes an integral part of your daily thinking in order to protect both you and the company from harm.

To achieve this, you can contribute personally with your responses to this survey. The survey is completely anonymous, the results will be published at a later date. You can participate in the survey from July 4 to July 18, 2016. The Survey takes approximately 15 minutes to complete.

\underline{Click here for the survey}

Thank you for your participation!

================================================

Willkommen zur Meinungsumfrage betreffend ``Security Awareness''!

Worum geht es? ``Security Awareness'' oder ``Sicherheitsbewusstsein'' ist sowohl im privaten wie auch im geschäflichen Umfeld eine unbedingte Notwendigkeit. In den Zeiten von Erpressungstrojanern welche Deine Daten verschlüsseln und perfekt gefälschten Spam-E-Mails welche Dich dazu auffordern Deine Kontoinformationen preiszugeben, ist es mehr als nur wichtig, dass ``Security Awareness'' ein fester Teil Deines Denkens wird und Dich und das Unternehmen schützt.

Dazu kannst Du jezt persönlich mit Deinen Antworten zu dieser Umfrage beitragen. Die Umfrage ist komplett anonym, die Resultate werden zu einem späteren Zeitpunkt publiziert. Du kannst an der Umfrage vom 4. bis 18. Juli 2016 teilnehmen. Die Beantwortung der Fragen benötigt etwa 15 Minuten.

\underline{Hier geht es zu der Umfrage}

Herzlichen Dank, dass Du Dir die Zeit nimmst um in der Meinungsumfrage ``Security Awareness'' mitzumachen!

}
\normalsize
\end{sloppypar}

\newpage

\subsubsection*{Nachfassung Online-Umfrage}
\addcontentsline{toc}{subsubsection}{Nachfassung Online-Umfrage}
\label{nachfassung_online_umfrage}

\begin{sloppypar}

{\fontfamily{cmtt}\selectfont
\footnotesize

(German translation see below)

Dear valued colleagues,

On July 18 around midnight CEST, the survey on Security Awareness is going to be closed. Thanks to your contribution, we already received more than 230 responses which already is a remarkable score. Nevertheless, we would be very pleased receiving more results. Therefore we kindly ask those of you who did not already participate in the survey to use the link provided below and provide us with your insight to the topic. 

Security Awareness in some aspects is like this survey: It starts with you!

\underline{Click here for the survey}

Best wishes and thank you for your participation!                                                                                                               

================================================

Werte Kolleginnen und Kollegen,

Am 18. Juli um Mitternacht MESZ endet die Umfrage zum Thema Security Awareness. Dank eurer Teilnahme haben wir bereits mehr als 230 Antworten erhalten, was bereits ein beachtenswertes Resultat ist. Trotzdem würden wir uns über weitere Antworten sehr freuen. Daher möchten wir diejenigen welche noch nicht mitgemacht haben freundlich darum bitten, via untenstehenden Link uns ihre Meinung zum Thema mitzuteilen. 

Mit Security Awareness ist es ein wenig wie mit dieser Umfrage: Es beginnt bei Dir!

\underline{Hier geht es zu der Umfrage}

Beste Grüsse und herzlichen Dank für eure Teilnahme!

}
\normalsize
\end{sloppypar}



\end{document}
