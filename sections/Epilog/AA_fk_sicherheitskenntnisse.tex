\documentclass[../../main.tex]{subfiles}

\begin{document}

%% - table metadata
%% - --------------------------------------

\sloppy 

\begin{table}[H]
\tablefontsize	
\centering
\caption{Online-Fragenkatalog: Kenntnisse zu Sicherheitsthemen}
\label{Kenntnisse zu Sicherheitsthemen}
\begin{tabular}{ |p{5.5cm}|p{5.5cm}|p{2.5cm}|c|c|}

\hline
\tableheaderbgcolor
\textbf{Fragestellung} & \textbf{Absicht} & \textbf{Themenbereich} & \textbf{Typ} & \textbf{ID}\\ 
\hline

Ich kann erkennen, ob der Datenaustausch mit einem Webservice via Browser verschlüsselt ist & Weiss die Person, dass bei gesicherten Verbindungen ein Schloss-Symbol im Browser angezeigt wird und kennt sie  den Unterschied zwischen "`http"' und "`https"' in der URL Zeile des Browsers? & ATTACK \newline DATASEC & B & V28 \\
\hline

E-Mail ist eine einfache und sichere Methode um sensitive Informationen geschützt zu übertragen & Wird E-Mail von der Person als bedenkenlos zu verwendendes Hilfsmittel angesehen und entsprechend eingesetzt? & DATASEC & A & V29 \\
\hline

Wenn ich ein Speichermedium formatiere, können die gelöschten Daten nicht wiederhergestellt werden & Weiss die Person, dass es sichere und unsichere Datenlöschverfahren gibt? & DEVICE & B & V30 \\
\hline

Ich kann erläutern, was eine "`Zwei-Faktor"' Authentifizierung ist & Hat die Person ein Bewusstsein zu den Faktoren "`Weiss ich"' (Passwort) und "`Habe ich"' (Token)? & KNOWHOW & A & V31 \\
\hline

Ich weiss, woran man eine "`Phishing"' E-Mail erkennen kann & Ist sich die Person der Phishing-Thematik bewusst? & ATTACK \newline SOCIAL & A & V32 \\
\hline

Ich kann erklären, was "`Ransomware"' ist & Ist diese Angriffsform bekannt? & ATTACK & A & V33 \\
\hline

Ich weiss, was "`Social Engineering"' bedeutet & Ist diese Angriffsform bekannt? & ATTACK \newline SOCIAL & A & V34 \\
\hline

\end{tabular}
\end{table}

\end{document}
