\documentclass[../../main.tex]{subfiles}

\begin{document}

%% - table metadata
%% - --------------------------------------

\sloppy 

\begin{table}[H]
\tablefontsize	
\centering
\caption{Online-Fragenkatalog: Allgemeine Angaben}
\label{Allgemeine Angaben}
\begin{tabular}{ |p{8.5cm}|p{5.5cm}|c|c|}

\hline
\tableheaderbgcolor
\textbf{Fragestellung} & \textbf{Kardinalität} & \textbf{Typ} & \textbf{ID}\\ 
\hline
Welches ist Deine Rolle im Unternehmen? &  Teammitglied \newline Teamleitung \newline Management & B & P1 \\

\hline
%In welchem Bereich\tablefootnote{Gemäss der in Kapitel \ref{erhebung_der_anspruchsgruppen} selektierten Anspruchsgruppen. } arbeitest Du? &  Geschäftsleitung\newline Produktentwicklung\newline Admin- und Gebäudemanagement\newline Marketing\newline Kommunikation\newline Personaldienst\newline Verkauf\newline Finanzen\newline Rechtsdienst\newline IT\newline Schulung\newline Kundensupport & B & P2 \\

In welchem Bereich\tablefootnote{Gemäss der in Kapitel \ref{erhebung_der_anspruchsgruppen} selektierten Anspruchsgruppen. } arbeitest Du? &  Academy\newline Admin- und Gebäudemanagement\newline Geschäftsleitung\newline IT\newline Konsultation \& Implementationssupport\newline Kundensupport\newline Marketing, Kommunikation \& Strategie\newline Personaldienst\newline Projektmanagement \& Business Analyse\newline Rechtsdienst \& Finanzen\newline Services\newline Softwareentwicklung\newline Verkauf  & B & P2 \\

\hline

Seit wann arbeitest Du für das Unternehmen? & Angabe Anzahl Jahre & B & P3 \\
\hline

Seit wann arbeitest Du in Deiner Branche? & Angabe Anzahl Jahre & B & P4 \\
\hline

Welches ist Dein höchster Schulabschluss?\tablefootnote{Abschlussbezeichnungen in Anlehnung an \citep{bundesamt_fur_statistik_bfs_statistik_2015}.} & Kein Abschluss \newline Berufslehre \newline Höhere Berufsbildung \newline Hochschule (Fachhochschule, Universität) & B & P5 \\

\hline

In welcher Region der Welt\tablefootnote{Regionenbezeichnungen gemäss \citep{united_nations_united_2013}.} bist Du aufgewachsen? & Afrika \newline Amerika (Nord- Mittel- Süd-, Karibik)\newline Asien \newline Europa \newline Ozeanien & B & P6 \\

\hline

\end{tabular}
\end{table}

\end{document}
