\documentclass[../../main.tex]{subfiles}

\begin{document}

\subsection*{KPMG Studie: Clarity on Cyber Security (2016)}
\label{kpmg_study_2016}
\addcontentsline{toc}{subsection}{KPMG Studie: Clarity on Cyber Security (2016)}

Die Studie zeigt auch deutlich auf, dass sich vor allem im Bereich des Social Engineering innerhalb eines Jahres eine Verdoppelung der registrierten Angriffe ereignet hat. Die Unternehmen machten zu der Frage, mit welchen Angriffsformen sie innerhalb der Jahre 2015 / 2016 schon angegriffen worden sind, die folgenden Angaben:

%% -
%% - keep this block together!!!!!!!!!!!!!!
%% - --------------------------------------
%% -
\addtocounter{figure}{1}
\begin{figure}[H]
    % --
% -- datatable 2015
% --

\pgfplotstableread[col sep=comma, header=false]{
% -- <percent value>, <startpoint from above> , <label>
      68,0,    Phishing
      38,1,    Social Engineering
      59,2,    Malware
      14,3,    Advanced Persistent Threat APT
}\datatableA

% --
% -- datatable 2016
% --

\pgfplotstableread[col sep=comma, header=false]{
% -- <percent value>, <startpoint from above> , <label>
      84,0,    Phishing
      64,1,    Social Engineering
      80,2,    Malware
      24,3,    Advanced Persistent Threat APT
}\datatableB

\begin{tikzpicture}
  \begin{axis}[
%    height = 5.5cm,
    xbar,
    y=-1.1cm,
    enlarge y limits={abs=0.45cm},
    axis x line       = none,
    tickwidth         = 0pt,
    y axis line style = { opacity = 0 },
    yticklabels from table={\datatableA}{2},
    ytick=data,
    yticklabel style={text width=9cm,align=right},
    nodes near coords,
    nodes near coords align={horizontal},
    nodes near coords={\pgfmathprintnumber\pgfplotspointmeta\%},
    reverse legend,
    legend style={
        at={(-1.0,0.6)},
        font=\footnotesize \selectfont,
        anchor=north east,
        legend columns=1
        },
    ]
    \addplot table [y=1, x=0] {\datatableB};
    \addplot table [y=1, x=0] {\datatableA};
    
    \legend {2016, 2015}

  \end{axis}
  
% --
% -- draw surrounding box
% --
  \node[
      draw=black, very thin,
      minimum width=\textwidth,
      fit=(current bounding box.north west) (current bounding box.south east),
    ]at (current bounding box.center){};
    
\end{tikzpicture}



    \caption*{Abbildung \thefigure: Angriffsformen auf Organisationen; aus: \citeauthor{bossart_clarity_2016} (\citeyear{bossart_clarity_2016}), S. 35.}
    \label{fig:natureofattacks}
\end{figure}
\addcontentsline{lof}{figure}{\numberline {\thefigure}{\ignorespaces Angriffsformen auf Organisationen}}
%% -
%% - keep this block together!!!!!!!!!!!!!!
%% - --------------------------------------
%% -

Die Auswirkungen der aus den vorgehend vorgestellten tatsächlich stattgefundenen Cyber Angriffen auf das operationelle Geschäft zeigten sich in diesen Bereichen:

%% -
%% - keep this block together!!!!!!!!!!!!!!
%% - --------------------------------------
%% -
\addtocounter{figure}{1}
\begin{figure}[H]
    \pgfplotstableread[col sep=comma, header=false]{
% -- <percent value>, <startpoint from above> , <label>
      44,0,  Unterbrechung der Geschäftsprozesse
      36,1,  Finanzieller Verlust
      28,2,  Abfluss von internen Informationen
      24,3,  Reputations-Schaden
      16,4,  Abfluss von vertraulichen Partner- oder Kundendaten
      16,5,  Unerlaubte Weitergabe von persönlichen Daten
      12,6,  Datenmanipulation
}\datatable

\begin{tikzpicture}
  \begin{axis}[
%    height = 5.5cm,
    xbar,
    y=-.7cm,
    enlarge y limits={abs=0.45cm},
    axis x line       = none,
    tickwidth         = 0pt,
    y axis line style = { opacity = 0 },
    yticklabels from table={\datatable}{2},
    ytick=data,
    yticklabel style={text width=9cm,align=right},
    nodes near coords,
    nodes near coords align={horizontal},
    nodes near coords={\pgfmathprintnumber\pgfplotspointmeta\%},
    ]
    \addplot table [y=1, x=0] {\datatable};
  \end{axis}

% --
% -- draw surrounding box
% --
  \node[
      draw=black, very thin,
      minimum width=\textwidth,
      fit=(current bounding box.north west) (current bounding box.south east),
    ]at (current bounding box.center){};
    
\end{tikzpicture}

    \vspace*{-5mm}
    \caption*{Abbildung \thefigure: Auswirkungen der Cyber-Angriffe; aus: \citeauthor{bossart_clarity_2016} (\citeyear{bossart_clarity_2016}), S. 43.}
    \label{fig:attackconsequences}
\end{figure}
\addcontentsline{lof}{figure}{\numberline {\thefigure}{\ignorespaces Auswirkungen der Cyber-Angriffe}}
%% -
%% - keep this block together!!!!!!!!!!!!!!
%% - --------------------------------------
%% -

Auf die Frage, mit welchen Mitteln oder Strategien der Cyber-Bedrohung erfolgreich engegengetreten werden solle, wurde vor allem die Koordination von Menschen, Prozessen und Technologien klar favorisiert:

%% -
%% - keep this block together!!!!!!!!!!!!!!
%% - --------------------------------------
%% -
\addtocounter{figure}{1}
\begin{figure}[H]
    \pgfplotstableread[col sep=comma, header=false]{
% -- <percent value>, <startpoint from above> , <label>
      75,0,    Koordination von Menschen -- Prozessen -- Technologien
      49,1,    Faktor Mensch
      34,2,    Technologien
      28,3,    Geschäftsprozesse
}\datatable

\begin{tikzpicture}
  \begin{axis}[
%    height = 5.5cm,
    xbar,
    y=-.7cm,
    enlarge y limits={abs=0.45cm},
    axis x line       = none,
    tickwidth         = 0pt,
    y axis line style = { opacity = 0 },
    yticklabels from table={\datatable}{2},
    ytick=data,
    yticklabel style={text width=9cm,align=right},
    nodes near coords,
    nodes near coords align={horizontal},
    nodes near coords={\pgfmathprintnumber\pgfplotspointmeta\%},
    ]
    \addplot table [y=1, x=0] {\datatable};
  \end{axis}
  
% --
% -- draw surrounding box
% --
  \node[
      draw=black, very thin,
      minimum width=\textwidth,
      fit=(current bounding box.north west) (current bounding box.south east),
    ]at (current bounding box.center){};
    
\end{tikzpicture}

    \vspace*{-5mm}
    \caption*{Abbildung \thefigure: Umgang mit Cyber-Bedrohungen; aus: \citeauthor{bossart_clarity_2016} (\citeyear{bossart_clarity_2016}), S. 48.}
    \label{fig:managecyberthreat}
\end{figure}
\addcontentsline{lof}{figure}{\numberline {\thefigure}{\ignorespaces Umgang mit Cyber-Bedrohungen}}
%% -
%% - keep this block together!!!!!!!!!!!!!!
%% - --------------------------------------
%% -

Die Untersuchung zeigt auch auf, welche der als risikomindernd angesehenen Massnahmen von den Unternehmen entweder als ganz fehlend oder teilweise fehlend eingestuft werden:

%% -
%% - keep this block together!!!!!!!!!!!!!!
%% - --------------------------------------
%% -
\addtocounter{figure}{1}
\begin{figure}[H]
    \pgfplotstableread[col sep=comma, header=false]{
% -- <percent value>, <startpoint from above> , <label>
      60,0,  Technische Überwachung verdächtiger Aktivitäten
      51,1,  Datenklassifizierung
      49,2,  Interdisziplinäre Koordination 
      40,3.2,  Ausrichtung der Geschäftsprozesse anhand der Cyber-Risiken
      38,4.4,  Massnahmen im Personaldienst 
      34,5.4,  Verbessertes Zugriffsmanagement
      15,6.4,  Firmenweite Risiko Governance
}\datatable

\begin{tikzpicture}
  \begin{axis}[
%    height = 5.5cm,
    xbar,
    y=-.7cm,
    enlarge y limits={abs=0.45cm},
    axis x line       = none,
    tickwidth         = 0pt,
    y axis line style = { opacity = 0 },
    yticklabels from table={\datatable}{2},
    ytick=data,
    yticklabel style={text width=9cm,align=right},
    nodes near coords,
    nodes near coords align={horizontal},
    nodes near coords={\pgfmathprintnumber\pgfplotspointmeta\%},
    ]
    \addplot table [y=1, x=0] {\datatable};
  \end{axis}

% --
% -- draw surrounding box
% --
  \node[
      draw=black, very thin,
      minimum width=\textwidth,
      fit=(current bounding box.north west) (current bounding box.south east),
    ]at (current bounding box.center){};
    
\end{tikzpicture}

    \vspace*{-5mm}
    \caption*{Abbildung \thefigure: Fehlende risikomindernde Massnahmen; aus: \citeauthor{bossart_clarity_2016} (\citeyear{bossart_clarity_2016}), S. 48.}
    \label{fig:missingmeasures}
\end{figure}
\addcontentsline{lof}{figure}{\numberline {\thefigure}{\ignorespaces Fehlende risikomindernde Massnahmen}}
%% -
%% - keep this block together!!!!!!!!!!!!!!
%% - --------------------------------------
%% -

Im weiteren wurden von den befragten Unternehmen die nachfolgenden Aussagen gemacht, wovon einige (mit einer Ausnahme) die schon fast klassischen IT-Klischees bedienen.

%% -
%% - keep this block together!!!!!!!!!!!!!!
%% - --------------------------------------
%% -
\addtocounter{figure}{1}
\begin{figure}[H]
    \pgfplotstableread[col sep=comma, header=false]{
% -- <percent value>, <startpoint from above> , <label>
      75,0,    Zustimmung dass die Geschäftsleitung Cyber-Gefahren als ein operationelles Risiko sieht
      64,1.5,  Das Unternehmen war schon Ziel von Social Engineering Attacken
      48,3,    Cyber Security ist mehrheitlich ein technisches Problem
      36,4,    Es gibt keinen Notfallplan für Cyber-Attacken
      19,5.5,  Die Mitarbeitenden sind sich der Cyber-Risiken genügend bewusst
}\datatable

\begin{tikzpicture}
  \begin{axis}[
%    height = 5.5cm,
    xbar,
    y=-.7cm,
    enlarge y limits={abs=0.45cm},
    axis x line       = none,
    tickwidth         = 0pt,
    y axis line style = { opacity = 0 },
    yticklabels from table={\datatable}{2},
    ytick=data,
    yticklabel style={text width=9cm,align=right},
    nodes near coords,
    nodes near coords align={horizontal},
    nodes near coords={\pgfmathprintnumber\pgfplotspointmeta\%},
    ]
    \addplot table [y=1, x=0] {\datatable};
  \end{axis}

% --
% -- draw surrounding box
% --
  \node[
      draw=black, very thin,
      minimum width=\textwidth,
      fit=(current bounding box.north west) (current bounding box.south east),
    ]at (current bounding box.center){};
    
\end{tikzpicture}


    \caption*{Abbildung \thefigure: Wesentliche Aussagen KPMG Studie; aus: \citeauthor{bossart_clarity_2016} (\citeyear{bossart_clarity_2016}), S. 33 ff.}
    \label{fig:keyfindings}
\end{figure}
\addcontentsline{lof}{figure}{\numberline {\thefigure}{\ignorespaces Wesentliche Aussagen KPMG Studie}}
%% -
%% - keep this block together!!!!!!!!!!!!!!
%% - --------------------------------------
%% -

\end{document}