\documentclass[../../main.tex]{subfiles}

\begin{document}

%% - table metadata
%% - --------------------------------------

\sloppy 

\begin{table}[H]
\tablefontsize	
\centering
\caption{Online-Fragenkatalog: Eigene Meinung zu Security Awareness}
\label{Eigene Meinung zu Security Awareness}
\begin{tabular}{ |p{5.5cm}|p{5.5cm}|p{2.5cm}|c|c|}

\hline
\tableheaderbgcolor
\textbf{Fragestellung} & \textbf{Absicht} & \textbf{Themenbereich} & \textbf{Typ} & \textbf{ID}\\ 
\hline
Was bedeutet der Begriff "`Security Awareness"' für Dich persönlich? &  Die Frage zielt darauf ab, stereotype abzufragen und (in leichter Form) die Erwartungshaltung zu bedienen. Diese Frage wird bewusst am Anfang des Fragebogens gestellt, um der befragten Person das Interesse an ihrer persönlichen Sicht der Dinge zu signalisieren. & nicht bestimmbar,\newline Auswertung erfolgt mit Interviews & C & F7 \\
\hline

\end{tabular}
\end{table}

%% - table metadata
%% - --------------------------------------

\sloppy 

\begin{table}[H]
\tablefontsize	
\centering
\caption{Online-Fragenkatalog: Erwartungshaltung Security Awareness}
\label{Erwartungshaltung Security Awareness}
\begin{tabular}{ |p{5.5cm}|p{5.5cm}|p{2.5cm}|c|c|}

\hline
\tableheaderbgcolor
\textbf{Fragestellung} & \textbf{Absicht} & \textbf{Themenbereich} & \textbf{Typ} & \textbf{ID}\\ 
\hline
Was fehlt / was braucht es Deiner Meinung nach bezüglich Security Awareness? & Die Person soll auf ihrer Meinung nach fehlende Massnahmen für Security Awareness aufmerksam machen. & nicht bestimmbar,\newline Auswertung erfolgt mit Interviews & C & F35 \\
\hline

Was erwartest Du von einer Security Awareness Kampagne? & Die Person soll ihre Erwartungshaltung bezüglich einer Security Awareness Kampagne formulieren. & nicht bestimmbar,\newline Auswertung erfolgt mit Interviews & C & F36 \\
\hline

\end{tabular}
\end{table}

\end{document}
