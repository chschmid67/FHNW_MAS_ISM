\documentclass[../../main.tex]{subfiles}

\begin{document}

\newpage
\subsection*{MELANI: Halbjahresberichte}
\label{melani_halbjahresberichte}
\addcontentsline{toc}{subsection}{MELANI: Halbjahresberichte}

\begin{sloppypar}

%% - table metadata
%% - --------------------------------------

\sloppy 

\begin{longtable}[c]{ |p{4.0 cm}|p{4.5cm}|p{8.0cm}|}
 
\hline
\tableheaderbgcolor
\textbf{Lagebericht} & \textbf{Thema} & \textbf{Beschreibung} \\ 
\endfirsthead
 
\hline
\tableheaderbgcolor
\textbf{Lagebericht} & \textbf{Thema} & \textbf{Beschreibung} \\ 
\endhead

\endfoot
 
\endlastfoot

\textbf{Halbjahresbericht 2015/2} & Cyberspionage & Cyberspionage gegen schweizerische Interessen ist eine Realität. In den vergangenen MELANI Halbjahresberichten wurde bereits über verschiedene Fälle berichtet. Auch der Jahresbericht des Nachrichtendienstes des Bundes (NDB) gibt einen Überblick über die Situation. Dabei ist Prävention eine wichtige, wenn nicht die wichtigste, Komponente im Kampf gegen Spionage. Hierbei ist der erste und wichtigste Schritt für ein Unternehmen. die Erkenntnis, dass es sich um eine reale und nicht um eine hypothetische Gefahr handelt. 
\newline
(vgl. \cite{melani_halbjahresbericht_2015}, S. 11.)\\
\hline

& Phishing & Seit April 2015 beobachtete MELANI eine neue Vorgehensweise bei Phishingangriffen gegen Schweizer Finanzinstitute. Hacker versenden dabei keine Phishing E-Mails mehr, sondern schalten kostenpflichtige Werbe-Anzeigen bei Suchmaschinenbetreibern [\dots]. Zu diesem Zweck kaufen die Betrüger Stichworte (Keywords) bei den Suchmaschinenbetreibern, die im Zusammenhang mit dem angegriffenen Finanzinstitut stehen: Haben es die Phisher beispielsweise auf Kunden der «Bank XY» abgesehen, schalten diese Phishing-Werbeanzeigen für das Stichwort «XY» oder «XY Bank». Werbeanzeigen in Suchmaschinen werden üblicherweise zuoberst und gut sichtbar vor den eigentlichen Suchresultaten angezeigt. Die Wahrscheinlichkeit, dass ein Benutzer anstelle des tatsächlichen Suchresultats auf die Werbeanzeige klickt, um auf die Seite der «Bank XY» zu gelangen, ist also gross. 
\newline
(vgl. \cite{melani_halbjahresbericht_2015}, S. 20f.)\\
\hline

& Phishing & Eine weitere Vorgehensweise, welche MELANI im zweiten Halbjahr 2015 vermehrt beobachtete, war Phishing unter Zuhilfenahme von PDF-Dateien. Dazu werden die üblichen Phishing-E-Mails versendet. [\dots] birgt diese neue Praktik einen wichtigen Vorteil für die Angreifer: Durch Verwendung von PDF-Dateien werden E-Mail-Filter ausgehebelt, die in der Regel nur in der E-Mail selbst nach gefährlichem Inhalt Ausschau halten. Internetkriminelle scheinen dies realisiert zu haben und setzen daher vermehrt auf diese Masche. 
\newline
(vgl. \cite{melani_halbjahresbericht_2015}, S. 22f.)\\
\hline

\textbf{Halbjahresbericht 2015/1} & KPMG-Studie & Neben den technischen Angriffen sind auch Methoden, welche die menschlichen Schwächen ausnützen, bei den Angreifern beliebt. Eine Studie der Wirtschaftsprüfungsgesellschaft KPMG stützt diese Beobachtung und zeigt, dass Firmen immer noch allzu sehr auf Technologie setzen und den menschlichen Faktor beim Schutz vor Cyber-Angriffen vernachlässigen. Grundsätzlich sollte ein integrierter und ausgewogener Ansatz verfolgt werden, der Menschen und Prozesse ebenso berücksichtigt wie Technologien. 
\newline
(vgl. \cite{melani_halbjahresbericht_2015-1}, S. 16.)\\
\hline

& Meldestelle & Um gegen die diversen Bedrohungen Abhilfe zu schaffen, sind präventive Massnahmen essentiell. Diese Massnahmen können technischer, organisatorischer und auch strafrechtlicher Art sein. [\dots] Eine weitere wichtige Massnahme gegen Cyber-Vorfälle ist die Etablierung einer Meldekultur, sowohl in einem Betrieb als auch generell in der Bevölkerung. Nur wenn die Mitarbeitenden das Gefühl haben, dass sie bei einer Meldung eines Vorfalls ernst genommen werden, werden sie auch weiterhin Meldungen machen.
\newline
(vgl. \cite{melani_halbjahresbericht_2015-1}, S. 25.)\\
\hline

& Hacker-Angriff auf deutschen Bundestag & [\dots] waren zwar nur wenige Endgeräte betroffen, allerdings seien die Angreifer tief in das System eingedrungen, hätten sich dort frei bewegen können und könnten jederzeit wieder aktiv werden. 
Im weiteren Verlauf soll es denn auch schwierig gewesen sein, die Angreifer aus dem Netz zu entfernen. Als Massnahme wurde der Neuaufbau von Teilen des Netzes ins Auge gefasst. Deshalb wurde das interne Netzwerk des Bundestags am 20. August 2015 für vier Tage abgeschaltet, um die Folgen des Cyber-Einbruchs zu beseitigen. Diese einschneidende Massnahme zeigt die Tragweite des Vorfalles.
\newline
(vgl. \cite{melani_halbjahresbericht_2015-1}, S. 26.)\\
\hline

\textbf{Halbjahresbericht 2014/2} & Spionage auf Geschäftsreisen & Bereits seit Längerem wird davor gewarnt, bei der Verwendung von öffentlichen WLAN- Anschlüssen besonders vorsichtig zu sein. [\dots] Hierbei war es möglich, in einem ungesicherten, d. h. offenen Netzwerk (zum Beispiel in einem Internetcafé) mit wenig Aufwand ein Session Hijacking durchzuführen und damit Nutzerdaten wie Passwörter auszuspähen. [\dots] Der Sicherheitsdienstleister Kaspersky publizierte [\dots] einen Report über eine Gruppe von Angreifern, die gezielte Angriffe in Funknetzwerken von Hotels verübten, welche über diese bislang bekannten Angriffe hinausgehen\footnote{\url{https://blog.kaspersky.com/darkhotel-apt/6613}}. Seit vier Jahren sollen gezielt hochrangige Manager auf deren Geschäftsreisen in Asien angegriffen worden sein. 
\newline
(vgl. \cite{melani_halbjahresbericht_2014}, S. 22.)\\
\hline

& Datendiebstahl & Ein weiterer Fokus bei Datendiebstählen bildet der Finanzsektor. Schlagzeilen machte diesbezüglich ein Angriff auf die US-Grossbank J.P. Morgan. Daten von rund 76 Millionen Haushalten und sieben Millionen Unternehmen sollen bei diesem Angriff, der Mitte August 2014 entdeckt wurde, kopiert worden sein. Demnach wurden Kundendaten wie Namen, Adressen, Telefonnummern und E-Mail-Adressen von den Servern des Kreditinstituts entwendet. [\dots] Nach Ansicht der Behörden deuten verschiedene Hinweise auf hochprofessionelle Hacker hin, die möglicherweise in Russland zu finden sind.
\newline
(vgl. \cite{melani_halbjahresbericht_2014}, S. 23f.)\\
\hline

& Erpressung & [\dots] wird ein weiterer noch stärkerer Trend beobachtet: Ransomware, welche es vor allem auf die Datenverfügbarkeit von Privatnutzern abgesehen hat. [\dots] Solche Trojaner erweisen sich als unerschöpfliche Geldquelle für die Erpresser. Viele Experten gehen von einem Anhalten dieses Trends aus. Der Erfolg hängt hauptsächlich von der Bereitschaft der Opfer ab, die geforderte Summe zu bezahlen, um wieder auf ihre Daten oder ihren Rechner zugreifen zu können.  [\dots] Für die Kriminellen [\dots] zählt vielmehr der Wert, den die Daten für das Opfer haben und was dieses gewillt ist zu zahlen, um die Daten wiederzubekommen. Es ist deshalb gefährlich, wenn Nutzer den Schutz ihrer persönlichen Daten vernachlässigen, weil sie denken, dass diese wertlos für potenzielle Angreifer sind.
\newline
(vgl. \cite{melani_halbjahresbericht_2014}, S. 33.)\\
\hline

\textbf{Halbjahresbericht 2014/1}  & Phishing Trends & Bei einem Phishing-Versuch, welcher das Logo der Swisscom missbraucht hatte, kam eine andere Vorgehensweise zum Einsatz. Hierbei köderte man das Opfer mit einer Umfrage. Zuerst wurden zehn Fragen zu Produkten und zur Qualität der Dienstleistungen der Firma gestellt. Am Ende der Umfrage wurde dann Name und Adresse verlangt, aber auch Kreditkartendaten waren Teil der obligatorischen Angabe. Durch die zehn durchaus sinnvollen Fragen wurde versucht, die Bedenken der Opfer zu zerstreuen. Wieso sollte ein Betrüger auch eine Kundenumfrage machen?
\newline
Die aktuellen Phishing-Versuche zeigen, dass es für E-Mail-Empfänger stets schwieriger wird, Phishing-Versuche als solche zu erkennen.
\newline
(vgl. \cite{melani_halbjahresbericht_2014-1}, S. 7f.)\\
\hline

& Cloud Computing & Da heute E-Mails meist nicht mehr auf einen Computer heruntergeladen und dort gespeichert werden, sondern via Webmail und somit in einer Cloud bearbeitet werden, halten die Benutzer häufig ein Backup für überflüssig. Diese Pflicht wird von den meisten Benutzern unbewusst an den E-Mail-Provider delegiert. Sie gehen davon aus, dass der Provider für die Backups sorgt.
\newline
(vgl. \cite{melani_halbjahresbericht_2014-1}, S. 11.)\\
\hline

& NSA Enthüllung & Eine weitere Enthüllung [\dots] machte publik, dass die NSA «teilweise» per Post verschickte Netzwerkgeräte und Peripheriegeräte abfange, um darauf Spionagesoftware zu installieren. Mitarbeiter der «Tailored Access Operations (TAO)» würden dann in diese Geräte spezielle Technik einbauen, die Spionagemöglichkeiten erlaubt. Danach würden diese wieder verpackt und an den Empfänger verschickt.
\newline
(vgl. \cite{melani_halbjahresbericht_2014-1}, S. 21.) \\
\hline

& Heiminfrastruktur & Router stehen immer häufiger im Fokus der Angreifer, da diese oftmals unsichere Konfigurationen oder Sicherheitslücken enthalten. Auf der anderen Seite ist die Sensibilität der Benutzer bezüglich der Sicherheit von Routern noch nicht so gross. Meist wird ein Gerät angeschlossen und dann während der gesamten Lebenszeit nicht mehr gewartet und aktualisiert. Erschwerend kommt hinzu, dass gerade bei älteren Geräten Updates nicht automatisch eingespielt werden können.
\newline
(vgl. \cite{melani_halbjahresbericht_2014-1}, S. 31.) \\
\hline

\textbf{Halbjahresbericht 2013/1}  & Phishing Trends & Es wird vermehrt beobachtet, dass neben den zahlreichen Phishing-Versuchen gegen Kreditkartenfirmen und Grossbanken in der letzten Zeit auch kleinere Banken von Phishing betroffen sind. Ein Grund dürfte sein, dass die Betrüger auf diese Finanzinstitute ausweichen in der Hoffnung, dass hier die Sicherheitsmassnahmen noch nicht so hoch sind, respektive die Kunden noch nicht oft mit diesem Phänomen konfrontiert worden sind.
\newline
(vgl. \cite{melani_halbjahresbericht_2013}, S. 9f.) \\
\hline

& Gezielte Social Engineering-\newline Angriffe gegen Schweizer Firmen & [\dots] erhielt der Finanzchef einer international tätigen KMU eine E-Mail, welche angeblich vom CEO an ihn gesendet worden war.

An diesem Fall besonders interessant ist der relativ grosse Rechercheaufwand, den die Täter im Vorfeld betreiben mussten. So mussten sich die Betrüger, um überhaupt ein solch unternehmensbezogenes Szenario erstellen zu können, mit der Organisationsstruktur der Firma befassen und diese analysieren.
\newline
(vgl. \cite{melani_halbjahresbericht_2013}, S. 11)\\
\hline

%\end{tabular}
%\end{table}

\caption{Auszüge aus MELANI Lageberichten\label{long}}\\
\end{longtable}
 
\end{sloppypar}

\end{document}