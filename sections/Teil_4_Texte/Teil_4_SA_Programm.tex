\documentclass[../../main.tex]{subfiles}


\begin{document}

\subsection{Einleitung}
\subsection{Ziele}
  Richtiges MA Verhalten bei Risiken und Sicherheitsvorfällen\\
  Akzeptanz der Benutzerweisungen\\
  Stärkung der IT- und Informationssicherheit
\subsection{Kritische Erfolgsfaktoren}
  Unterstützung durch die Geschäftsleitung ''Awareness für Awareness''\\
  Nutzung und Fokussierung verschiedener Kommunikations Kanäle\\
    "Diskussionsmöglichkeiten, kein Fachjargon in Policies, einfache Sprache"\\
  Kontinuität\\
  Roadmap\\
  Berücksichtigung von kulturellen Aspekten\\
  Lokale Sprache / kulturelle Tabus im jeweilgen Kulturkreis
\subsection{Komponenten}
  Welche Komponenten aufgrund welcher Massnahmen
\subsubsection{Poster}
    "siehe SANS"
\subsubsection{Newsletter}
    Warum ein Passwortmanager?\\
    Online Einkaufen: aber sicher!\\
    Sichere dein neues Tablet!\\
    Mach dein Heimnetzwerk dicht!\\
    Was ist Malware?\\
    Sichere Deine Kinder!\\
    Lass Dich nicht angeln!
\subsubsection{Kommunikationskanäle}
    Security Blog / Intranet / Diskussionsforum
\subsubsection{Events}
    Einbindung in etablierte Mitarbeiterevents / Live-Hacking / Security Spiel
\subsubsection{Giveaways}
\subsubsection{Policies}
    Umgang mit E-Mail / Starke Passwörter / Womit arbeiten wir (Datenklassifizierung)
\subsubsection{Die Marke CISO}
    Logo / Slogan / Video / Comic / Figuration
\subsection{Metriken}
  als Matrix
  Erhebungsintervall
  Kommunikation der Resultatemessung (An wen, wie, wann)
\subsubsection{Schlüssel-Metriken}
    (Grundeinstellung) \% MA haben eine positive Grundeinstellung zu Security\\
    (Wahrnehmung) \% MA glauben, dass ihr Verhalten Einfluss auf Security hat\\
    (Wissen) Kannst Du einem Freund oder Kollegen erklären, was das ist...?\\
    SPAM, Phishing, Botnet, Wurm, Spyware, Virus, Trojaner\\
    Besucher auf Security Intranetseite\\
    Feedback der Mitarbeitenden\\
      Flurfunk / Direkte Anfrage\\
    Tests (anonymisisert)\\
      Präparierte Daten herumliegen lassen (wie viele werden abgegeben)\\
      Phishing emails /Testanrufe Telefon / Umfrage
\subsection{Phasenplan}
\subsubsection{Phase I: Sensibilisierung}
    Aufmerksamkeit gewinnen (Zielgruppenorientiert)
\subsubsection{Phase II: Umsetzung}
    Masse: Überzeugen / demonstrieren / aufklären / informieren
\subsubsection{Phase III: Anerkennung}
    Messen / Bestätigen / Festigen / Verankern



\end{document}