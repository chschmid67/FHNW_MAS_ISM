\documentclass[../../main.tex]{subfiles}

\begin{document}

%% - do not number the subsections, but put them in the TOC
%% - --------------------------------------

\subsubsection*{Sinn und Zweck dieser Arbeit}
\addcontentsline{toc}{subsubsection}{Sinn und Zweck dieser Arbeit}

\subsubsection*{Zielpublikum}
\addcontentsline{toc}{subsubsection}{Zielpublikum}

\subsubsection*{Abgrenzung}
\addcontentsline{toc}{subsubsection}{Abgrenzung}

\subsubsection*{Überblick Aufbau des Dokumentes}
\addcontentsline{toc}{subsubsection}{Überblick Aufbau des Dokumentes}

Die vorliegende Arbeit gliedert sich in die folgenden Teile:

\begin{sloppypar}
Der \textbf{erste Teil} zeigt auf, welche theoretischen Grundlagen für die "`Security Awareness NEXT GENERATION'' zu berücksichtigen sind...
\end{sloppypar}

\begin{sloppypar}
Mit dem \textbf{zweiten Teil} wird das Unternehmens selbst beschrieben und mit welchen Methoden die Security Awareness Ist-Situation im Unternehmen untersucht wird. Die Anspruchsgruppen für die Untersuchung werden definiert. Es wird ein Fragekatalog für eine Online-Umfrage entwickelt und die Auswertungsmethode beschrieben. Es wird aufgezeigt, welche Themenbereich von der Umfrage abgedeckt werden und warum diese ausgewählt wurden. Da zusätzlich zu dem Fragenkatalog noch Interviews mit einigen der befragten Mitarbeitenden durchgeführt werden, wird die Interviewform und deren Resultatgewinnung beschrieben.
\end{sloppypar}

\begin{sloppypar}
Der \textbf{dritte Teil} skizziert ein Security Awarenes Programm
\end{sloppypar}

\begin{sloppypar}
Im \textbf{vierte und letzten Teil} wird ...
\end{sloppypar}


\end{document}

