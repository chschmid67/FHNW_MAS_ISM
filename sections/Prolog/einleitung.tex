\documentclass[../../main.tex]{subfiles}

\begin{document}

%% - do not number the subsections, but put them in the TOC
%% - --------------------------------------

\subsubsection*{Ausgangslage}
\addcontentsline{toc}{subsubsection}{Ausgangslage}

\begin{sloppypar}
Die \companylong, ein in der Finanzindustrie tätiges Softwareunternehmen entwickelt und vertreibt in weltweit verteilten Niederlassungen ein Core-Bankensystem als Produkt.

Durch das schnelle Wachstum des Unternehmens (2007: Ein Standort in der Schweiz mit ca. 350 Mitarbeitenden, 2014: 7 internationale Standorte mit ca. 1600 Mitarbeitenden) und die sich rasch verändernde technologische Landschaft stellt es eine neue Herausforderung dar, Sicherheitsaspekte effizient in die täglichen Arbeitsabläufe einzubetten und Security Awareness nachhaltig im Bewusstsein der Mitarbeitenden zu festigen.
\end{sloppypar}

\subsubsection*{Sinn und Zweck dieser Arbeit}
\addcontentsline{toc}{subsubsection}{Sinn und Zweck dieser Arbeit}

\begin{sloppypar}
Der Schwerpunkt dieser Masterarbeit liegt auf der Erarbeitung eines Grobkonzeptes für ein Security Awareness Programm. Dieses soll die Bedürfnisse der Anspruchsgruppen des Unternehmens abdecken, in die Unternehmenskultur eingebettet sein sowie interkulturelle Aspekte berücksichtigen.

Die Forschungsfrage dieser Masterarbeit zielt darauf ab, wie und mit welchen Methoden unter Berücksichtigung von betriebssoziologischen, unternehmenskulturellen, respektive interkulturellen Aspekten eine Beeinflussung des persönlichen Sicherheitsverhaltens zugunsten einer für die Mitarbeitenden verständlichen Sicherheitskultur erreicht werden kann. Dabei steht vor allem die methodische Transformation in umsetzbare, praktische Massnahmen im Vordergrund.
\end{sloppypar}

\subsubsection*{Zielpublikum}
\addcontentsline{toc}{subsubsection}{Zielpublikum}

\begin{sloppypar}
Diese Arbeit richtet sich an alle Personen, welche in einer Unternehmung entweder für die Konzeption eines Security Awareness Programmes verantwortlich sind oder bei einer entsprechenden Umsetzung / Operationalisierung direkt beteiligt sind, im Speziellen:
\begin{itemize}
  \item Gesamtverantwortliche für die Informationssicherheit in einem Unternehmen \newline (Chief Information Security Officer CISO)
  \item Gesamtverantwortliche für das Informations- und Kommunikationsmanagement \newline in einem Unternehmen (Chief Information Officer CIO)
  \item ICT-Sicherheitsbeauftragte (Information Security Manager ISM)
\end{itemize}

\end{sloppypar}

\subsubsection*{Abgrenzung}
\addcontentsline{toc}{subsubsection}{Abgrenzung}

\begin{sloppypar}
Die Umsetzung der Massnahmen im Rahmen der Dauer eines Security Awareness Programmes (typischerweise 12 Monate) und das nochmalige Messen der Veränderung der Sicherheitskultur nach dessen Durchführung können aus zeitlichen Gründen nicht im Umfang dieser Masterarbeit durchgeführt werden. Grundsätzlich soll also mit dem empirischen Teil dieser Arbeit die Basis für eine spätere qualitative Wirkungsanalyse geschaffen werden. Ebenso ist eine Ertragsberechnung, besser bekannt unter dem Begriff \acrfull{rosi} auf das in das Security Awareness Programm investierte Kapital nicht Bestandteil dieser Arbeit. 
\end{sloppypar}

\end{document}

