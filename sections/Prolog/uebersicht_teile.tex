\documentclass[../../main.tex]{subfiles}

\begin{document}

\begin{sloppypar}
Die vorliegende Arbeit gliedert sich in die folgenden Teile:

%% - first part
%% - --------------------------------------

Der \textbf{erste Teil} zeigt auf, welche theoretischen Grundlagen für die "`Security Awareness NEXT GENERATION"' zu berücksichtigen sind. Zuerst wird aufgezeigt worin sich der neue NEXT GENERATION Ansatz vom bisherigen OLD SCHOOL Ansatz unterscheidet. Anschliessend werden die Normenwerke ISO / IEC 27001 und BSI-Standards 100-2 und 100-4 auf ihre Relevanz hinlänglich Security Awareness untersucht. Die MELANI Halbjahresberichte und die letzte KPMG Studie bezüglich Cyber Security schliessen die Grundlagen ab. Das Kapitel "`Kulturelle Betrachtung"' beschreibt die in einem Unternehmen vorherrschenden Kulturschichten auf und beschreibt ihre Relevanz für Security Awareness. Das Kapitel "`Psychologische Aspekte"` schliesst den ersten Teil ab.

%% - second part
%% - --------------------------------------

Mit dem \textbf{zweiten Teil} wird das Unternehmen {\companyshort} selbst beschrieben und mit welchen Methoden die Security Awareness Ist-Situation im Unternehmen untersucht wurde. Die Anspruchsgruppen für die Untersuchung werden definiert. Es wird ein Fragekatalog für eine Online-Umfrage entwickelt und die Auswertungsmethode beschrieben. Es wird aufgezeigt, welche Themenkomplexe von der Umfrage abgedeckt werden und warum diese ausgewählt wurden. Da zusätzlich zu dem Fragenkatalog noch Interviews mit einigen Mitarbeitenden aus den definierten Anspruchsgruppen durchgeführt werden, wird die Interviewform und deren Resultatgewinnung beschrieben.

%% - third part
%% - --------------------------------------

Der \textbf{dritte Teil} beschreibt in einem ersten Schritt die Auswertung der empirischen Daten (Online Umfrage, Experteninterviews) und skizziert aus den daraus gewonnenen Erkenntnissen einen Vorschlag für ein Security Awarenes Programm. Dabei werden aus den Ergebnissen der Empirie die möglichen Security Awareness Ziele herausgearbeitet sowie Massnahmen zu deren Umsetzung definiert. Die Massnahmen werden dabei auf die Unternehmenskultur unter Berücksichtigung von interkulturellen Aspekten abgestimmt. Der zweite Schritt achtet dabei vor allem auch die Anwendung von integrierter und systemischer Kommunikation sowie den Einbezug von klassischen und innovativen Marketingideen.

%Um die herausgearbeiteten Massnahmen bezüglich ihrer Wirksamkeit messen zu können wird im Rahmen dieser Arbeit ein geeignetes, für die jeweiligen Massnahmen spezifisches Verfahren ausgearbeitet. Dies beinhaltet die konkrete Beschreibung von Messungen und den dafür zu verwendenden Mitteln sowie die Kriterien zur Bewertung der Griffigkeit der Massnahmen.

%% - fourth part
%% - --------------------------------------

Im \textbf{vierten und letzten Teil} wird nach einem Fazit und einer kritischen Eigenreflexionn des Autors zur gesamten Studienarbeit die zusammenfassende Empfehlung dieser Masterarbeit für die Durchführung eines Security Awareness Programmes "`NEXT GENERATION"' festgehalten. Ein persönliches Schlusswort des Autors beschliesst diese  Masterarbeit.

\end{sloppypar}

\end{document}

