\documentclass[../../main.tex]{subfiles}

\begin{document}

\begin{sloppypar}
Die vorliegende Arbeit gliedert sich in die folgenden Teile:

%% - first part
%% - --------------------------------------

Der \textbf{erste Teil} zeigt auf, welche theoretischen Grundlagen für die "`Security Awareness NEXT GENERATION'' zu berücksichtigen sind. Die theoretischen Grundlagen sind dabei nicht nur normativer oder organisatorischer Natur Es soll aufgezeigt werden, wie mittels des NEXT GENERATION Ansatzes für ein diffuses Thema wie Security Awareness durch verschiedene, kognitiv aufeinander abgestimmte Massnahmen ein neues, in sich fundiertes Sicherheitsbewusstsein bei den Mitarbeitenden geweckt und gefestigt werden kann.

Einbettung in die Unternehmenskultur, Erkenntnissen aus der Tiefenpsychologie, involvierte Kulturkreise erarbeitet.

Die Security Awareness ist mit den Marketing- und Kommunikationsaktivitäten des Unternehmens aligniert.


%% - second part
%% - --------------------------------------

Mit dem \textbf{zweiten Teil} wird das Unternehmen \company selbst beschrieben und mit welchen Methoden die Security Awareness Ist-Situation im Unternehmen untersucht wird. Die Anspruchsgruppen für die Untersuchung werden definiert. Es wird ein Fragekatalog für eine Online-Umfrage entwickelt und die Auswertungsmethode beschrieben. Es wird aufgezeigt, welche Themenbereich von der Umfrage abgedeckt werden und warum diese ausgewählt wurden. Da zusätzlich zu dem Fragenkatalog noch Interviews mit einigen der befragten Mitarbeitenden durchgeführt werden, wird die Interviewform und deren Resultatgewinnung beschrieben.


%% - third part
%% - --------------------------------------

Der \textbf{dritte Teil} skizziert ein Security Awarenes Programm

In einem zweiten Schritt werden aus den Ergebnissen der Empirie die möglichen Security Awareness Ziele herausgearbeitet und Massnahmen zu deren Umsetzung definiert. Die Massnahmen werden dabei auf die Unternehmenskultur unter Berücksichtigung von interkulturellen Aspekten abgestimmt. Der zweite Schritt achtet dabei vor allem auch die Anwendung von integrierter und systemischer Kommunikation sowie den Einbezug von klassischen und innovativen Marketingideen bei der Konkretisierung von Massnahmen


Es werden die Grundlagen für die Erreichung eines Unternehmenszieles (Security Awareness ist ein integraler Bestandteil im Arbeitsalltag) geschaffen.

Security Awareness ist auf allen Führungs- und Mitarbeiterebenen bekannt, akzeptiert und wird auch im Arbeitsalltag gelebt.

Security Awareness wird von den Mitarbeitenden nicht mehr als Behinderung, sondern als ein für sie vorteilhafter, praxisorientierter Arbeitsaspekt wahrgenommen, welcher auch im privaten Umfeld sinnvolle Anwendungen findet.

Um die herausgearbeiteten Massnahmen bezüglich ihrer Wirksamkeit messen zu können wird im Rahmen dieser Arbeit ein geeignetes, für die jeweiligen Massnahmen spezifisches Verfahren ausgearbeitet. Dies beinhaltet die konkrete Beschreibung von Messungen und den dafür zu verwendenden Mitteln sowie die Kriterien zur Bewertung der Griffigkeit der Massnahmen.

%% - fourth part
%% - --------------------------------------

Im \textbf{vierte und letzten Teil} wird ...

\end{sloppypar}

\end{document}

