\documentclass[../../main.tex]{subfiles}

\begin{document}

\begin{sloppypar}
Die \company, ein in der Finanzindustrie tätiges Softwareunternehmen, entwickelt und vertreibt in weltweit verteilten Niederlassungen ein Core-Bankensystem als Software-Suite.

Durch das schnelle Wachstum des Unternehmens (2007: Ein Standort in der Schweiz mit ca. 350 Mitarbeitenden, 2015: acht internationale Standorte mit ca. 2100 Mitarbeitenden) und die sich rasch verändernde technologische Landschaft stellt es eine grosse Herausforderung dar, Sicherheitsaspekte effizient in die täglichen Arbeitsabläufe einzubetten. Security Awareness als ständige Disziplin soll nicht nur akzeptiert, sondern auch nachhaltig im Bewusstsein der Mitarbeitenden verankert werden. Um dies zu erreichen müssen neue, zeitgemässe Ansätze mit der vorherrschenden Firmenkultur abgestimmt und letztendlich in diese integriert werden. Durch die Internationalisierung und die Durchmischung der  verschiedenen Herkunftskulturen an den einzelnen Standorten kommt Interkulturälität als nicht zu vernachlässigender Faktor für Security Awareness hinzu.

Die immer noch weit verbreitete "`OLD SCHOOL"' Praxis beschränkt sich im Wesentlichen auf die Durchführung von Trainingsprogrammen, welche modularisiert von einer mittlerweile grossen Anzahl Anbietern als Service bezogen werden können. Diese Trainingsprogramme adressieren dabei den kognitiv-motorischen (objektbezogenen) Lernbereich (\cite{helisch_security_2009}). Sie bieten eine eher quantitative Aussage an, da die richtige Beantwortung von möglichst vielen Fragen nicht zwangsläufig eine Steigerung der Security Awareness mit sich bringt.

"`NEXT GENERATION"' Security Awareness hingegen geht von einem ganzheitlichen, mit der Unternehmenskultur alignierten und durch die systemische Integration mit den Hauptträgern der Unternehmenskommunikation (Marketing / Kommunikation) abgestimmten Ansatz aus (\cite{helisch_security_2009}). Das Endziel dieser Arbeit ist die Empfehlung eines Security Awareness Programmes, welches die Erwartungen der Anspruchsgruppen des Unternehmens abdeckt, in die Unternehmenskultur eingebettet ist sowie interkulturelle Aspekte berücksichtigt.
\end{sloppypar}

\subsubsection*{Dank}

\blindtext

\end{document}

