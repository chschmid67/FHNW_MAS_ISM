\documentclass[../../main.tex]{subfiles}

\begin{document}

\subsubsection{Normenwerke}

\begin{sloppypar}
Nachfolgend werden die (im europäischen Raum) bekanntesten Normenwerke ISO 27001 und BSI Grundschutz im Hinblick auf Security Awareness vorgestellt. Weitere Normen wie SOX 404 oder die "'Richtlinie 96/46/EG des europäischen Parlaments und des Rates zum Schutz natürlicher Personen bei der Verarbeitung personenbezogener Daten"' werden hier nicht berücksichtigt.
\end{sloppypar}

\paragraph*{ISO/IEC 27001:2013}\mbox{}

\begin{sloppypar}
Die Internationale Organisation für Normung \acrshort{iso} bildet zusammen mit der Internationalen Elektrotechnischen Kommission \acrshort{iec} die Körperschaft zur Formung weltweiter Standardisiserungen. Im Bereich Informationsrechnolgie wurde von der ISO und IEC ein gemeinsamer technischer Ausschuss, das \acrshort{jtc} 1, gebildet. Dieser technische Ausschuss ist verantwortlich für die ISO-Norm 27001:2013. Diese Norm beschreibt Aufbau, Implementierung, Unterhalt und kontinuierliche Verbesserung von Informations Sicherheits Management Systemen. Die aktuelle Version ISO/IEC 27001:2013 ersetzt die erste Ausgabe der Norm von 2005.

Für die Security Awareness wird im ISO-Sinne von \textit{Requirements Specifications} der Abschnitt 7 (Support) relevant:

\begin{quote}
\textbf{7.2 Competence}\newline

The organization shall:\newline
b) ensure that these persons are competent on the basis of appropriate education, training, or experience;\newline
c) where applicable, take actions to acquire the necessary competence, and evaluate the effectiveness
of the actions taken;\newline
\newline
NOTE Applicable actions may include, for example: the provision of training to, the mentoring of, or the reassignment
of current employees; or the hiring or contracting of competent persons (\cite{iso/iec_jtc_1_iso/iec_2013}, S.5).
\end{quote}

Die dazu zu implementierenden \textit{Controls} werden in Annex A Abschnitt 7, respektive 12 beschrieben:

\begin{quote}
\textbf{A.7.2.2 Information security awareness, education and training}\newline

All employees of the organization and, where relevant, contractors shall receive appropriate awareness education and training and regular updates in organizational policies and procedures, as relevant for their job function (\cite{iso/iec_jtc_1_iso/iec_2013}, S.11).
\newline
\newline
\textbf{A.12.2.1 Controls against malware}\newline

Detection, prevention and recovery controls to protect against malware shall be implemented, combined with appropriate user awareness (\cite{iso/iec_jtc_1_iso/iec_2013}, S.16).
\end{quote}

\end{sloppypar}

\paragraph*{BSI-Standard 100-2: IT-Grundschutz Vorgehensweise}\mbox{}

\begin{sloppypar}
Das deutsche \acrlong{bsi} (\acrshort{bsi}) empfiehlt die Einbindung aller Mitarbeitenden in den Sicherheitsprozess vom Beginn des Arbeitsverhältnisses bis zu dessen Beendigung (vgl. \cite{bsi_bsi-standard_2008}, S. 34f.).

Der mit der Empfehlung aus den Grundschutz-Katalog verknüpfte Baustein "`B1 Übergreifende Aspekte'" dokumentiert unter B 1.13 "`Sensibilisierung und Schulung zur Informationssicherheit"' die typischen Gefährdungen (Organisatorische Mängel, Menschliche Fehlhandlungen, Vorsätzliche Handlungen) welche unter dem Aspekt "'Mitarbeitende und Sicherheitsprozess"' auftreten können. Der Baustein  definiert Massnahmen für die Planung und Konzeption, Beschaffung, Umsetzung und Betrieb für die Einbindung aller Mitarbeiter in den Sicherheitsprozess.
\end{sloppypar}

\paragraph*{BSI-Standard 100-4: Notfallmanagement}\mbox{}

\begin{sloppypar}
Security Awareness betrifft auch den Bereich des Notfallmanagements für ein Unternehmen. Das BSI weist in seinem Standard über Notfallmanagement darauf hin, dass das Notfallmanagement fest in der Unternehmenskultur verankert werden muss. Eine entsprechende Sensibilisierung und Schulung der Mitarbeitenden ist dafür eine notwendige Voraussetzung (vgl. \cite{bsi_bsi-standard_2008-1}, S. 26).

Für die \companyshort{} wurde im Jahre 2014 ebenfalls eine Masterarbeit zum Thema "`Notfallmanagement"' erarbeitet. Der Autor weist bezüglich der Durchführung eines Awareness-Programmes auf folgendes hin:

\begin{quote}
"'Das dafür notwendige Wissen und der benötigte Aufwand für die Vorbereitung, Durchführung und Auswertung einer solchen Schulung dürfen keinesfalls unterschätzt werden. Gerade für Awareness Programme sollte ein Partner beigezogen werden, der sich auf dem Thema spezialisiert hat"' (\cite{schultheiss_business_2015}, S. 67).
\end{quote}

\end{sloppypar}

\subsubsection{Aktuelle Cyber-Security Berichte und Studien}

\begin{sloppypar}
Die nachfolgend aufgeführten Berichte und Studien sollen aufzeigen, wie eng das Thema Security Awareness mit Cyber Crime und Cyber-Security zusammenhängt. Die ausgewählten Beiträge sollen die Notwendigkeit von Massnahmen für den Aufbau einer tragfähigen Sicherheitskultur untermauern.
\end{sloppypar}

\paragraph*{KPMG Studie: Clarity on Cyber Security (2016)}\mbox{}

\begin{sloppypar}
Die im Mai 2016 von der KPMG AG (Schweiz) herausgegebene zweite Ausgabe der Publikation "`Clarity on Cyber Security"' fokussiert explizit auf den Zustand der Cyber Security in der Schweiz.\footnote{Dazu wurden 35 Grossfirmen sowie 25 KMU's im Rahmen einer Studie zu Cyber Risiken befragt.}  Nachstehend eine Zusammenfassung der wichtigsten Aussagen und Erkenntnisse, welche auch für Security Awareness eine Relevanz haben. Zusammenfassend sagt die Studie, Schweizer Unternehmen unterschätzen die Cyberrisiken im Zusammenhang mit dem Internet der Dinge. Sie arbeiten im Bereich der digitalen Sicherheit noch immer zu wenig zusammen und verfügen über ein mangelhaftes Verständnis der Bedrohungslage. Eine bedeutende Anzahl von Schweizer Firmen droht hier den Anschluss zu verlieren.(vgl. \citeauthor{bossart_clarity_2016} \citeyear{bossart_clarity_2016}, S. 6ff.).

\begin{quote} 
"`A number of Swiss organizations manage to at least keep up with the speed of the evolving threat landscape. The most advanced succeed in reducing the risk and leverage cyber to enable new business and operating models (for instance digitalization). Others struggle to keep up with the rapidly evolving threat landscape. They won’t be able to develop their business into Industry 4.0."' (\citeauthor{bossart_clarity_2016} \citeyear{bossart_clarity_2016}, S. 4).
\end{quote} 

\begin{quote} 
"`An understanding of the motivation, intent, strategy, tactics and the tools of the attackers is critical in order to anticipate threats and effectively prepare for, prevent, detect and respond to attacks. Bespoke correlation of technical and non-technical information is a must."' (\citeauthor{bossart_clarity_2016} \citeyear{bossart_clarity_2016}, S. 8).
\end{quote} 

\begin{quote} 
"`Most banks in Switzerland actively communicate with their employees and customers on cyber risks. [\dots] this should be done particularly when there is a risk of social engineering. This technique uses sophisticated methods that not only focus on the bank customers, but also on the bank employees. The perception of these employees is often that they operate in a very secure infrastructure."' (\citeauthor{bossart_clarity_2016} \citeyear{bossart_clarity_2016}, S. 21).
\end{quote} 

\begin{quote} 
"`Another related key point is that the IT department should not be the sole owner and driver of Cyber Security. It is the responsibility of the whole organisation."' (\citeauthor{bossart_clarity_2016} \citeyear{bossart_clarity_2016}, S. 29).
\end{quote} 

Für eine auszugsweise Zusammenfassung der wichtigsten Resultate der KPMG Studie bezüglich Security Awareness siehe Anhang C, Seite \pageref{kpmg_study_2016}.  
\end{sloppypar}

\paragraph*{MELANI: Halbjahresberichte}\mbox{}

\begin{sloppypar}
Die "`\acrlong{melani}"' des Bundes (\acrshort{melani}) veröffentlicht in ihren seit 2005 erscheinenden Halbjahresberichten die wichtigsten Tendenzen und Entwicklungen rund um Vorfälle und Geschehnisse in den Informations- und Kommunikationstechnologien, erklärt die technische Funktionsweise aktueller Angriffe, gibt eine Übersicht über Ereignisse im In- und Ausland und beleuchtet die wichtigsten Entwicklungen im Bereich der Prävention (vgl. \citeauthor{melani_lageberichte_2016}, \citeyear{melani_lageberichte_2016}).

Für eine auszugsweise Zusammenfassung der die Security Awareness betreffenden MELANI Halbjahresberichte siehe Anhang C, Seite \pageref{melani_halbjahresberichte}. 
\end{sloppypar}

\end{document}