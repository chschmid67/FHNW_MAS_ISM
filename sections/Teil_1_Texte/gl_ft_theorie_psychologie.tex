\documentclass[../../main.tex]{subfiles}

\begin{document}

\newpage

\subsection{Psychologische Aspekte}

\begin{sloppypar}

\begin{quote}
\textsc{"`Kindern erzählt man Geschichten zum Einschlafen. Erwachsenen, damit sie aufwachen"'} \newline \newline \footnotesize{\textit{Jorge Bucay, Autor, Psychiater und Gestalttherapeut}}
\end{quote}

Dieses Zitat von Bucay zeigt, was bei Menschen schon seit Jahrhunderten funktioniert: Geschichten. Geschichten sind eine (mitunter wahrscheinlich die älteste und daher auch die bewährteste) Möglichkeit, einer Botschaft Eingang in das Seelische bei den Menschen, respektive bei den Mitarbeitenden zu verschaffen. 

Eine gute, realistische Geschichte beginnt im Alltag der Menschen, genau jenem (Arbeits-) Alltag also, den es über die Security-Awareness-Geschichte zu beeinflussen gilt. Eine gute Geschichte erregt die Aufmerksamkeit des Publikums, sie erzeugt Impact, sie wird verstanden, da die Zuhörerschaft sich in das Geschehen hineinversetzen, oder noch besser, sich damit identifizieren kann. Die Mitarbeitenden wollen Security als eine gute Story, einen CISO als lebendiges Gesicht, eine Involvierung ihrer Alltagssituationen in die Geschichte (vgl. \citeauthor{marchesi_security_2014} \citeyear{marchesi_security_2014}, S. 10). Dann wird Security als Thema buchstäblich be-greifbar und erhält auch die nötige Glaubwürdigkeit. Ohne das Geschichten-Element bleibt Security und die damit verbundene Awareness "`draussen"' als ein seltsamer, unbelebter Teil des Unternehmens (vgl. \citeauthor{helisch_security_2009} \citeyear{helisch_security_2009}, S. 97). Das Mittel "`Geschichte"' soll also eigentlich die Awareness für die Awareness wecken, Identifikationsfiguren bekanntmachen / erschaffen und in den Konsumenten das Bedürfnis nach mehr Wissen über die Handlung selber, ihrer Protagonisten und letztendlich das Bedürfnis nach Fortsetzungsgeschichten wecken.

Soll Security Awareness efolgreich vermarktet werden, sprich: in der Zielgruppe der Mitarbeitenden den Kaufimpuls "`ich will das haben"' auslösen (siehe Kapitel \ref{wissen_wollen_können}, Seite \pageref{wissen_wollen_können}), so muss die Psychologie hinzugezogen werden. Sie kann dazu verwendet werden, die Geschichte von Security Awareness so erzählen zu können, dass sie zum beeinflussenden Faktor des zukünftigen Verhaltens im schwer zu störenden seelischen Alltag der Mitarbeitenden wird (vgl. \citeauthor{helisch_security_2009} \citeyear{helisch_security_2009}, S. 75 f.).

\end{sloppypar}

\subsubsection{Security Awareness aus dem Blickwinkel der Gestaltpsychologie}

\begin{sloppypar}
In der Gestaltpsychologie\footnote{Unter "`Gestalt"' wird hier eine "`Ganzheit"' im Sinne eines Musters oder einer besonderen Art, wie die vorhandenen Einzel-Elemente organisiert sind, verstanden (vgl. \citeauthor{staemmler_was_2009} \citeyear{staemmler_was_2009}, S. 64).} geht es um die Veränderung des eigenen Erlebens und Verhaltens über die Bewusstseinswerdung. Das im deutschen verwendete Wort "`Bewusstheit"' wird in der Gestaltpsychologie weitgehend als Synonym für Awareness verwendet. Im Bewusstsein wird zwischen "`Achtsamkeit"', d.h. das Richten der Aufmerksamkeit auf unmittelbar wahrzunehmendes und dem "`Gewahrsein"', d.h. dem Erkennen von umfassenden Zusammenhängen  unterschieden. Beides wird in der Gestaltpsychologie dabei als Voraussetzung für die seelische Veränderung des eigenen Erlebens und Verhaltens angesehen (vgl. \citeauthor{helisch_security_2009} \citeyear{helisch_security_2009}, S. 76 f.).

Aus der Perspektive der Gestaltpsychologie muss es also insofern ein Ziel von Awareness-Massnahmen sein, die Bewusstheit der Mitarbeitenden nicht nur auf Gefahren, sondern auch auf ihre eigene Verantwortung zu fördern. Der Dialog muss dabei auf Augenhöhe stattfinden. Der zuständige "`Gestalttherapeut für Security Awareness"' (\companyshort : CISO) wird seinen "`Klienten"' (\companyshort : Mitarbeitende) trotz seiner Qualifikationen nicht in der Rolle eines überlegenen Experten gegenübertreten. Er wird ihnen vielmehr als persönlich erkennbarer Mensch, der sie mit Interesse und Engagement auf der Security Awareness Reise begleitet, gegenübertreten (vgl. \citeauthor{helisch_security_2009} \citeyear{helisch_security_2009}, S. 79 und 81, \citeauthor{staemmler_was_2009} \citeyear{staemmler_was_2009}, S. 12). \citeauthor{staemmler_was_2009} schreibt dazu:

\begin{quote}
"`[\dots] dass Menschen in der Regel nur dann das Gefühl entwicklen, verstanden zu werden, wenn ihr Gegenüber nicht nur intellektuell begreift, um was es ihnen geht, sondern auch emotional mitschwingt und Anteilnahme empfindet"' (\citeauthor{staemmler_was_2009} (\citeyear{staemmler_was_2009}), S. 15).
\end{quote}

\end{sloppypar}

\subsubsection{Sicherndes und entsicherndes Verhalten}

\begin{sloppypar}
Laut \citeauthor{helisch_security_2009} fördern optimierte Arbeitsabläufe und eine unpersönliche Unternehmenskultur eine Reduzierung von Eigenheiten und den Verlust von Identität. Die Arbeitsabläufe sind störungsfrei und sauber -- die Arbeitsumgebung ist entmenschlicht. Es ist weniger direkter menschlicher Kontakt zu anderen Kolleginen und Kollegen nötig, da eine Email schreiben schneller und bequemer ist als aufzustehen und hinzugehen. Die Mitarbeitenden als soziale Wesen finden jedoch Mittel und Wege, in einer solchen Umgebung etwas Eigenes unterzubringen, und sei es nur über die Ausgestaltung des Passwortes. Zu strenge Sicherheitspolicies (das Passwort muss aus einer zufälligen Zeichenfolge bestehen und darf keine ganzen Wörter enthalten) lässt keinen Raum für die Unterbringung von Persönlichem und die Umsetzung wird als unmenschlich und versachlichend empfunden. 

Dabei sollten gemäss \citeauthor{helisch_security_2009} die Mitarbeitenden dazu ermutigt werden, persönliches in ein Passwort einzubauen. Es ist dabei wichtig, eine verständliche Systematik und Anleitung abzugeben damit das persönliche im Passwort kunstvoll versteckt und den weiteren Sicherheitsanforderungen trotzdem entsprochen werden kann (vgl. \citeauthor{helisch_security_2009} \citeyear{helisch_security_2009}, S. 92 f., \citeauthor{marchesi_security_2014} \citeyear{marchesi_security_2014}, S. 9).

\end{sloppypar}


\begin{comment}



\subsection{Marketing / Kommunikation}

\subsubsection{4P}
    Produkt => was\\
    Preis => wieviel\\
    Platz => wo\\
    Promotion => Info
    
\subsubsection{Wirkungsweise von Marketing}

    Wahrnehmung / Verarbeitung / Verhalten


\subsection{Kommunikation}

\subsubsection{Wirkungsweise von Kommunikation}

    Systemische Kommunikation\\
    Integrierte Kommunikation\\
    Einbindung Unternehmenswerte

\end{comment}


\end{document}