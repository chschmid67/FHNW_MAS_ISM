\documentclass[../../main.tex]{subfiles}

\begin{document}

\subsection{Kulturelle Betrachtung}

\begin{sloppypar}
Die Kulturfrage stellt an die Security Awareness besondere Anforderungen. Auf die Frage, was Kultur überhaupt ist, antwortet \textsc{Schein} (1985), zitiert nach \citeauthor{helisch_security_2009}, \citeyear{helisch_security_2009}, S. 187:

\begin{quote}
"`Zusammenfassend kann 'Kultur' auch als 'die Gesamtheit der Grundannahmen, Werte, Normen, Einstellungen und Überzeugungen einer sozialen Einheit' (\textsc{Schein}) verstanden werden."'
\end{quote}

Die kulturellen Unterschiede sind vor allem durch einen nicht-sofort-sichtbaren Teil geprägt. Die Nichtwahrnehmung dieser Dinge (Annahmen, Glauben, Weltanschauungen, Denkweisen) kann zu deutlichen Missverständnissen bis hin zur Kommunikationsstörung führen (vgl. \citeauthor{helisch_security_2009} \citeyear{helisch_security_2009}, S. 188). Um solchen Irrtümern und groben Versehen bereits schon von Anfang an vorzubeugen, sollte man sich möglichst früh in der Planungsphase von Security Awareness mit diesen nicht greifbaren Bestandteilen konstruktiv auseinandersetzen. Auch ist die Inanspruchnahme von Unterstützung durch aus der Zielkultur stammenden Personen ein adäquates Mittel, kulturelle Klippen erfolgreich zu umschiffen.
\end{sloppypar}

\subsubsection{Unternehmenskultur}

\begin{sloppypar}
Die Unternehmenskultur ist meistens ein gewachsenes, gemeinsames Verständnis von Werten, Ideen, Vorstellungen und Handlungsweisen, welche die Mitglieder der Organisation verfolgen. Meist tun sie dies, ohne sich dessen bewusst zu sein. Unternehmenskultur bestimmt, welche Handlungen erwünscht und welche unerwünscht sind (vgl. \citeauthor{helisch_security_2009} \citeyear{helisch_security_2009}, S. 24 ff.).

Im Falle der \companyshort ist die Unternehmenskultur nicht direkt gewachsen, sondern sie wurde aus einer grösseren Mitarbeiterumfarge herausfaktorisiert (siehe dazu auch Kapitel \ref{unternehmenskultur} auf Seite \pageref{unternehmenskultur}). 
\end{sloppypar}

\subsubsection{Managementkultur}

\begin{sloppypar}
Über allen kulturellen Betrachtungen gilt es, die vorherrschende Managementkultur nicht zu vergessen. Oft geniessen höhere (Top-) Manager Sonderprivilegien, welche sie (ihrer Meinung nach, Anm. des Autors) von der Teilnahme an einer Sicherheitskultur, respektive an sicherheitskonformen Verhalten entbinden, oder sie können es sich aufgrund ihrer hohen Stellung erlauben, die Sicherheitsvorschriften nicht zu beachten. Dieses Verhalten hat selbstverständlich eine breite Signalwirkung auf den Rest der Belegschaft.

Gerade beim Top-Management ist es daher emminent wichtig, eine gewisse "`Awareness für Awareness"' zu schaffen. Bei Security Awareness Kampagnen zählt die Integration der Führungsebene zu den kritischen Erfolgsfaktoren. Das Top-Management als Zielgruppe ist besonders zu behandeln, hängt doch der Gesamterfolg für eine Security Awareness Kampagne letztendlich von der Überzeugung der Entscheiderebene ab. Ist diese Ebene nicht davon überzeugt, dass Security Awareness nötig ist und entsprechend kein adäquates Budget dafür bewilligt wird, ist der Erfolg einer Security Awareness Kampagne oder die Implementation von begleitenden Massnahmen stark in Frage gestellt (vgl. \citeauthor{helisch_security_2009} \citeyear{helisch_security_2009}, S. 142 f.).
\end{sloppypar}

\subsubsection{Sicherheitskultur}

\begin{sloppypar}
Die im Unternehmen vorhandene (oder eben nicht vorhandene) Sicherheitskultur lehnt sich stark an die zuvor gestreifte Unternehmenskultur an. Um die Mitarbeitenden für das Thema Sicherheit sensibilisieren zu können, muss mehr oder weniger stark in die Unternehmenskultur eingegriffen werden. Dabei kommen die Faktoren Prägnanz, Verbreitungsgrad und Verankerungstiefe zum Tragen.

Die \textit{Prägnanz} bestimmt dabei, wie klar ist, was an Sicherheitskonformen Verhalten erwünscht ist und was nicht, wie eindeutig und konsistent sind die Standards definiert und wie klar ist das Werte- und das Symbolsystem gehalten. Der \textit{Verbreitungsgrad} gibt eine Indikation darüber aus, wie viele andere Organisationsmitglieder die Kultur bereits teilen. Die \textit{Verankerungstiefe} zeigt auf, inwieweit sich eine Sicherheitskultur bereits internalisisert hat, d.h. als akzeptierte Verhaltensnorm etabliert, bzw. verinnerlicht hat. (vgl. \citeauthor{helisch_security_2009} \citeyear{helisch_security_2009}, S. 25 ff.). Helisch warnt aber auch vor einer möglichen Unterminierung der Unternehmenskultur durch eine dem Unternehmen nicht angepasste Sicherheitskultur:

\begin{quote}
"`Wichtig ist, dass Sicherheit keine Gegenkultur zur Unternehmenskultur wird, Sicherheits- und Unternehmenskultur müssen sich ergänzen bzw. zueinander passen"' (\citeauthor{helisch_security_2009} (\citeyear{helisch_security_2009}), S. 27).
\end{quote}

\end{sloppypar}

\subsubsection{Fachkultur}

\begin{sloppypar}
Ergänzend zu diesen Ausführungen sind die neuesten Erkenntnisse aus der Awareness-Branche zu berücksichtigen. Ein reines Kampagnenmodell (wie ursprünglich in der vorliegenden Arbeit geplant) durchzuführen, ist nach Meinung von \citeauthor{wesselmann_awareness:_2016} (\citeyear{wesselmann_awareness:_2016}) nur bedingt sinnvoll wenn nicht auch der Fachkultur, respektive den bereits vorhandenen Kompetenzen (IT Technologien, Sicherheitsbewusstsein) Rechnung getragen wird. Besondere Anspruchsgruppen wie Hightech-Entwickler oder hochrangige Business Manager rechnen sich zumindest teilweise selbst die Kompetenz zu, in Sicherheitsfragen mitzureden und fordern ein Mitspracherecht beim "`Wie"' des Umgangs mit modernen Kommunikationsmitteln (vgl. \citeauthor{wesselmann_awareness:_2016} \citeyear{wesselmann_awareness:_2016}, S. 6 f.).

Dies ist bei der Umsetzung von Awareness-Massnahmen zu berücksichtigen. \citeauthor{wesselmann_awareness:_2016} (\citeyear{wesselmann_awareness:_2016}) schlägt vor, eine vom Service-Gedanken getragene, auf Augenhöhe stattfindende Kooperation zwischen den beteiligten Seiten (IT/ Security / Anwender) einzugehen.
\end{sloppypar}

\subsubsection{Interkulturalität}

\begin{sloppypar}
Bei einem international aufgestellten Unternehmen wie der \companyshort{} kommt der Berücksichtigung der Interkulturalität ein besonderer Stellenwert zu. \citeauthor{helisch_security_2009} (\citeyear{helisch_security_2009}, S. 186) ist der Ansicht, dass der Erfolg einer Security Awareness Kampagne in einem internationalen Unternehmen ausschliesslich auf dem tiefen Verständnis der im Konzern vereinten unterschiedlichen Kulturen basiert.

Im Zentrum dieses Verständnisses steht daher die Frage, was bei einer Security Awareness Kampagne genau an eine spezifische, beispielsweise östliche oder arabische Kultur angepasst werden muss. Auch die Art und Weise, wie Regeln und Policies vermittelt werden in die verschiedenen Kulturkreise vermittelt werden müssen, bzw. was um jeden Preis zu vermeiden ist (vgl. \citeauthor{helisch_security_2009} \citeyear{helisch_security_2009}, S. 185 f.). Auf diese Frage gibt die Theorie der High- und Low-Context Kulturen einige Hinweise.
\end{sloppypar}

\paragraph*{Low-Context vs. High-Context Kulturen}\mbox{}

\begin{sloppypar}
Kulturell bedingte Unterschiede zwischen den Denk- und Verhaltensweisen von Mitarbeitenden aus verschiedenen Ländern kann zu erheblichen Problemen führen. Dabei wird beim Begriff \say{interkulturelle Kommunikation} davon ausgegenagen, dass wenn Angehörige verschiedener \say{Kulturen} aufeinandertreffen und miteinander kommunizieren, Verständigungsprobleme semantischer Art entstehen können. Missverständnisse können etwa durch Ausdrucks-, Darstellungs- und Handlungsweisen entstehen. Lautstärke, Tonfall, Mimik, Gestik, Körpersprache und Grad der Freundllichkeit sind dabei die treibenden Faktoren. Während Beispielsweise in Low-Context Kulturen die Vermittlung von Inhalten stets involvierend, interaktiv und spielerisch ist, wird in High-Context Kulturen oft auf Frontalunterricht mit auswendiglernen gesetzt (vgl. \citeauthor{helisch_security_2009} \citeyear{helisch_security_2009}, S. 193 ff.).

Werden also in einem Low-Context Seminar über Security Awareness als spielerische Elemente eingesetzt und Rollenspiele durchgeführt, so kann bei einem aus einer High-Context stammenden Teilnehmer durchaus der Eindruck von Unseriösität, resp. Lächerlichkeit gegenüber einer eigentlich ernsten Sache entstehen.

Bezüglich der Art der interkulturellen Kommunikation weist \citeauthor{hall_beyond_1976} (\citeyear{hall_beyond_1976}) darauf hin, dass die Anwendung von "`situativen Dialekten"', d.h. die Verwendung der jeweiligen Fach- und Umgangssprache sowie die Durchführung der Kommunikation an Orten an welchen sich die angesprochenen Menschen wohl und sicher fühlen, für den letztendlichen Erfolg massgeblich beiträgt (vgl. \citeauthor{hall_beyond_1976} \citeyear{hall_beyond_1976}, S. 133 ff.)

Das Modell des Kulturen-Kontexts geht davon aus, dass sich Kulturen klassifizieren lassen. Das Low-Context-High-Context Modell nach \citeauthor{hall_beyond_1976} ordnet die Ethien nach ihrem kulturellen Kontext:  

\begin{quote}
"`Context in some sense is just one of many ways of looking at things. Failure to take contexting differences into account, however, can cause problems [\dots]. High-context cultures make greater distinctions between insiders and outsiders than low-context cultures do. People raised in high-context systems expect more of others than do the participants in low-context systems. When talking about something that they have on their minds, a high-context individual will expect his interlocutor to know what’s bothering him, so that she doesn't have to be specific. The result is that he will talk around and around the point, in effect putting all the pieces in place except the cruical one. Placing it properly -- this keystone -- is the role of his interlocutor. To do this for him is an insult and a violation of his individuality"' (\citeauthor{hall_beyond_1976} (\citeyear{hall_beyond_1976}), S. 113).
\end{quote}

In diesem Modell werden Ethnien wie Niederländer, Deutsche und Amerikaner grundsätzlich als offene, unkomplizierte von wenig bis keinem Hierarchiedenken durchdrungene Low-Context Kulturen dargestellt. Andere Ethnien wie Japaner oder Araber werden als streng reglementierte, mehrheitlich durch Traditionen und Hierarchien gesteuerte High-Context Kulturen beschrieben. 
\end{sloppypar}

\paragraph*{Vorsicht Fettnäpfchen: Humor, Bildsprache, Slogans, Farben und Symbole}\mbox{}

\begin{sloppypar}

\begin{quote}
\textsc{"`Wahr ist nicht was A sagt, sondern was B versteht"'} \newline \newline \footnotesize{\textit{Paul Watzlawick, Kommunikationswissenschaftler, \dag{} 2007}}
\end{quote}

\citeauthor{watzlawick_menschliche_2011} (\citeyear{watzlawick_menschliche_2011}, S. 71 ff.) sind der Ansicht, dass es grundsätzlich zwei verscheidene Weisen gibt, in denen Objekte zum Gegenstand von Kommunikation werden können. Als Ausdrucksmöglichkeit für ein Objekt kann eine \textit{Analogie} (z.B. eine Zeichnung) oder eine \textit{Digitalisierung} (z.B. ein Namen) verwendet werden. Der Unterschied zwischen digitaler und analoger Kommunikation lässt sich dadurch gut erklären, dass das blosse einmalige Hören (digital) einer unbekannten Sprache nicht zu deren Verstehen führt, während sich jedoch weitreichende Informationen aus der Beobachtung (analog) von Zeichensprache und Ausdrucksgebärden ableiten lassen, selbst wenn die sie verwendende Person einer fremden Kultur angehört. 

Digitale Sprache wird daher mit Vorteil für die denotative\footnote{mehrdeutiger Ausdruck der Semantik} Kommunikation auf der Inhaltsebene eingesetzt, da sie eine logische Syntax hat. Analoge Kommunikation findet hingegen auf der Beziehungsebene statt. Nach \citeauthor{watzlawick_menschliche_2011} sind alle Analogen Kommunikationen Beziehungsapelle, d.h. Vorschläge über die künftigen Regeln in der Beziehung, dem Umgang miteinander (vgl. \citeauthor{watzlawick_menschliche_2011} \citeyear{watzlawick_menschliche_2011}, S. 114 f.).

Einhergehend mit dieser Feststellung kommt der vielfach anzutreffenden Doppeldeutigkeit von analogen Mitteilungen (Tränen der Freude oder Tränen des Schmerzes, die Auslegung von Zurückhaltung als Takt oder Gleichgültigkeit, die geballte Faust als Drohung oder als Selbstbeherrschung) bei der Kommunikation von Security Awareness beim Einsatz von Bildsprache, Humor, Slogans, etc. eine besondere Bedeutung zu.
\end{sloppypar}

\subparagraph*{Nonverbale (analoge) Kommunikationselemente}\mbox{}

\begin{sloppypar}
Den das Unternehmen repräsentierenden (oder im Falle einer Security Awareness Kampagne eingesetzten) Symbolen wie Logos, Figuren oder auch der Gebrauch von Farben muss aufgrund der kulturell bedingten, unterschiedlichen Bedeutungsweisen bei den Mitarbeitenden eine besondere Aufmerksamkeit zuteil werden, ebenso beim Einsatz von Bildsprache. Semantisch behaftete Symbole, Figuren, Farben\footnote{Die Farbe Weiss gilt in europäischen Kulturen als Farbe der Reinheit, der Unschuld und Güte. In weiter östlichen Kulturen wird Weiss hingegen mit Trauer (Pakistan) bis hin zur Symbolfarbe des Todes (Japan) asoziiert (vgl. \citeauthor{helisch_security_2009} \citeyear{helisch_security_2009}, S. 197 f.).  } und Bilder bedürfen einer kulturbedingten Adaption, da ansonsten das Missverständnis vorprogrammiert ist. Beispiel: Nackte Füsse gelten in muslimisch geprägten und asiatischen Ländern als ein Ausdruck der Verachtung. Dasselbe gilt für Schuhwerk, man erinnere sich hierzu an den Journalisten Muntazer al-Zaidi, welcher 2008 während einer Pressekonferenz den damaligen amerikanische Präsidenten George W. Bush mit zwei Schuhen bewarf.\footnote{"`Der Angreifer hatte die Tatwaffe bewusst gewählt. Denn es gibt in der arabischen Welt kaum ein anderes Kleidungsstück, mit dem sich so viel Verachtung ausdrücken und so viel Erniedrigung produzieren lässt. Die kulturelle Bedeutung der Schuh-Attacke war Bush wohl nicht bewusst, als er schlagfertig ins Mikrofon tönte, es handele sich um ein Produkt der Größe zehn (europäisch 43-44)"' (\cite{kornelius_bush_2010}).  }

\end{sloppypar}

\subparagraph*{Verbale (digitale) Kommunikationselemente}\mbox{}

\begin{sloppypar}
Beim Einsatz von Humor und Slogans ist bei jedem adressierten Kulturkreis der Beizug von muttersprachlich versierten Mitarbeitenden ein Muss, nicht nur zur Vermeidung von sprachlichen Stilblüten bei der Auswahl, resp. der Übersetzung von Slogans oder Claims. Es sind mitunter umgangssprachliche Gepflogenheiten oder historische Ereignisse mit zu berücksichtigen\footnote{"`Einer der wohl bekanntsten Slogans, der nicht funktionierte: Der skandinavische Staubsaugerhersteller Electrolux übersetzte seinen Werbeslogan \say{Nichts saugt wie ein Electrolux} für den amerikanischen Markt mit \say{Nothing sucks like an Electrolux}. Dort bedeutet der Spruch jedoch: \say{Nichts ist so [ätzend] (Wortänderung C.S.) wie ein Electrolux}. Ganz klar: Ein Slogan, der in dem einem Land zieht, bedeutet noch lange keinen Erfolg in einem anderen Land – zumal, wenn er falsch übersetzt wurde"' (\cite{schneider_marken-flops_2011}). }. (vgl. \citeauthor{helisch_security_2009} \citeyear{helisch_security_2009}, S. 195 ff.). \cite{helisch_security_2009} empfiehlt grundsätzlich klar, dass wenn digitale Kommunikationselemente (also Schrift oder Sprache)in anderen Ländern informationen an den Mann oder die Frau gebracht werden sollen, die dortigen Mitarbeiter in ihrer Landessprache via Geschichten angesprochen werden sollen. Geschichten funktionieren in allen Kulturkreisen und helfen mit, Abstrakte Inhalte besser verständlich zu machen (vgl. \citeauthor{helisch_security_2009} \citeyear{helisch_security_2009}, S. 199).
\end{sloppypar}

\end{document}

