\documentclass[../../main.tex]{subfiles}

\begin{document}

\subsection{Einleitung}

\subsubsection{Security Awareness — eine perspektivische Darstellung}
    Marktbetrachtung\\
    Wirtschaftlichkeit\\
    (Risiko-)Faktor Mensch\\
    Bewusstseinsebene
\subsubsection{Sicherheitselemente}
    Pyramide beschreiben (Menschen / Prozesse / Produkte / Technologien / Infrastrukturen)
\subsubsection{Sicherheitsziele Confidentiality - Integrity - Availability (CIA)}
    Sicherheitsziele aus der Perspektive Security Awareness
\subsubsection{Risiken und Security Awareness}
    Welche Risiken kennen wir und was heisst dies bezüglich Security Awareness? => <kes> Sicherheitsstudie\\
    Pro und Contra Security Awareness
\subsubsection{Kulturelle Betrachtung}
    Gesellschaftskultur\\
    Sicherheitskultur\\
    Persönliche Sicherheitskultur\\
    Interkulturalität
 \subsubsection{OLD SCHOOL Security Awareness vs. Next Generation}
    Heutige Ansätze / Problemfelder / was kann resp. muss anders gemacht werden
\subsubsection{Wissen - Wollen - Können}
    Wissen: Kognitiver Faktor\\
    Wollen: Marketing + Kommunikation\\
    Können: Empowerment / Changemanagement
\subsubsection{Awareness und Change}
\subsection{Normenwerke}
  Welche Normenwerke gibt es und was sagen diese zu Security Awareness?
\subsubsection{ISO IEC 2700x}
    - ISO 27002 (2013) § 7.2.2 Deliver information security awareness programs
\subsubsection{Richtlinie 95/46/EG}
Richtlinie 95/46/EG des Europäischen Parlaments und des Rates vom 24. Oktober 1995 zum Schutz natürlicher Personen bei der Verarbeitung personenbezogener Daten und zum freien Datenverkehr
\subsubsection{FINMA Rundschreiben 2008/21}
\subsubsection{SOX 404}
\subsubsection{Empfehlungen für Security Awareness}
  - BSI IT-Grundschutz\\
  - COBIT
\subsection{Lerntheorie}
\subsubsection{Lernkonzepte / Lerntypen}
    soziales Lernen \\
    individuelles Lernen \\
    Formales Lernen \\
    Informelles Lernen
\subsubsection{Lernkonzepte}
    E-Learning \\
    Lernspiele \\
    Training
\subsubsection{Konsequenzen für Security Awareness}
\subsection{Marketing}
\subsubsection{4P}
    Produkt => was\\
    Preis => wieviel\\
    Platz => wo\\
    Promotion => Info
\subsubsection{Wirkungsweise von Marketing}
    Wahrnehmung / Verarbeitung / Verhalten
\subsubsection{Konsequenzen für Security Awareness}
\subsection{Kommunikation}
\subsubsection{Wirkungsweise von Kommunikation}
    Systemische Kommunikation\\
    Integrierte Kommunikation\\
    Einbindung Unternehmenswerte
\subsubsection{Konsequenzen für Security Awareness}
\subsection{Psychologische Aspekte}
\subsubsection{Bewusstsein / Gestaltpsychologie / Figuration}
\subsubsection{Sicherndes vs. entsicherndes Verhalten}
\subsubsection{Konsequenzen für Security Awareness}

\end{document}