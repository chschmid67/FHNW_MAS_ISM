\documentclass[../../main.tex]{subfiles}

\begin{document}

\subsubsection{Potentielle Anspruchsgruppen}

\begin{sloppypar}
Für die Datenerhebung wurden die folgenden potentiellen Anspruchsgruppen mit den von ihnen hauptsächlich zugreifbaren Datenbeständen identifiziert. Für diese Datenbestände wird anhand ihres Integritäts- und Vertraulichkeitsanspruches der jeweilige Schutzbedarf eingeschätzt um damit diejenigen Anspruchsgruppen zu bestimmen, die bei der nachfolgenden Datenerhebung berücksichtigt werden müssen.
\end{sloppypar}

%% - table metadata
%% - --------------------------------------

\begin{table}[H]
\tablefontsize
\centering
\caption{Liste aller Anspruchsgruppen}
\label{potentielle_anspruchsgruppen}

%% - set width of columns
%% - ---------------------------------------

\begin{tabular}{ |l|l| }

%% - Header row with shadowing
%% - --------------------------------------

\hline
\tableheaderbgcolor
\textbf{Anspruchsgruppe} & \textbf{Zugreifbare Datenbestände}\\ 
\hline

%% - table data
%% - --------------------------------------

\textbf{Produktentwicklung} & Quellcode                     \\ \hline
\textbf{Gebäudemanagement} & Zutrittskontrolle              \\ \hline
\textbf{Marketing} & Firmenidentifikation                   \\ \hline
\textbf{Einkauf} & Offerten                                 \\ \hline
\textbf{Kommunkation} & Interne Mitarbeiterinformationen    \\ \hline
\textbf{Personaldienst} & Personaldossiers                  \\ \hline
\textbf{Verkauf} & Produkteofferten                         \\ \hline
\textbf{Verkaufsanbahnung} & Anonymisierte Produktdaten     \\ \hline
\textbf{Finanzen} & Buchhaltungsdaten                       \\ \hline
\textbf{Qualitätssicherung} & Verfahrensbeschriebe          \\ \hline
\textbf{Rechtsabteilung} & Verträge                         \\ \hline
\textbf{IT} &  Systempasswörter                             \\ \hline
\textbf{Schulung} & Prüfungsfragen                          \\ \hline

\end{tabular}
\end{table}

\subsubsection{Ermittlung der Anspruchsgruppen für die Befragung}
\subfile{sections/Teil_2_Texte/anspruchsgruppen_ermittlung.tex}

\paragraph*{Definition der Schutzbedarfskategorien}\mbox{}
\label{Schutzbedarfskategorien}

\begin{sloppypar}
Die nachfolgend beschriebenen Schutzbedarfskategorien wurden gemäss der Publikation \citep{bsi_bsi-standard_2008} vom \acrlong{bsi} (\acrshort{bsi}) definiert. 

%% abkürzungsverzeichnis, BibliograFie

\end{sloppypar}

%% - table metadata
%% - --------------------------------------

\begin{table}[H]
\centering
\tablefontsize	
\caption{Schutzbedarfskategorien}
\label{schutzbedarfskategorien}

%% - set width of columns
%% - ---------------------------------------

\begin{tabular}{ |p{4cm}|p{6cm}|{c}| }

%% - Header row with shadowing
%% - --------------------------------------

\hline
\tableheaderbgcolor
\textbf{Schutzbedarfskategorie} & \textbf{Ausprägung} & \textbf{Schutzbedarf-Faktor}\\ 
\hline

%% - table data
%% - --------------------------------------

\textbf{normal} & Normaler Schutzbedarf & 1 \\ \hline
\textbf{hoch} & Erhöhter Schutzbedarf & 2  \\ \hline
\textbf{sehr hoch} & Höchster Schutzbedarf & 3 \\ \hline

\end{tabular}
\end{table}

\paragraph*{Berechnung der Relevanz der Anspruchsgruppen}\mbox{}

\begin{sloppypar}
Bei der Beschreibung von allgemeinen Schutzzielen wird oft das Tripel \textit{\acrlong{cia}} (\acrshort{cia})\footnote{Vertraulichkeit, Integrität, Verfügbarkeit} eingesetzt (\cite{harris_cissp_2013}). Für die Berechnung der Relevanz der Anspruchsgruppen werden daraus die beiden Prinzipien Vertraulichkeit $(C)$ und Integrität $(I)$ verwendet.

Für die aufgeführten zugreifbaren Datenbestände der Anspruchsgruppen wird die Relevanz als Produkt $P =  Vertraulichkeit \cdot Integrit"at$ berechnet. Dieses Produkt wird später bei der Definition des Selektionskriteriums eingesetzt. 

Die Festlegung der Vertraulichkeit, respektive der Integrität für die zugreifbaren Datenbestände basiert auf Kurzgesprächen, welche vom Autor mit Vertretern aller Anspruchsgruppen auf Zufallsbasis (Flurgespräch, Zugfahrt, Kaffeepause, etc.) und somit informell  zu deren Datenbeständen geführt wurden. 
\end{sloppypar}

%% - table metadata
%% - --------------------------------------

\begin{table}[H]
\centering
\tablefontsize	

\caption{Relevanzberechnung von Datenbeständen der Anspruchsgruppen }
\label{relevanz_datenbestände}

%% - set width of columns
%% - ---------------------------------------

\begin{tabular}{ |l|l|{c}|{c}|{c}| }

%% - Header row with shadowing
%% - --------------------------------------

\hline
\tableheaderbgcolor
\textbf{Anspruchsgruppe} & \textbf{Zugreifbare Datenbestände} & \textbf{Vertraulichkeit $(C)$} & \textbf{Integrität $(I)$} & \textbf{Relevanz $(C \cdot I)$}\\ 
\hline

%% - table data
%% - --------------------------------------

\textbf{Produktentwicklung} & Quellcode & hoch & sehr hoch & 6\\ \hline
\textbf{Gebäudemanagement} & Zutrittskontrolle & hoch & hoch & 4  \\ \hline
\textbf{Marketing} & Firmenidentifikation & hoch & hoch & 4\\ \hline
\textbf{Einkauf} & Lieferantendaten + Offerten & normal & normal & 1\\ \hline
\textbf{Kommunikation} & Interne Mitarbeiterinformationen & hoch & hoch & 4\\ \hline
\textbf{Personaldienst} & Personaldossiers & sehr hoch & sehr hoch & 9 \\ \hline
\textbf{Verkauf} & Produkteofferten & Hoch & hoch & 6\\ \hline
\textbf{Verkaufsanbahnung} & Anonymisierte Produktdaten  & hoch & normal & 2\\ \hline
\textbf{Finanzen} & Buchhaltungsdaten & sehr hoch & sehr hoch & 9\\ \hline
\textbf{Qualitätssicherung} & Verfahrensbeschriebe  & normal & hoch & 2\\ \hline
\textbf{Rechtsabteilung} & Verträge & sehr hoch & sehr hoch & 9\\ \hline
\textbf{IT} &  Systempasswörter & sehr hoch & sehr hoch & 9\\ \hline
\textbf{Schulung} & Prüfungsfragen & hoch & sehr hoch & 6\\ \hline

\end{tabular}
\end{table}

\paragraph*{Modell zur Bestimmung der Anspruchsgruppen}\mbox{}

\begin{sloppypar}
Nachfolgend wird die berechnete Relevanz der Anspruchsgruppen in einem Modell dargestellt. Es wird eine Selektionsfläche definiert, anhand welcher die definitiven Anspruchsgruppen für die Datenerhebung bestimmt werden.

Das Modell für die Bestimmung der Anspruchsgruppen wird als eine quadratische Matrix (Menge) mit insgesamt 9 Elementen, bestehend aus 3 Zeilen und 3 Spalten aufgebaut. Der Zeilenvektor $C$ beschreibt dabei die Vertraulichkeit, der Spaltenvektor $I$ die Integrität. Als Skala für den Zeilen- und Spaltenvektor wird der Schutzbedarf-Faktor welcher bei der Definition der Schutzbedarfskategorien (siehe Seite \pageref{Schutzbedarfskategorien}) festgelegt wurde, verwendet.

Die Menge $\{M\}$ aller geordneten Paare $(C,I)$ ergibt sich aus dem kartesischen Produkt von $Vertraulichkeit \times Integrit"at$:

\[\big\{\,M\,\big\} \coloneqq \bigg\{\,(C \times I) \bigg\}\]

\end{sloppypar}

\newpage
\subparagraph*{Definition der Selektionsfläche}\mbox{}

\begin{sloppypar}
Die Selektionsfläche ist ebenfalls eine Teilmengemenge von geordneten Paaren $(C,I)$ der Menge $\{M\}$ und wird über ihre berechnete Relevanz bestimmt. Sie muss dabei für eine Selektion des Paares mindestens 3 betragen:
\[\big\{\,A\,\big\} \coloneqq \bigg\{\,(C \cdot I) \geq 3 \bigg\}\]

Die Selektionsfläche besteht somit aus der Menge $\{A\}$ von geordneten Koordinatenpaaren, welche das Selektionkriterium erfüllen:
\[\big\{\,A\,\big\} \coloneqq \big\{\;(3,1),(2,2),(1,3),(3,2),(2,3),(3,3)\;\big\}\]
Das Abbild dieser Menge $A$ ist eine echte Teilmenge von $M$, kurz $A \subsetneq M$.

\end{sloppypar}

\begin{figure}[H]
    \centering
    \begin{tikzpicture}

%% - Draw grid
%% - --------------------------------------

\draw[step=2cm,gray,very thin] (0,0) grid (6,6);

%% - Draw Grid Ticks
%% - --------------------------------------

\foreach \x in {2,4}
    \draw (\x cm,3pt) -- (\x cm,-3pt);

\foreach \y in  {2,4}
    \draw (3pt,\y cm) -- (-3pt,\y cm);

%% - Draw Arrows
%% - --------------------------------------

\draw [decoration={markings,mark=at position 1 with
    {\arrow[scale=3,>=stealth]{>}}},postaction={decorate}] (0,0) -- (6,0);

\draw [decoration={markings,mark=at position 1 with
    {\arrow[scale=3,>=stealth]{>}}},postaction={decorate}] (0,0) -- (0,6);
    
%% - label x- and y-Axis
%% - --------------------------------------

\node[scale=1.2] at (8,0) {Vertraulichkeit};
\node[scale=1.2] at (0,6.5) {Integrität};

%% - Draw x- and y- Axis scale
%% - --------------------------------------

\node[draw,scale=1,shape=rectangle,draw=none, ] at (1,-0.5) {Tief (1)};
\node[draw,scale=1,shape=rectangle,draw=none] at (3,-0.5) {Mittel (2)};
\node[draw,scale=1,shape=rectangle,draw=none] at (5,-0.5) {Hoch (3)};

\node[draw,scale=1,shape=rectangle,draw=none, rotate=90] at (-0.5,1) {Tief (1)};
\node[draw,scale=1,shape=rectangle,draw=none, rotate=90] at (-0.5,3) {Mittel (2)};
\node[draw,scale=1,shape=rectangle,draw=none, rotate=90] at (-0.5,5) {Hoch (3)};

%% - Draw visible nodes in grid
%% - --------------------------------------

\node[draw,scale=2,shape=rectangle,draw=none,gray] at (1,1) {1};
\node[draw,scale=2,shape=rectangle,draw=none,gray] at (3,1) {2};
\node[draw,scale=2,shape=rectangle,draw=none,gray] at (5,1) {3};

\node[draw,scale=2,shape=rectangle,draw=none,gray] at (1,3) {2};
\node[draw,scale=2,shape=rectangle,draw=none,gray] at (3,3) {4};
\node[draw,scale=2,shape=rectangle,draw=none,gray] at (5,3) {6};

\node[draw,scale=2,shape=rectangle,draw=none,gray] at (1,5) {3};
\node[draw,scale=2,shape=rectangle,draw=none,gray] at (3,5) {6};
\node[draw,scale=2,shape=rectangle,draw=none,gray] at (5,5) {9};

%% - Draw selection line and label
%% - --------------------------------------

\draw [red, ultra thick, dashed] (-1,4) -- (2,4) -- (2,2) -- (4,2) -- (4,0);

\node[draw,scale=1,shape=rectangle,draw=none, red] at (-2.5,4) {Selektionslinie};


\end{tikzpicture}
    \caption{Modell zur Bestimmung der Anspruchsgruppen}
    \label{fig:raster1}
\end{figure}

\paragraph*{Selektion der definitiven Anspruchsgruppen}\mbox{}

\begin{sloppypar}
Werden nun alle definierten Anspruchsgruppen anhand ihrer Koordinatenpaare $(\;C,I\;)$ in das Modell übertragen, so ergibt sich das folgende Bild:
\end{sloppypar} 

\begin{figure}[H]
    \begin{tikzpicture}

%% - Draw grid
%% - --------------------------------------

\draw[step=2cm,gray,very thin] (0,0) grid (6,6);

%% - Draw Grid Ticks
%% - --------------------------------------

\foreach \x in {2,4}
    \draw (\x cm,3pt) -- (\x cm,-3pt);

\foreach \y in  {2,4}
    \draw (3pt,\y cm) -- (-3pt,\y cm);

%% - Draw Arrows
%% - --------------------------------------

\draw [decoration={markings,mark=at position 1 with
    {\arrow[scale=3,>=stealth]{>}}},postaction={decorate}] (0,0) -- (6,0);

\draw [decoration={markings,mark=at position 1 with
    {\arrow[scale=3,>=stealth]{>}}},postaction={decorate}] (0,0) -- (0,6);
    
%% - label x- and y-Axis
%% - --------------------------------------

\node[scale=1.0] at (7.5,0) {Vertraulichkeit};
\node[scale=1.0] at (0,6.5) {Integrität};

%% - Draw x- and y- Axis scale
%% - --------------------------------------

\node[draw,scale=1,shape=rectangle,draw=none, ] at (1,-0.5) {Tief};
\node[draw,scale=1,shape=rectangle,draw=none] at (3,-0.5) {Mittel};
\node[draw,scale=1,shape=rectangle,draw=none] at (5,-0.5) {Hoch};

\node[draw,scale=1,shape=rectangle,draw=none, rotate=90] at (-0.5,1) {Tief};
\node[draw,scale=1,shape=rectangle,draw=none, rotate=90] at (-0.5,3) {Mittel};
\node[draw,scale=1,shape=rectangle,draw=none, rotate=90] at (-0.5,5) {Hoch};

%% - Draw visible nodes in grid
%% - --------------------------------------

%\node[draw,scale=2,shape=rectangle,draw=none,gray] at (1,1) {1};
%\node[draw,scale=2,shape=rectangle,draw=none,gray] at (3,1) {2};
%\node[draw,scale=2,shape=rectangle,draw=none,gray] at (5,1) {3};

%\node[draw,scale=2,shape=rectangle,draw=none,gray] at (1,3) {2};
%\node[draw,scale=2,shape=rectangle,draw=none,gray] at (3,3) {4};
%\node[draw,scale=2,shape=rectangle,draw=none,gray] at (5,3) {6};

%\node[draw,scale=2,shape=rectangle,draw=none,gray] at (1,5) {3};
%\node[draw,scale=2,shape=rectangle,draw=none,gray] at (3,5) {6};
%\node[draw,scale=2,shape=rectangle,draw=none,gray] at (5,5) {9};

%% - Draw selection line and label
%% - --------------------------------------

\draw [red, ultra thick, dashed] (0,4) -- (2,4) -- (2,2) -- (4,2) -- (4,0);
%\node[draw,scale=1,shape=rectangle,draw=none, red] at (-2.5,4) {Selektionslinie};

%% - Add datapoints
%% - --------------------------------------

\node[draw,shape=circle,cyan,very thick,minimum width=2em] at (3.5,4.5) {A};
\node[draw,shape=circle,cyan,very thick,minimum width=2em] at (2.5,3.5) {B};
\node[draw,shape=circle,cyan,very thick,minimum width=2em] at (3.5,3.0) {C};
\node[draw,shape=circle,red,very  thick,minimum width=2em] at (1.0,1.0) {D};
\node[draw,shape=circle,cyan,very thick,minimum width=2em] at (2.5,2.5) {E};
\node[draw,shape=circle,cyan,very thick,minimum width=2em] at (5.5,5.5) {F};
\node[draw,shape=circle,cyan,very thick,minimum width=2em] at (5.0,3.0) {G};
\node[draw,shape=circle,red,very  thick,minimum width=2em] at (3.0,1.0) {H};
\node[draw,shape=circle,cyan,very thick,minimum width=2em] at (4.5,4.5) {I};
\node[draw,shape=circle,red,very  thick,minimum width=2em] at (1.0,3.0) {J};
\node[draw,shape=circle,cyan,very thick,minimum width=2em] at (4.5,5.5) {K};
\node[draw,shape=circle,cyan,very thick,minimum width=2em] at (5.5,4.5) {L};
\node[draw,shape=circle,cyan,very thick,minimum width=2em] at (2.5,5.5) {M};

%% - legend
%% - --------------------------------------

\matrix  [matrix of nodes,row sep=1mm,column 1/.style={nodes={circle,scale=0.5,draw,minimum width=1em}},column 2/.style={right,font=\footnotesize}] at (13,3)
{
A   & Produktentwicklung \\
B   & Gebäudemanagement \\
C   & Marketing \\
D   & Einkauf \\
E   & Kommunkation \\
F   & Personaldienst \\
G   & Verkauf \\
H   & Verkaufsanbahnung \\
I   & Finanzen \\
J   & Qualitätssicherung \\
K   & Rechtsabteilung \\
L   & IT \\
M   & Schulung \\
};


\end{tikzpicture}

    \caption{Anwendung des Modells auf Anspruchsgruppen}
    \label{fig:raster2}
\end{figure}

\begin{sloppypar}
Die in \textcolor{cyan}{cyan} dargestellten Anspruchsgruppen sind in der \textcolor{mylightblue}{hellblauen} Selektionsfläche enthalten und erfüllen somit das Auswahlkriterium. Die \textcolor{gray}{grau} dargestellten Anspruchsgruppen erfüllen das Auswahlkriterium nicht.

Die Auswertung des Modells führt zu den nachfolgend aufgeführten Liste der definitiven Anspruchsgruppen, welche somit gleichzeitig die Zielpopulation darstellt.
\end{sloppypar}

%% - table metadata
%% - --------------------------------------

\begin{table}[H]
\tablefontsize	
\caption{Liste der definitiven Anspruchsgruppen}
\label{definitive_anspruchsgruppen}

%% - set width of columns
%% - ---------------------------------------

\begin{tabular}{ |p{8cm}| }

%% - Header row with shadowing
%% - --------------------------------------

\hline
\tableheaderbgcolor
\textbf{Anspruchsgruppe} \\ 
\hline

%% - table data
%% - --------------------------------------

\textbf{Produktentwicklung}             \\ \hline
\textbf{Gebäudemanagement}              \\ \hline
\textbf{Marketing}                      \\ \hline
\textbf{Kommunikation}                  \\ \hline
\textbf{Personaldienst}                 \\ \hline
\textbf{Verkauf}                        \\ \hline
\textbf{Finanzen}                       \\ \hline
\textbf{Rechtsabteilung}                \\ \hline
\textbf{IT}                             \\ \hline
\textbf{Schulung}                       \\ \hline

\end{tabular}
\end{table}

\begin{sloppypar}
Die Anspruchsgruppen "`Einkauf"', "`Verkaufsanbahnung"' und "`Qualitätssicherung"' werden aufgrund ihrer Positionierung ausserhalb der Selektionsfläche vom weiteren Datenerhebungsverfahren ausgeschlossen.
\end{sloppypar}

\end{document}