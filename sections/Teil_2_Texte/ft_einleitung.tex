\documentclass[../../main.tex]{subfiles}

\begin{document}

\begin{sloppypar}
In diesem zweiten Teil wird die Firma {\companylong} als solches beschrieben. Es wird dabei auf die Firmengeschichte, Firmenorganisation, Unternehmenswerte, Kompetenzmodell und die vorherrschende Firmenkultur eingegangen. Die Wachstumsaspekte (vor allem im internationalen Bereich) werden beleuchtet.

Anschliessend wird beschrieben, anhand welcher Anspruchsgruppen der Security Awareness Ist-Zustand im Unternehmen mittels einer Online-Umfrage und persönlichen Interviews untersucht wird. Neben der allgemeinen Vorgehensbeschreibung wird die Zielpopulation und die darin enthaltene Stichprobe definiert. In einem weiteren Schritt werden die zur Erhebung der Ist-Situation relevanten Themenkomplexe festgelegt und begründet. Danach wird die Grundstruktur der Umfrage beschrieben (Fragebogenabschnitte und Antworttypen). Die detaillierten Fragen des Fragenkataloges werden definiert, die Absicht der Frage wird deklariert und mit den zuvor definierten Antworttypen und Themenkomplexen verknüpft. Für die persönlichen Befragungen wird die Interviewform und deren Auswertungsmethode beschrieben.

Das hier grob skizzierte Vorgehen basiert auf Erkenntnissen des Autors, welche aus der Durchführung von früheren Umfragen innerhalb der {\companyshort} gewonnen wurden. Es wurde deshalb ausgewählt, weil es bei der ersten Ausführung als Nullmessung dient und mit dem exakt gleichen Fragesatz zu einem späteren Zeitpunkt (Beispielsweise nach einem Jahr oder nach der Implementation von Korrekturmassnahmen) wiederholt werden kann. Die dabei festgestellte Differenz zeigt auf, in welchen Themenkomplexen sich das Sicherheitsbewusstsein zwischen den beiden Messungen verändert hat.

Ein Nachteil dieser Methode liegt darin, dass aufgrund des technologischen / unternehmerischen / gesellschaftlichen Wandels bestimmte Fragen ihre Relevanz verlieren könnnen, während andere zum Zeitpunkt der Nullmessung noch irrelevante oder unbekannte Themen nicht abgefragt werden konnten.
\end{sloppypar}

\end{document}

