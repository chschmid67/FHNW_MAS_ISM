\documentclass[../../main.tex]{subfiles}

\begin{document}

\paragraph*{Analyse der Einzelfragen}\mbox{}

\begin{sloppypar}
Zur quantitativen Darstellung der Antwortauswertung der Einzelfragen wird ein Netzdiagramm\footnote{Auch bekannt unter Spinnennetzdiagramm, Radardiagramm oder Kiviat-Diagramm.} verwendet. Diese Darstellungsart eignet sich besonders gut zum Visualisieren von Evaluationen für zuvor festgelegte Kriterien zweier (oder mehrerer) Serien. 
\end{sloppypar}


\paragraph*{Semantik und Skala der Achsen}\mbox{}

\begin{sloppypar}
Für die Auswertung der Online-Fragen werden sieben Achsen verwendet. Jede dieser Achsen entspricht einem der zuvor festgelegten Themenbereiche (Siehe Kapitel \ref{beschreibung_themenbereiche}, Seite \pageref{beschreibung_themenbereiche}). Die Skala einer einzelnen Achse reicht von 0\% bis 100\%, wobei der Schnittpunkt aller sieben Achsen als 0\% definiert wird.\footnotemark
\end{sloppypar}

\footnotetext{Der Prozentwert zeigt den Erfüllungsgrad der verschiedenen Themenbereiche auf, wobei 0\% als "`nicht erfüllt"' und 100\% als "`vollständig erfüllt"' zu interpretieren sind.}

\paragraph*{Zuordnung der Fragen zum Netzdiagramm}\mbox{}

\begin{sloppypar}
Die Fragen werden anhand des ihnen zugewiesenen Themenbereiches mit ihrem \acrfull{id} in das Diagramm übertragen, dadurch werden sie zu Fragegruppen gebündelt. Einzelne Fragen können ihrer Natur nach auf mehrere Themenbereiche einwirken; dies ist durch eine Mehrfachnennung in den verschiedenen Fragegruppen sowie anhand einer entsprechenden Farbcodierung erkennbar.
\end{sloppypar}

\begin{figure}[H]
 \centering
    %%
%% - Der verwwendete Code für die Darstellung der Grafik
%% - ist geistiges Eigentum von Dominik Renzel
%% - © 2009 
%% - http://dbis.rwth-aachen.de/cms/staff/renzel
%%
%%

\begin{tikzpicture}[scale=1]
  \path (0:0cm) coordinate (O); % define coordinate for origin

  % draw the spiderweb
  \foreach \X in {1,...,\D}{
    \draw (\X*\A:0) -- (\X*\A:\R);
  }

  \foreach \Y in {0,...,\U}{
    \foreach \X in {1,...,\D}{
      \path (\X*\A:\Y*\R/\U) coordinate (D\X-\Y);
      \fill (D\X-\Y) circle (1pt);
    }
    \draw [opacity=0.3] (0:\Y*\R/\U) \foreach \X in {1,...,\D}{
        -- (\X*\A:\Y*\R/\U)
    } -- cycle;
  }

  % define labels for each dimension axis (names config option)
  \path (1*\A:\L+2mm) node (L1) {\scriptsize KNOWHOW};
  \path (2*\A:\L) node (L2) {\scriptsize SOCIAL};
  \path (3*\A:\L+5mm) node (L3) {\scriptsize DEVICE};
  \path (4*\A:\L+5mm) node (L4) {\scriptsize COMPANY};
  \path (5*\A:\L) node (L5) {\scriptsize ATTACK};
  \path (6*\A:\L+2mm) node (L6) {\scriptsize WPLACE};
  \path (7*\A:\L+5mm) node (L7) {\scriptsize DATASEC};

%% - Achse KNOWHOW
%% - --------------------------------------

\matrix [nodes={minimum size=5mm},column sep=1mm, row sep = 1mm] at (1*\A:\L+20mm)
{
\node [draw,scale=0.7,shape=circle, minimum width=3em] {S8}; & \node [draw,scale=0.7,shape=circle, cyan, minimum width=3em] {S9}; & \node [draw,scale=0.7,shape=circle, cyan, minimum width=3em] {S11};\\
\node [draw,scale=0.7,shape=circle, cyan, minimum width=3em] {S12}; & \node [draw,scale=0.7,shape=circle, minimum width=3em] {V31};\\
};

%% - Achse SOCIAL
%% - --------------------------------------

\matrix [nodes={minimum size=5mm},column sep=1mm, row sep = 1mm] at (2*\A:\L+15mm)
{
\node [draw,scale=0.7,shape=circle, minimum width=3em] {S13}; & \node [draw,scale=0.7,shape=circle, minimum width=3em] {S14}; & \node [draw,scale=0.7,shape=circle, minimum width=3em] {S15}; & \node [draw,scale=0.7,shape=circle, cyan, minimum width=3em] {U19}; \\
\node [draw,scale=0.7,shape=circle, cyan, minimum width=3em] {U24}; & \node [draw,scale=0.7,shape=circle, cyan, minimum width=3em] {U26}; & \node [draw,scale=0.7,shape=circle, cyan, minimum width=3em] {V32}; & \node [draw,scale=0.7,shape=circle, cyan, minimum width=3em] {V34};  \\
\\
};

%% - Achse DEVICE
%% - --------------------------------------

\matrix [nodes={minimum size=5mm},column sep=1mm, row sep = 1mm] at (3*\A-8:\L+20mm)
{
\node [draw,scale=0.7,shape=circle, cyan, minimum width=3em] {S9}; & \node [draw,scale=0.7,shape=circle, minimum width=3em] {S10}; & \node [draw,scale=0.7,shape=circle, cyan, minimum width=3em] {U18}; \\
\node [draw,scale=0.7,shape=circle, cyan, minimum width=3em] {U19}; & \node [draw,scale=0.7,shape=circle, cyan, minimum width=3em] {U20}; & \node [draw,scale=0.7,shape=circle, minimum width=3em] {V30};  \\
};

%% - Achse COMPANY
%% - --------------------------------------

\matrix [nodes={minimum size=5mm},column sep=1mm, row sep = 1mm] at (4*\A-18:\L+15mm)
{
\node [draw,scale=0.7,shape=circle, cyan, minimum width=3em] {U17}; & \node [draw,scale=0.7,shape=circle, cyan, minimum width=3em] {U18}; & \node [draw,scale=0.7,shape=circle, cyan, minimum width=3em] {U23};  \\
\node [draw,scale=0.7,shape=circle, cyan, minimum width=3em] {U24}; & \node [draw,scale=0.7,shape=circle, minimum width=3em] {U25}; & \node [draw,scale=0.7,shape=circle, cyan, minimum width=3em] {U26};  \\
};

%% - Achse ATTACK
%% - --------------------------------------

\matrix [nodes={minimum size=5mm},column sep=1mm, row sep = 1mm] at (5*\A:\L+15mm)
{
\node [draw,scale=0.7,shape=circle, cyan, minimum width=3em] {S9}; & \node [draw,scale=0.7,shape=circle, cyan, minimum width=3em] {S12}; & \node [draw,scale=0.7,shape=circle,cyan, minimum width=3em] {U19}; & \node [draw,scale=0.7,shape=circle, cyan, minimum width=3em] {U22};  \\
\node [draw,scale=0.7,shape=circle, cyan, minimum width=3em] {V28}; & \node [draw,scale=0.7,shape=circle, cyan, minimum width=3em] {V32}; & \node [draw,scale=0.7,shape=circle, minimum width=3em] {V33}; & \node [draw,scale=0.7,shape=circle, cyan, minimum width=3em] {V34};  \\
};

%% - Achse WPLACE
%% - --------------------------------------

\matrix [nodes={minimum size=5mm},column sep=1mm, row sep = 1mm] at (6*\A:\L+20mm)
{
\node [draw,scale=0.7,shape=circle, cyan, minimum width=3em] {S11}; & \node [draw,scale=0.7,shape=circle, cyan, minimum width=3em] {U17}; & \node [draw,scale=0.7,shape=circle, cyan, minimum width=3em] {U21};  \\
\node [draw,scale=0.7,shape=circle, cyan, minimum width=3em] {U22}; & \node [draw,scale=0.7,shape=circle, cyan, minimum width=3em] {U23}; & \node [draw,scale=0.7,shape=circle, cyan, minimum width=3em] {U24};  \\
};

%% - Achse DATASEC
%% - --------------------------------------

\matrix [nodes={minimum size=5mm},column sep=1mm, row sep = 1mm] at (7*\A+7:\L+15mm)
{
\node [draw,scale=0.7,shape=circle, cyan, minimum width=3em] {S11}; & \node [draw,scale=0.7,shape=circle, cyan, minimum width=3em] {U18}; & \node [draw,scale=0.7,shape=circle, cyan, minimum width=3em] {U19}; \\
};

%% - Achse DATASEC
%% - --------------------------------------

\matrix [nodes={minimum size=5mm},column sep=1mm, row sep = 1mm] at (7*\A-11:\L+15mm)
{
\node [draw,scale=0.7,shape=circle, cyan, minimum width=3em] {U20}; & \node [draw,scale=0.7,shape=circle, cyan, minimum width=3em] {U21}; & \node [draw,scale=0.7,shape=circle, cyan, minimum width=3em] {V28};  \\
\node [draw,scale=0.7,shape=circle, minimum width=3em] {V29}; \\
};

%% - Add Legend and draw box around it
%% - --------------------------------------

\matrix [column sep=1mm] at (-5.5,-4.5)
{
\node [draw,scale=0.3,shape=circle, cyan, minimum width=3em] {}; & \node [anchor=west] {\footnotesize Mehrfachverwendung}; \\
\node [draw,scale=0.3,shape=circle, minimum width=3em] {}; & \node [anchor=west] {\footnotesize Einmalverwendung}; \\
};

\draw (-7.7,-5.1) rectangle (-3.4,-3.8);

\end{tikzpicture}


 \caption{Netzdiagramm mit zugeordneten Einzelfragen}
 \label{Netzdiagramm Schema fragepositionen}
\end{figure}

\paragraph*{Projektion der Umfragewerte auf die Achsen}\mbox{}

\begin{sloppypar}
Für die Auswertung der Online-Fragen werden wie zuvor in Kapitel \ref{allgemeine_vorgehensbeschreibung},  Seite \pageref{allgemeine_vorgehensbeschreibung} bereits beschrieben, nur Antworten auf Fragen vom Typ A und B verwendet.

Um den Erfüllungsgrad eines Themenbereiches darstellen zu können, werden die Durchschnittswerte respektive der Ja-Anteil der relevanten Antworten auf ihre jeweiligen Themenbereichsanteile projiziert und anschliessend summiert.
\newpage
Nachstehend ein Beispiel für die Berechnung des Erfüllungsgrades für den Themenbereich "`SOCIAL"'. Für alle sieben Themenbereiche wird für die Bestimmung ihres jeweiligen Abdeckungsgrades dasselbe Verfahren angewandt.\footnotemark
\end{sloppypar}

%% - table metadata
%% - --------------------------------------

\sloppy 

\begin{table}[H]
\tablefontsize	
\centering
\caption{Beispielberechnung Themenabdeckung SOCIAL}
\label{Beispielberechnung Themenabdeckung}
\begin{tabular}{p{5.0cm}lllllllll}
\rowcolor[HTML]{BBDAFF} 
\multicolumn{10}{l}{\cellcolor[HTML]{BBDAFF}\textbf{Block 1: Zusammenfassung}}\\
\hline
\textbf{Themenbereich}                    & \multicolumn{9}{l}{SOCIAL}\\
\textbf{Relevante Einzelfragen}           & \multicolumn{9}{l}{S13,S14,S15,U19,U24,U26,V32,V34}\\
\textbf{Erfüllungsgrad Themenbereich}     & \multicolumn{9}{l}{64\%}\\
\textbf{}                                 &      &      &      &      &      &      &      &      &       \\[-3ex]
\rowcolor[HTML]{BBDAFF} 
\multicolumn{10}{l}{\cellcolor[HTML]{BBDAFF}\textbf{Block 2: Antworten der relevanten Einzelfragen}}\\
\hline
\textbf{ID}                               & S13  & S14  & S15  & U19  & U24  & U26  & V32  & V34  &\\
\textbf{Fragetyp}                         & A    & A    & A    & A    & B    & B    & A    & A    &\\
\hline
\textbf{User 1}                           & 57   & 34   & 67   & 23   & 1    & 3    & 12   & 92   &\\
\textbf{User 2}                           & 23   & 43   & 23   & 98   & 2    & 1    & 15   & 98   &\\
\textbf{User 3}                           & 92   & 28   & 95   & 50   & 2    & 1    & 23   & 88   &\\
\textbf{User 4}                           & 88   & 93   & 71   & 45   & 2    & 3    & 10   & 28   &\\
\textbf{User 5}                           & 7    & 38   & 67   & 7    & 2    & 2    & 60   & 12   &\\
\textbf{User 6}                           & 62   & 37   & 36   & 49   & 2    & 1    & 27   & 50   &\\
\textbf{}                                 &      &      &      &      &      &      &      &      &\\[-3ex]
\rowcolor[HTML]{BBDAFF} 
\multicolumn{10}{l}{\cellcolor[HTML]{BBDAFF}\textbf{Block 3: Antworten Analogskala}}\\
\hline
\textbf{Durchschnitt (AVG)}               & 57.3 & 35.0 & 61.7 & 57.0 &      &      & 16.7 & 92.7 &\\
\textbf{Standardabweichung (STDEV)}       & 28.2 & 6.2  & 29.6 & 31.0 &      &      & 4.6  & 4.1  &\\
\textbf{}                                 &      &      &      &      &      &      &      &      &\\[-3ex]
\rowcolor[HTML]{BBDAFF} 
\multicolumn{10}{l}{\cellcolor[HTML]{BBDAFF}\textbf{Block 4: Antworten Auswahlmenü}}\\
\hline
\textbf{Ja}                               &      &      &      &      & 16.7\% & 50\%  &      &      &\\
\textbf{Nein und unbekannt}               &      &      &      &      & 83.3\% & 50\%  &      &      &\\
\hline
\textbf{Korrekte Antwort (Soll)}                 &      &      &      &      & \textbf{NEIN}   & \textbf{JA}    &      &      &    \\
\textbf{}                                 &      &      &      &      &        &       &      &      &\\[-3ex]
\rowcolor[HTML]{BBDAFF} 
\multicolumn{9}{l}{\cellcolor[HTML]{BBDAFF}\textbf{Block 5: Abdeckungsberechnung Themenbereich}}& TOTAL\\
\hline
\textbf{Frageanteil am Themenbereich} & 12.5 & 12.5 & 12.5 & 12.5 & 12.5 & 12.5 & 12.5 & 12.5 &  \\
\textbf{Prozentuale Erfüllung}           & 57\%  & 35\%  & 62\%  & 57\%  & 83\%  & 50\%  & 17\% & 93\% &\\
\textbf{Relativer Erfüllungsbetrag}      & 7.2  & 4.4  & 7.7  & 7.1  & 10.4  & 6.3  & 2.1  & 11.6 & 64   
\end{tabular}
\end{table}

\footnotetext{Die Tabelle setzt sich aus insgesamt \textbf{fünf Blöcken} zusammen: \newline 
\textbf{Block 1} ist die Zusammenfassung, welche neben der Nennung des Themenbereiches die ID's der Einzelfragen aus der dem Themenbereich zugeordneten Fragegruppe auflistet. Der berechnete Erfüllungsgrad für den Themenbereich wird in Prozent ausgedrückt. \newline 
\textbf{Block 2} listet pro ID den Fragetyp und die erhaltenen Antworten tabellarisch auf. \newline 
\textbf{Block 3} stellt die für die Antworten von \textbf{Typ A} (Analogskala) berechneten Werte arithmetischer Durchschnitt (AVG) und Standardabweichung (STDEV) dar. "`Unbekannt"' Antworten werden für die Berechnung von Durchschnitt und Standardabweichung nicht berücksichtigt. \newline
\textbf{Block 4} stellt für Antworten von \textbf{Typ B} (Auswahlmenü) den berechneten prozentualen Anteil der Ja-Antworten dar. Die Nein-Antworten werden zusammen mit den "`Unbekannt"' Antworten als  Komplement zu 100 \% ausgewiesen. Es wird pro Frage definiert, ob der "`Ja"'- oder der "`Nein"'-Anteil für die Berechnung des Erfüllungsgrades der Frage relevant ist. \newline
\textbf{Block 5} zeigt den Frageanteil, welcher jede Frage vom Themenbereich abdeckt. Dieser Frageanteil ist im Regelfall gleichmässig verteilt; hier könnte (falls gewünscht) eine Gewichtung der Antworten vorgenommen werden. Die Summe aller Frageanteile beträgt jedoch immer 100. In der letzten Zeile wird für jede Frage der Erfüllungsgrad des Frageanteils anhand des zuvor bestimmten Durchschnittswertes (Typ A), resp. aus dem relevanten prozentualen Ja-, resp.  Nein Anteil (Typ B) berechnet.}

\end{document}