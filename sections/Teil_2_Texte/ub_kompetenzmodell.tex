\begin{sloppypar}
Das Kompetenzmodell der \companyshort{} hängt eng mit den eben vorgestellten Unternehmenswerten zusammen. Das Modell umfasst fünf Kompetenzgebiete mit einem jeweils spezifischen Hauptfokus. Die Kompetenzen der Mitarbeitenden sollen gezielt angesprochen und aufgebaut werden. Das Kompetenzmodell wird hier nur auszugsweise wiedergegeben:
\end{sloppypar}

\paragraph*{Umgestaltung des Bankensektors}\mbox{}
\begin{sloppypar}
Unter "`Umgestaltung"' werden diejenigen Fähigkeiten verstanden welche nötig sind, um die im Finanzsektor anstehenden Veränderungen bewältigen zu können, Markttrends frühzeitiger als die Mitbewerber zu erkennen und so die Zukunft des Bankings entscheidend mitzugestalten:
\begin{itemize}
  \item Generierung von hohem Kundennutzen
  \item Umfassendes unternehmerisches Verhalten in der eigenen Firma 
  \item Umgang mit Unsicherheiten und stetiger Veränderung
\end{itemize}
\end{sloppypar}

\paragraph*{Aufbau einer Vernetzungskultur}\mbox{}
\begin{sloppypar}
Durch den Aufbau tragfähiger und intensiv genutzter sozialer Netzwerke über persönliche und geschäftliche Kontakte wird die Effektivität der Mitarbeitenden gesteigert. Zukünftige Herausforderungen werden durch die Vernetzung besser bewältigt und absorbiert:
\begin{itemize}
  \item Gegenseitige Unterstützung und Wissenstransfer
  \item Produktives Konfliktmanagement (Umgang mit Spannungen, schwierigen Situationen) 
  \item Einsatz von wirksamer, effizienter Kommunikation 
\end{itemize}
\end{sloppypar}

\paragraph*{Aufbau Persönlicher Kompetenzen}\mbox{}
\begin{sloppypar}
Die persönliche Entwicklung der Mitarbeitenden spricht den Aufbau sozialer Fähigkeiten durch Eigenverantwortung, Selbstreflexion und Kritikfähigkeit an:
\begin{itemize}
  \item Souveräner Umgang mit stressvollen Situationen
  \item Nachvollziehbare Entscheidungen treffen 
  \item Aufbau von Vertrauen, Akzeptanz von konstruktiver Kritik
\end{itemize}
\end{sloppypar}

\paragraph*{Aufbau von Führungskompetenzen}\mbox{}
\begin{sloppypar}
Durch ein Klima der Wertschätzung und sinnvollen, erreichbaren Zielen werden die individuellen Fähigkeiten von Mitarbeitern mit den Kompetenzen von Teams und den Zielen des Unternehmens aligniert. Die Mitarbeitenden werden dazu motiviert, ihren Beitrag zu leisten durch:
\begin{itemize}
  \item Führung durch sinnvolle, nachvollziehbare Entscheide
  \item Förderung durch individuelle Entwicklungsperspektiven 
\end{itemize}
\end{sloppypar}

\paragraph*{Aufbau von Fachkompetenzen}\mbox{}
\begin{sloppypar}
Sowohl methodische als auch Kenntnisse der Prozesse ermöglichen den Aufbau von Fachkompetenzen. Diese fliessen wiederum in hergestellte Produkte und angebotene Services ein:
\begin{itemize}
  \item Wissen über bestehende Prozesslandschaft, Prozesstreue bei der Ausführung
  \item Beherrschung des eigenen Spezialgebietes und Anwendung von Methodiken 
\end{itemize}
\end{sloppypar}
