\begin{sloppypar}
Das Unternehmen besitzt seit 2014 ein neues Wertemodell. Das Modell soll die als zentral und wichtig identifizierten inneren Unternehmenswerte \textit{Nexus, Insight, Rigour, Courage} und \textit{Drive} repräsentieren. Im Rahmen eines Kulturentwicklungsprogrammes wurden diese Werte nach der "`nextexpertizer"' Methode\footnote{siehe \url{www.nextpractice.de/kulturentwicklung.html}} von Prof. Dr. Peter Kruse (\dag{} 2015) evaluiert und an die Belegschaft kommuniziert. Die Grundidee dabei ist, dass ein Unternehmen dann stark ist wenn die Mitarbeitenden eine gemeinsame Kultur und damit einhergehend die gleichen Werte teilen und diese Werte auch in ihren Arbeitsalltag einbinden. Nachstehend sind diese Unternehmenswerte beschrieben.
\end{sloppypar}

\paragraph*{Nexus -- Verknüpfung}\mbox{}

\begin{sloppypar}
Dieser Wert repräsentiert die Vernetzung der Mitarbeitenden untereinander. Er soll dabei vor allem den Wissensaustausch und die Zusammenarbeit unter den Mitarbeitenden als Verpflichtung im Arbeitsalltag verankern.
\end{sloppypar}

\paragraph*{Insight -- Einblick}\mbox{} 

\begin{sloppypar}
Dieser Wert symbolisiert Aufgeschlossenheit, Empathie und das Wissen über das grosse Ganze des Unternehmens. Die Mitarbeitenden sollen dadurch besser mit Veränderungen im geschäftlichen Ökosystem umgehen können, Chancen wahrnehmen und zukünftige Ressourcen für das Unternehmen erkennen können.
\end{sloppypar}

\paragraph*{Rigour -- Strenge}\mbox{} 

\begin{sloppypar}
Dieser Wert hält die Mitabeitenden dazu an, organisiert, standardisiert und konsequent die Komplexität ihres Arbeitsalltages anzugehen, sowie Standards einzuhalten und den definierten Prozessen zu folgen. Es wird vermittelt, dass es ist notwendig ist Normen zu akzeptieren und interne oder regulatorische Regeln zu befolgen.
\end{sloppypar}

\paragraph*{Courage -- Mut}\mbox{} 

\begin{sloppypar}
Dieser Wert vermittelt den Mitarbeitenden, dass sie aktiv Verantwortung im Sinne von Eignerschaft übernehmen, ihre persönliche Komfortzone verlassen und Grenzen überschreiten sollen. Veränderungen sollen grundsätzlich positiv wahrgenommen und begrüsst werden. 
\end{sloppypar}

\paragraph*{Drive -- Antrieb}\mbox{} 

\begin{sloppypar}
Dieser Wert ermutigt die Mitarbeitenden dazu neugierig zu sein, aus Fehlern zu lernen und agil zu bleiben. Die persönliche Bereitschaft, bei der Einschätzung von zukünftigen Herausforderungen auch Unsicherheiten zu akzeptieren, werden als von entscheidender Bedeutung für die Fähigkeit des Unternehmens, wertschöpfende Lösungen anzubieten, betrachtet.
\end{sloppypar}
