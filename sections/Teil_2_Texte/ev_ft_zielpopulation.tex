\documentclass[../../main.tex]{subfiles}

\begin{document}

\paragraph*{Grundgesamtheit}\mbox{}

\begin{sloppypar}
Als Grundgesamtheit der Empirie werden grundsätzlich alle internen Mitarbeitenden (zirka 1'900 Personen) an allen Standorten der {\group} genannt.

\subparagraph*{Ausschluss von der Grundgesamtheit}\mbox{}

Externe Mitarbeitende und Lieferanten werden aufgrund ihrer Zugehörigkeit zu anderen Unternehmen von der Empirik ausgeschlossen. Ebenso werden diejenigen Mitarbeiter nicht miteinbezogen, welche an einem \acrshort{bpo} Standort arbeiten. Als Grund hierfür wird das für diese Standorte typische, sich jedoch fundamental vom Rest der Gruppengesellschaften  unterscheidende Geschäftsmodell (operationeller Betrieb eines Bankinstitutes versus Softwareentwicklung) angeführt.

Aus \todo{Kerstin Reuter fragen}betriebsrechtlichen Gründen werden die Standorte in Deutschland in der Empirik ebenfalls nicht berücksichtigt. Als Grund hierfür wird das  Deutsche Bundesdatenschutzgesetz BDSG angeführt. Massgeblich für diese Entscheidung sind dafür die Paragrafen \S8 "`Datenerhebung und -speicherung für eigene Geschäftszwecke"' und \S40 "`Geschäftsmässige Datenerhebung und -speicherung für Zwecke der Markt- oder Meinungsforschung"' (\cite{bmjv_bundesdatenschutzgesetz_1990}). Die Erfüllung dieser Paragrafen würde dazu führen, dass die Umfrage sowohl zeitlich als auch inhaltlich stark beeinflusst werden würde.

Die verbleibende Grundgesamtheit stellt eine hinreichend heterogene, das Unternehmen sinnvoll repräsentierende Menge dar.
\end{sloppypar}

\paragraph*{Zielpopulation}\mbox{}

\begin{sloppypar}
Die Zielpopulation beschränkt sich somit auf die Niederlassungen in der Schweiz, Frankreich, England, Südostasien und Ozeanien, welche den selektierten Anspruchsgruppen angehören. Dies führt zu einer Grössenordnung von zirka 830 Personen für die Zielpopulation mit welcher die Umfrage und die Interviews durchgeführt werden.
\end{sloppypar}

\subparagraph*{Stichprobenbildung und Repräsentanz}\mbox{}

\begin{sloppypar}
Um eine gute Repräsentanz\footnote{"`\textbf{Repräsentanz} setzt voraus, dass alle Einheiten einer vorab definierten Grundgesamtheit eine berechenbare Chance haben, für die Stichprobe ausgewählt zu werden"' (\citeauthor{berekoven_marktforschung:_2009} \citeyear{berekoven_marktforschung:_2009}, Seite 107).} zu gewährleisten, wurden die Vertreter der Anspruchsgruppen aus der Zielpopulation für die Stichprobe gemäss dem Hinweis von \textsc{Fahrmeier} ausgewählt:

\begin{quote}
"`darauf zu achten, dass die Stichprobe ein möglichst getreues Abbild der Gesamtpopulation ist. Dies erreicht man insbesondere durch zufällige Stichproben, wobei zufällig nicht mit willkürlich gleichzusetzen ist, sondern beispielsweise meint, dass jede statistische Einheit dieselbe Chance hat, in die Stichprobe aufgenommen zu werden"' (\textsc{Fahrmeier} u.a. 1997, S. 14, zitiert nach \cite{kirchhoff_fragebogen:_2010}, S. 15).
\end{quote}

Die Stichprobe ist somit vom der Verfahrensbildung her konstruiert (vgl. \cite{berekoven_marktforschung:_2009}, S. 49). Diese bewusste Auswahl wurde in Anlehnung an das von \cite{berekoven_marktforschung:_2009} beschriebene Cut-off-Verfahren getroffen:

\begin{quote}
"`Dabei beschränkt man die Erhebung auf solche Elemente der Grundgesamtheit, denen für den Untersuchungstatbestand ein besonderes Gewicht zukommt. Das Verfahren eignet sich nur dann, wenn die einzelnen Elemente in der Grundgesamtheit ein starkes Ungleichgewicht aufweisen und relativ wenigen Elementen ein sehr hoher Erklärungsbeitrag [\dots] zuzumessen ist"' (\citeauthor{berekoven_marktforschung:_2009} \citeyear{berekoven_marktforschung:_2009}, Seite 52).
\end{quote}

Stichprobenausfälle (d.h. das Nichtreagieren auf den Fragebogen bei der Erhebung der Daten) kann zu einer Schweigeverzerrung bei der Auswertung führen. Dieser partielle Antwortausfall wird bewusst in Kauf genommen. 
\end{sloppypar}

\end{document}
