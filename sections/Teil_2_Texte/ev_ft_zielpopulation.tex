\documentclass[../../main.tex]{subfiles}

\begin{document}

\paragraph*{Zielpopulation}\mbox{}

\begin{sloppypar}
Als Zielpopulation (Grundgesamtheit) der Empirie werden potentiell alle internen Mitarbeitenden an allen Standorten der Aquina AG genannt, welche den selektierten Anspruchsgruppen angehören. Dies führt zu der Grössenordnung von ($n \approx 1200$) für die Zielpopulation.
\end{sloppypar}

\subparagraph*{Ausschluss von der Zielpopulation}\mbox{}

\begin{sloppypar}
Aus betriebsrechtlichen Gründen werden die Standorte in Deutschland in der Empirik nicht berücksichtigt. Als Grund hierfür wird das  Deutsche Bundesdatenschutzgesetz BDSG angeführt. Massgeblich für diese Entscheidung sind dafür die Paragrafen \S8 "`Datenerhebung und -speicherung für eigene Geschäftszwecke"' und \S40 "`Geschäftsmässige Datenerhebung und -speicherung für Zwecke der Markt- oder Meinungsforschung"' (\cite{bmjv_bundesdatenschutzgesetz_1990}). Die Vorschriften zur Erfüllung dieser Paragrafen würde dazu führen, dass die Umfrage und somit die Masterarbeit nicht mehr in der vorgegebenen Zeit durchgeführt werden kann.
\end{sloppypar}

\paragraph*{Stichprobe}\mbox{}
\begin{sloppypar}
Die Stichprobe beschränkt sich somit auf die Niederlassungen in der Schweiz, Frankreich, England, Singapur, Philippinen und Australien. Dies führt zu ($n \approx 750)$ als Grössenordnung für die Stichprobe mit welcher die Umfrage und die Interviews durchgeführt werden.
\end{sloppypar}

\subparagraph*{Stichprobenselektivität}\mbox{}

\begin{sloppypar}
Stichprobenausfälle (d.h. das Nichtreagieren auf den Fragebogen bei der Erhebung der Daten) kann zu einer Schweigeverzerrung bei der Auswertung führen. Dieser partielle Antwortausfall wird bewusst in Kauf genommen. 
\end{sloppypar}

\end{document}
