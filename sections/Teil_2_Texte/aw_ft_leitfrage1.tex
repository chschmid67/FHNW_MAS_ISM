\documentclass[../../main.tex]{subfiles}

\begin{document}

\begin{sloppypar}
Frage: Was bedeutet der Begriff "`Security Awareness"' für Dich persönlich?
\end{sloppypar}

%% -
%% - keep this block together!!!!!!!!!!!!!!
%% - --------------------------------------
%% -
\addtocounter{figure}{1}\begin{figure}[H]
    \pgfplotstableread[col sep=comma, header=false]{
% -- <percent value>, <startpoint from above> , <label>
      51,0,  Persönliches Verhalten
      43,0.7,  Datenschutz/sicherheit/klassifizierung/zugriff
      39,1.4,  Risikobewusstsein
      35,2.1,  Wissen über Bedrohungen
      34,2.8,  Kenntnisse Securitypolicies
      29,3.5,  Betrifft die physische Sicherheit
      25,4.2,  Umgang mit Passwörtern
      19,4.9,  Umgang mit Firmendaten
      18,5.6,  Security Schulung
      16,6.3,  Umgang mit Email
      9,7.0,  Verhalten in der Öffentlichkeit
      9,7.7,  Weltanschauung/Kultur/Soziales
      8,8.4,  Produktesicherheit
      6,9.1,  Umgang mit persönlichen Daten
      6,9.8,  Umgang mit Kundendaten
      6,10.5,  Security Incidents
      6,11.2,  Technische Hilfsmittel
      4,11.9,  Verantwortung Management
      2,12.6,  Erkennung verdächtige Aktivitäten
}\datatable

\begin{tikzpicture}

  \begin{axis}[
%    height = 5.5cm,
    xbar,
    y=-.7cm,
    enlarge y limits={abs=0.45cm},
    axis x line       = none,
    tickwidth         = 0pt,
    y axis line style = { opacity = 0 },
   yticklabels from table={\datatable}{2},
    ytick=data,
%    yticklabel style={text width=9cm,align=right},
    nodes near coords,
    nodes near coords align={horizontal},
%    nodes near coords={\pgfmathprintnumber\pgfplotspointmeta\%},
    ]
    \addplot table [y=1, x=0] {\datatable};
  \end{axis}
  
  % --
% -- draw surrounding box
% --
  \node[
      draw=black, very thin,
      minimum width=\textwidth,
      fit=(current bounding box.north west) (current bounding box.south east),
    ]at (current bounding box.center){};
    
\end{tikzpicture}
    \vspace*{-5mm}
    \caption*{Abbildung \thefigure: Auswertung Leitfrage 1 (Anzahl Nennungen der Sammelbegriffe $\vert$ $\Sigma: 365$, $n: 265$)}
    \label{pgfplot_leitfrage1}
\end{figure}
\addcontentsline{lof}{figure}{\numberline {\thefigure}{\ignorespaces Auswertung Leitfrage 1}}
%% -
%% - keep this block together!!!!!!!!!!!!!!
%% - --------------------------------------
%% -

\subparagraph*{Kernaussagen zu Leitfrage 1}\mbox{}

%% - table metadata
%% - --------------------------------------

\begin{table}[H]
\tablefontsize	
\caption{Kernaussagen zu Leitfrage 1}
\label{kernaussagen_leitfrage1}

%% - set width of columns
%% - ---------------------------------------

\begin{tabular}{ |p{\textwidth-1cm}| }

%% - Header row with shadowing
%% - --------------------------------------

%% - table data
%% - --------------------------------------

\hline
[\dots]sich nicht nur des Problems bewusst sein (physische Security Awareness / virtuelle Security Awareness), sondern auch über das Ausmass des Problems (Be aware of the Problem and the size of the Problem).\\ 
\hline
The term "`Security Awareness"' means staying up-to-date with the potential security threats and countermeasures. It also means knowing about do's-and-dont's of the security domain.\\
\hline
To enable Security Awareness, specific trainings and systems are required to enable both the people and the organization to proactively recognize potentials threats as well as identify and correct existing the securitymeasures. \\ 
\hline
Security Awreness bedeutet, sich der Sicherheitsthematik bewusst sein, sowohl im beruflichen wie auch im privaten Umfeld. \\ 
\hline
Security Awareness is a responsibility of each employee.\\ 
\hline
Security Awareness sollte Klarheit darüber bringen, wie man sich zu verhalten hat wenn ein Verstoss gegen die Sicherheitsvorschriften der Firma erkannt wird. \\ 
\hline
[\dots]ist das Bewusstsein für die eigene Verantwortung, die Unternehmenswerte zu schützen und sich nicht blindlings auf technische Sicherheit zu verlassen. \\ 
\hline

\end{tabular}
\end{table}

\subparagraph*{Vorschläge aus der Umfrage}\mbox{}

%% - table metadata
%% - --------------------------------------

\sloppy 

\begin{table}[H]
\tablefontsize	
\centering
\caption{Vorschläge aus Interviews und Online-Umfrage zu Leitfrage 1}
\label{vorschlaege_leitfrage1}
\begin{tabular}{ |p{3.8cm}|p{2.5cm}|p{2.5cm}|p{3.8cm}|p{3.0cm}|}

\hline
\tableheaderbgcolor
\textbf{Ideen-Beschreibung} & \textbf{Kosten-\newline schätzung} & \textbf{Aufwand-\newline schätzung} & \textbf{Nutzen} & \textbf{Quadrant}\\ 

\hline
Be aware of the problem: Sense of Urgeny wecken mit Schock &  Kosten sind tief &  Aufwand ist tief  & Wachrütteln der Mitarbeitenden, dadurch erhöhte Bereitschaft sich mit dem Thema auseinanderzusetzen. & zur Umsetzung empfohlen\\
\hline
Clear Desk Pushmails &  Kosten sind tief &  Aufwand ist tief  & Regelmässige Erinnerung daran, dass eine Clear Desk Policy besteht und diese einzuhalten ist.  & zur Umsetzung empfohlen\\
\hline
Risikoanalyse, Risikotransfer, Risikübernahme + Verantwortungsübernahme durch Management &  Kosten sind tief &  Aufwand ist mittel  & Benennung der Risikoklassen und wer von der GL dafür verantwortlich ist. Management-Entscheidung: Wir als Firma wollen uns auf Risikoklasse A,B und C konzentrieren, auf die Risikoklassen X,Y,Z nicht. Sichtbare Übernahme der Verantwortung durch das Management, hat Signalwirkung für Mitarbeitende. & zur Umsetzung empfohlen\\
\hline
Security Awareness Slogan \newline \textbf{"`It starts with you"'} \newline für Awareness-Kampagne &  Kosten sind tief &  Aufwand ist tief  & Security Awareness hat ein Motto. Die Mitarbeitenden als Leser des Mottos werden direkt angesprochen. Kurze und einprägsame Wortwahl, sofortiges Begreifen und memorisieren. & zur Umsetzung empfohlen\\
\hline
High-Level: Integration in Lighthouse/Townhall. Perspektive: Security der \companyshort{} in 5 Jahren &  Kosten sind tief &  Aufwand ist tief  & Dem Mitarbeitenden aufzeigen, dass das Thema Security Awareness auf Top-Management Level angekommen ist und dort verankert ist. Strategische Positionierung kann aufgezeigt werden, Security Awareness ist keine Eintagesfliege.& zur Umsetzung empfohlen\\
\hline

\end{tabular}
\end{table}

\end{document}