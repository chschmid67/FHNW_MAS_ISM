\documentclass[../../main.tex]{subfiles}

\begin{document}

\begin{sloppypar}
Frage: Was fehlt/was braucht es Deiner Meinung nach bezüglich Security Awareness?
\end{sloppypar}

%% -
%% - keep this block together!!!!!!!!!!!!!!
%% - --------------------------------------
%% -
\addtocounter{figure}{1}\begin{figure}[H]
    \pgfplotstableread[col sep=comma, header=false]{
% -- <percent value>, <startpoint from above> , <label>
      79,0,  Einfache Arbeitsanweisungen/Klare Richtlinien/Verhaltenskodex
      63,0.7,  Security Awareness Kampagne
      55,1.4,  Regelmässige Informationsveranstaltungen
      49,2.1,  Sicherheitsschulungen und Trainings
      22,2.8,  Persönliches Bewusstsein fördern
      20,3.5,  Technische Massnahmen
      19,4.2,  Bekannte Meldestellen: Security Officer/CISO
      15,4.9,  Vorbildverhalten des Managements
      15,5.6,  Informationen/Warnungen zu aktuellen Cyberbedrohungen
      12,6.3,  Es fehlt nichts/keine Idee
      6,7.0,  Risiko- und Datenklassifikation
}\datatable

\begin{tikzpicture}

  \begin{axis}[
%    height = 5.5cm,
    xbar,
    y=-.7cm,
    enlarge y limits={abs=0.45cm},
    axis x line       = none,
    tickwidth         = 0pt,
    y axis line style = { opacity = 0 },
   yticklabels from table={\datatable}{2},
    ytick=data,
%    yticklabel style={text width=9cm,align=right},
    nodes near coords,
    nodes near coords align={horizontal},
%    nodes near coords={\pgfmathprintnumber\pgfplotspointmeta\%},
    ]
    \addplot table [y=1, x=0] {\datatable};
  \end{axis}
  
  % --
% -- draw surrounding box
% --
  \node[
      draw=black, very thin,
      minimum width=\textwidth,
      fit=(current bounding box.north west) (current bounding box.south east),
    ]at (current bounding box.center){};
    
\end{tikzpicture}
    \vspace*{-5mm}
    \caption*{Abbildung \thefigure: Auswertung Leitfrage 4 (Anzahl Nennungen der Sammelbegriffe $\vert$ $\Sigma: 355$, $n: 197$)}
    \label{pgfplot_leitfrage4}
\end{figure}
\addcontentsline{lof}{figure}{\numberline {\thefigure}{\ignorespaces Auswertung Leitfrage 4}}
%% -
%% - keep this block together!!!!!!!!!!!!!!
%% - --------------------------------------
%% -

\subparagraph*{Kernaussagen zu Leitfrage 4}\mbox{}

%% - table metadata
%% - --------------------------------------

\begin{table}[H]
\tablefontsize	
\caption{Kernaussagen zu Leitfrage 4}
\label{kernaussagen_leitfrage4}

%% - set width of columns
%% - ---------------------------------------

\begin{tabular}{ |p{\textwidth-1cm}| }

%% - Header row with shadowing
%% - --------------------------------------

%% - table data
%% - --------------------------------------

\hline
[\dots]Viele Mitarbeiter bewegen sich ohne sichtbar getragenen Badge - es fehlt z.B. die Ermutigung Leute ohne Badge anzusprechen.\\ 
\hline
Zeigt konkrete Alltagssituationen zu Security Risiken, um die Awareness zu steigern. Situationen, die den Mitarbeitern geläufig sind.\\
\hline
Laufende Security Awarenewess Massnahmen, z.B. Informationen zu aktuellen / neu auftretenden Security Risiken. \\ 
\hline
Es braucht Verantwortungsdefinition für jede Stufe von Mitarbeitern und die Förderung des Bewusstseins durch gezielte Massnahmen. \\ 
\hline
Es sollte klar sein, was gutes verhalten ist und was schlechtes verhalten ist.\\ 
\hline
I probably need a checklist that shows me how to escalate problems. \\ 
\hline
Having an external company do ethical-hacking to us (-> "attack us ethically") and presenting the results. \\ 
\hline
Klare, strukturierte Anweisungen was mit Firmendaten erlaubt ist, respektive was nicht. \\ 
\hline
Sensibilisierung der Führungskräfte. \\ 
\hline
Wenn wir schon so viel in Security und Security Awareness investieren, sollten wir es auch am Markt als massgeschneidertes Produkt ausgerichtet auf \companyshort-Bankeninstallationen verkaufen.. \\ 
\hline

\end{tabular}
\end{table}

\subparagraph*{Vorschläge aus der Umfrage}\mbox{}

%% - table metadata
%% - --------------------------------------

\sloppy 

\begin{table}[H]
\tablefontsize	
\centering
\caption{Vorschläge aus Interviews und Online-Umfrage zu Leitfrage 4}
\label{vorschlaege_leitfrage4}
\begin{tabular}{ |p{3.8cm}|p{2.5cm}|p{2.5cm}|p{3.8cm}|p{3.0cm}|}

\hline
\tableheaderbgcolor
\textbf{Ideen-Beschreibung} & \textbf{Kosten-\newline schätzung} & \textbf{Aufwand-\newline schätzung} & \textbf{Nutzen} & \textbf{Quadrant}\\ 

\hline
Security als Produkt: Welche Funktionalitäten bietet das \companyshort Core System out-of-the-box und wie kann das parametrisiert werden? Security Awareness als Sicherheitsberatung, Sicherheitslösung für eine \companyshort Bank. &  Kosten sind hoch &  Aufwand ist hoch  & Es wird vom Hersteller ein \companyshort-spezifisches Sicherheitspaket für das Bankenkernsystem mit flankierenen Massnahmen wie z.B. eine Security Awareness Kampagne mit Fokus auf das \companyshort System angeboten.& langfristig einplanen\\
\hline
Security Online Schulung in Academy anbieten. &  Kosten sind mittel &  Aufwand ist hoch  & Ein spezifischer Sicherheits-Kurs sowohl für \companyshort Mitarbeitende als auch für \companyshort Kunden. Fokus: das \companyshort Bankensystem . & In Ressourcenplanung aufnehmen\\
\hline
Gründung der "`Security Blackbelts"' Truppe (Teammitglieder quer durch die gesamte Organisation mit Affinität zu Security Themen). &  Kosten sind tief &  Aufwand ist tief  & Es sind Ansprechpartner für Securitybelange in den Teams vor Ort. Keine Bespitzelungsaufgaben, sondern unterstützend wirken und so den Securitygedanken in den Teams verankern  & zur Umsetzung empfohlen\\
\hline
Es braucht einen schnellen, einfachen intern offen zugänglichen Kanal für alle Mitarbeitenden zur Meldung von Beobachtungen oder störenden Patterns wie: "`warum ist diese Türe nie verschlossen"'. &  Kosten sind tief &  Aufwand ist tief  & Es wird mit der Zeit natürlich, Dinge welche auffallen zu hinterfragen und darüber zu diskutieren.   & zur Umsetzung empfohlen\\
\hline
Alle Mitarbeitenden müssen zwingend einmal im Jahr ein Web-basiertes Security-Training absolvieren. &  Kosten sind hoch &  Aufwand ist mittel  & Dies kann gegenüber den \companyshort Kunden verkauft werden, im Sinne einer stattfindenden, kontinuierlichen Mitarbeiterschulung auf Securitythemen.  & in Kostenplanung aufnehmen\\
\hline
Neueintredtende Mitarbeiter sollen ein Security Welcome-Package durchlaufen (Awareness, Web-basiertes Training, Sicherheit in der \companyshort. &  Kosten sind tief &  Aufwand ist mittel  & Neueintretenden Mitarbeitende werden gleich von Anfang an mit Securitythemen abgeholt und entsprechend sensibilisiert.  & zur Umsetzung empfohlen\\
\hline
\end{tabular}
\end{table}

\end{document}