
%% - table metadata
%% - --------------------------------------

\begin{table}[h]
\centering
\caption{Fragetypen}
\label{my-label}

%% - set width of 1st and 2nd column
%% - --------------------------------------

\begin{tabular}{ |p{2cm}|p{14cm}| }

%% - Header row with shadowing
%% - --------------------------------------

\hline
\rowcolor[HTML]{C0C0C0} 
\textbf{Fragentyp} & \textbf{Fragendefinition} \\ 
\hline

%% - table data
%% - --------------------------------------

\textbf{Typ A}     & Dieser Antworttyp wird bei Fragen verwendet, die eher mit dem Bauchgefühl, resp. Schätzung beantwortet werden sollen. Es existiert keine eigentliche harte Ja/nein Grenze, sondern es besteht ein Spielraum. Die befragte Person soll die Möglichkeit erhalten, eine gewisse Unsicherheit auf dem mit der Frage verbundenen Thema zu offenbaren. Durch die Spannweite der Antworten wird es möglich bei der Auswertung Bereiche aufzuzeigen, in denen zwar Wissen oder Bewusstsein vorhanden ist es aber noch Aufklärungs- resp. Ausbildungsbedarf gibt. \\ 
\hline
\textbf{Typ B}     & Dieser Antworttyp wird bei Fragen verwendet, die ein klares Ja oder Nein zur Antwort haben können. \\ \hline
\textbf{Typ C}     & Dieser Antworttyp ermöglicht der befragten Person ihre Antwort unstrukturiert und frei zu formulieren.  \\ \hline

\end{tabular}
\end{table}


