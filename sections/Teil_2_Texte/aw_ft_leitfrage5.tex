\documentclass[../../main.tex]{subfiles}

\begin{document}

\begin{sloppypar}
Frage: Was erwartest Du (inhaltlich, resultatmässig) von einer Security Awareness Kampagne?

\end{sloppypar}

%% -
%% - keep this block together!!!!!!!!!!!!!!
%% - --------------------------------------
%% -
\addtocounter{figure}{1}\begin{figure}[H]
    %\documentclass[../../main.tex]{subfiles}

%\begin{document}

\pgfplotstableread[col sep=comma, header=false]{
% -- <percent value>, <startpoint from above> , <label>
      45,0,  Aktuelle/lesbare Informationen zum Cybercrimegeschehen
      45,1.2,  Integration in Firmenveranstaltungen
      44,2.4,  Verständliche Empfehlungen zum Einsatz von IT für den Alltag (Firma/Heim)
      30,3.6,  Wirkungsvolle Sicherheitsschulungen und Trainings
      26,4.8,  Live Hacking Event
      24,6.0,  Praktizierbare Verhaltensanweisungen (Sicherheitsvorfall/Datenaustausch/etc.)
      13,7.2,  Informationen zu physischer Sicherheit
      13,8.4,  Vereinfachung/Verbesserung der Prozesse/Systemlandschaften
      12,9.8,  Praktisches Know-How Datenklassifizierung/-schutz/-sicherheit
      11,11.2,  Offene+transparente Kommunikation von Sicherheitsvorfällen
      4,12.4,  Verantwortungsübernahme des Managements
}\datatable

\begin{tikzpicture}

  \begin{axis}[
%    height = 5.5cm,
    xbar,
    y=-.7cm,
    enlarge y limits={abs=0.45cm},
    axis x line       = none,
    tickwidth         = 0pt,
    y axis line style = { opacity = 0 },
   yticklabels from table={\datatable}{2},
    ytick=data,
    yticklabel style={text width=9cm,align=right},
    nodes near coords,
    nodes near coords align={horizontal},
%    nodes near coords={\pgfmathprintnumber\pgfplotspointmeta\%},
    ]
    \addplot table [y=1, x=0] {\datatable};
  \end{axis}
  
% --
% -- draw surrounding box
% --
  \node[
      draw=black, very thin,
      minimum width=\textwidth,
      fit=(current bounding box.north west) (current bounding box.south east),
    ]at (current bounding box.center){};
    
\end{tikzpicture}

%\end{document}

%    \vspace*{-5mm}
    \caption*{Abbildung \thefigure: Auswertung Leitfrage 5 (Anzahl Nennungen der Sammelbegriffe $\vert$ $\Sigma: 355$, $n: 197$)}
    \label{pgfplot_leitfrage5}
\end{figure}
\addcontentsline{lof}{figure}{\numberline {\thefigure}{\ignorespaces Auswertung Leitfrage 5}}
%% -
%% - keep this block together!!!!!!!!!!!!!!
%% - --------------------------------------
%% -

\subparagraph*{Kernaussagen zu Leitfrage 5}\mbox{}
%% - table metadata
%% - --------------------------------------

\begin{table}[H]
\tablefontsize	
\caption{Kernaussagen zu Leitfrage 5}
\label{kernaussagen_leitfrage5}

%% - set width of columns
%% - ---------------------------------------

\begin{tabular}{ |p{\textwidth-1cm}| }

%% - Header row with shadowing
%% - --------------------------------------

%% - table data
%% - --------------------------------------

\hline
Es muss nach einer Security Awareness Kampagne bekannt sein, was ein sicherheitsrelevanter Vorfall ist und wie bei einem solchen zu reagieren ist.\\ 
\hline
Inhaltliche Erwartung: Den Mitarbeitenden mittels "`schocktherapeutischen"' Ansätzen die Augen öffnen und so das nötige Dringlichkeitsgefühl für Security verstärken und verankern..\\
\hline
Neue Mitarbeitende sollten standardmässig als "Willkommenspaket" eine Security Awareness Schulung durchlaufen. \\ 
\hline
Nach der Security Awareness Kampagne muss mittels "`ethical hacking"' die Resonanz der kommunizierten Massnahmen geprüft werden.  Bei Fehlverhalten keine Namensennungen; jeder weiss ja selber, wo er / sie reingefallen ist. \\ 
\hline
Als Resultat erwarte ich Sicherheitstools, die wir gebrauchen können, um Daten auszutauschen und zu schützen. Ich erwarte Richtlinien, wo welche Daten wie abgelegt werden müssen. Ich erwarte eine Liste, mit welchen von unseren Kunden und Partners ein gesicherter Mailverkehr möglich ist.\\ 
\hline
Ich erwarte klare Richtlinien für den Umgang mit Daten im jeweiligen Arbeitskontext (unterwegs, beim Kunden, zu Hause, in der Firma). \\ 
\hline
Es sollten konkrete, anonymisierte Fälle vorgestellt werden, wie die Security in unserer Firma schon verletzt wurde. \\ 
\hline
Gute Merkblätter, welche frei von IT-Lingo und Security-Slang sind.Etwas, was man gerne liest. Kurz, prägnant, eindeutig. Bitte in den verschiedenen Muttersprachen abgeben! \\ 
\hline
Die Kampagne muss auch Aspekte der physischen Sicherheit umfassen, also Brandschutz, Evakuation, medizinische Soforthilfe. Die Kantonspolizei sollte eingeladen werden und etwas zum Thema Einbruchsschutz vortragen. \\ 
\hline

\end{tabular}
\end{table}

\subparagraph*{Vorschläge aus der Umfrage}\mbox{}

%% - table metadata
%% - --------------------------------------

\sloppy 

\begin{table}[H]
\tablefontsize	
\centering
\caption{Vorschläge aus Interviews und Online-Umfrage zu Leitfrage 5}
\label{vorschläge_leitfrage5}
\begin{tabular}{ |p{3.8cm}|p{2.5cm}|p{2.5cm}|p{3.8cm}|p{3.0cm}|}

\hline
\tableheaderbgcolor
\textbf{Ideen-Beschreibung} & \textbf{Kosten-\newline schätzung} & \textbf{Aufwand-\newline schätzung} & \textbf{Nutzen} & \textbf{Quadrant}\\ 

\hline
Selbstgedrehte Videos zum Thema Security Awareness auf dem Intranet verfügbar machen. &  Kosten sind tief &  Aufwand ist mittel  & Die Security-Protagonisten werden im Video gezeigt, CISO, CIO, CEO, SecOff). Somit werden ihre Gesichter bekannt. Das im Viodeo angesprochene Thema kann ganz auf die \companyshort zugeschnitten sein.& zur Umsetzung empfohlen\\
\hline
Penetration Test von einer externen Beratungsfirma durchführen lassen. Abgedeckte Gebiete: Physische Security, Social Engineering, Angriff auf Webseite, Remote-Login Infrastruktur, etc. Die Ergebnisse ungeschönt präsentieren. &  Kosten sind mittel &  Aufwand ist mittel  & Aussenwirkung: Die Kunden der \companyshort stellen in ihren Security Audits immer öfters Fragen zu diesem Thema, es wird erwartet, dass die Firma hier etwas unternimmt. Dies lässt sich dann entsprechend belegen. Innenwirkung: Für die Mitarbeitenden wird ersichtlich, dass es das Unternehmen mit der Security Awareness ernst meint und sich auch selber auf den Prüfstand stellen lässt. & zur Umsetzung empfohlen\\

\hline
Durchführung Live Hacking Event: Einbruch in ein beliebiges Auto auf dem Parkplatz mittels eines USB-Jammers.  &  Kosten sind tief & Aufwand ist tief & Wachrüttelung von den Mitarbeitenden, dass sie auch in ihren privaten Umgebung (sogar im Auto!) jederzeit angreifbar sind. Erhöhung der Glaubwürdigkeit der Sicherheitskampagne. & zur Umsetzung empfohlen\\

\hline
Durchführung einer Phishing-Aktion mit externer Sicherheitsfirma.  &  Kosten sind mittel &  Aufwand ist mittel  & Phishing als eines der Hauptanliegen der Mitarbeitenden (siehe begleitende Umfrage zu dieser Studie) wird direkt thematisiert. Steigerung der Awareness gegenüber Phishing Attacken. & zur Umsetzung empfohlen\\

\hline
Vortrag der Kantonspolizei Zürich zum Thema physische Sicherheit an einer grösseren Mitarbeiterveranstaltung.  &  Kosten sind tief &  Aufwand ist tief  & Für die Mitarbeiteden ist ersichtlich, dass sich Security Awareness nicht nur auf das IT-Umfeld des Unternehmens oder den heimischen PC beschränkt, sondern dass es noch viele andere sehr wichtige Themen gibt, welche auch zuR Security Awareness gehören.  & zur Umsetzung empfohlen\\
\hline

\end{tabular}
\end{table}

\end{document}