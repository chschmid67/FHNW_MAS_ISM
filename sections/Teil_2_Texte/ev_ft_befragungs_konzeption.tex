\documentclass[../../main.tex]{subfiles}

\begin{document}

\begin{sloppypar}
Von einem konzeptionellen Standpunkt aus wird die Online-Umfrage nicht vorangekündigt, währenddem für die Leitfadeninterviews die zu befragenden Probanden vorher elektronisch um die Teilnahme an den Interviews angefragt werden. 

Die Anfrage dient dazu, die Bereitschaft zur Beantwortung abzufragen und, falls diese gegeben ist, die sorgfältige und rechtzeitige Beantwortung der Interviewfragen sicherzustellen. (vgl. \citeauthor{berekoven_marktforschung:_2009} \citeyear{berekoven_marktforschung:_2009}, S. 111 f.).

Der Interview- und der Online-Umfrage liegen entsprechend formulierte Begleitschreiben bei, welche die Empfänger über den Zweck und die Wichtigkeit der Umfrage in Kenntnis setzen. Für die Formulierung dieser Begleitschreiben wurden die Hinweise von \citeauthor{berekoven_marktforschung:_2009} bezüglich der zu beachtenden Aspekte beachtet (Zweck und Zielsetzung der Anfrage, Argumentation für das Schreiben, Zeitliche Aspekte der Umfrage, Zusicherung der Vertraulichkeit bzw. Anonymität, Dank im Voraus) (vgl. \citeauthor{berekoven_marktforschung:_2009} \citeyear{berekoven_marktforschung:_2009}, S. 112 f.; \citeauthor{kirchhoff_fragebogen:_2010} \citeyear{kirchhoff_fragebogen:_2010}, S. 29).

Gemäss \citeauthor{berekoven_marktforschung:_2009} (\citeyear{berekoven_marktforschung:_2009}, S. 113 f.) sollen bei Online-Umfragen mehrere Nachfass-Aktionen in verschiedenen Intervallen durchgeführt werden. Für diese Arbeit wird jedoch aufgrund der knappen Zeitverhältnisse etwa eine Woche vor Ende der Laufzeit nur eine Nachfass-Aktion ausgeführt.

Für den genauen Wortlaut des Begleitschreibens der Online-Umfrage siehe Anhang A, Seite \pageref{begleitschreiben_online_umfrage}; für den genauen Wortlaut der Begleitschreiben der Leitfaden-Interviews siehe Anhang A, Seite \pageref{begleitschreiben_interview}. Die Formulierung des Schreibens für die Nachfass-Aktion ist im Anhang A, Seite \pageref{nachfassung_online_umfrage} ersichtlich.

\end{sloppypar}

\paragraph*{Online-Umfrage}\mbox{}

\begin{sloppypar}
Um eine standardisierte Erfassung von Antworten zu ermöglichen, wird die Online-Umfrage mit einem auf die zentralen Fragestellungen ausgerichteten Fragenkatalog durchgeführt. Jede dieser Fragen ist einem oder mehreren Themenkomplexen zugeordnet. Durch die anschliessende Auswertung soll der Abdeckungsgrad für den jeweiligen Themenkomplex bestimmt und dargestellt werden. Die Themenkomplexe werden nachstehend im Kapitel \ref{beschreibung_themenbereiche} definiert.

Der Fragenkatalog der im Rahmen dieser Arbeit entwickelt wird, wird für die Durchführung der Umfrage auf einer geeigneten Online-Plattform für die Anspruchsgruppen zur Verfügung gestellt.
\end{sloppypar}

\paragraph*{Leitfadeninterviews}\mbox{}

\begin{sloppypar}
Als Ergänzung zu der Online-Umfrage finden zusätzliche Leitfadeninterviews mit ausgewählten Probanden (Experten\footnote{Als Experte gilt jemand, der auf einem begrenzten Gebiet über ein klares und abrufbares Wissen verfügt (\citeauthor{mayer_interview_2013} \citeyear{mayer_interview_2013}, Seite 41)}) aus den Anspruchsgruppen statt. Diese Interviews sind als Ergänzung zu der Online-Umfrage zu verstehen. Als Leitlinie werden diejenigen Fragen aus der Online-Umfrage verwendet, zu denen jeweils eine persönliche Meinung abgefragt wird, Beispielsweise "`Was bedeutet für Dich Security Awareness?"'. Dadurch soll sichergestellt werden, dass keine wesentlichen Aspekte der Forschungsfrage im Interview übersehen werden (vgl. \citeauthor{mayer_interview_2013} \citeyear{mayer_interview_2013}, S. 41).

Da die Interviews aus Zeitgründen nicht transkribiert werden, werden sie akustisch aufgenommen und nach dem Interview interpretativ ausgewertet. \citeauthor{mayer_interview_2013} (\citeyear{mayer_interview_2013}) sagt dazu:

\begin{quote}
"`Eine zentrale Aufgabe der Interpretation liegt darin, dass der Interpret nicht nur das, was wörtlich gesagt wurde decodiert, sondern das Gesagte auch interpretiert. Zu beachten ist jedoch, dass es keine eindeutige Interpretation [\textellipsis] geben kann, jedes Interview steht einer Anzahl konkurrierender Deutungen offen. Wichtig bei der Interpretation sind neben einer \textbf{\textit{genauen}} [Hervorhebung C.S.] Aufzeichnung des Gesprächtextes, die umfassende Betrachtung des Befragten und dessen Äusserungen"' (\citeauthor{mayer_interview_2013} \citeyear{mayer_interview_2013}, S. 25).
\end{quote}

Um die Repräsentativität der Interviews sicherzustellen, werden von jeder der selektierten Anspruchsgruppe mindestens zwei Interviews durchgeführt und wie beschrieben ausgewertet. Die Resultate der interpretativen Interviewauswertung werden für eine Gesamtaussage  mit den Resultaten Online-Umfrage zusammen kombiniert.

\end{sloppypar}

\end{document}
