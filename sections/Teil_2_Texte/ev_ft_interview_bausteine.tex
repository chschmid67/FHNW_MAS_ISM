\documentclass[../../main.tex]{subfiles}

\begin{document}

\paragraph*{Grundstruktur}\mbox{}

\begin{sloppypar}
Ein Interview besteht aus mehreren Frageblöcken. Pro Frageblock gibt es eine vordefinierte Stichworteliste, eine Liste für zusätzliche, dem Interview entnommenen Stichworten und einen "`Ideenspeicher"'. Diese Elemente werden nachstehend beschrieben.

Die Interviews werden von ihrem Ablauf her als halbstrukturierte, "`freie Interviews"' durchgeführt. Der halbstrukturierte Teil besteht darin, die Gesprächsentwicklung an den (ebenfalls in der Online-Umfrage vorkommenden) offen gehaltenen Kernfragen zu orientieren:

\begin{itemize}
  \item Was bedeutet Security Awareness für Dich persönlich?
  \item Was würdest Du bei Dir zu Hause bezüglich Security als die grösste Risikoquelle identifizieren? 
  \item Was würdest Du im Unternehmen bezüglich Security als die grösste Risikoquelle identifizieren?
  \item Was fehlt / was braucht es Deiner Meinung nach bezüglich Security Awareness?
  \item Was erwartest Du von einer Security Awareness Kampagne?
\end{itemize}

Der freie Teil des Interviews soll es ermöglichen, dass sich die Themenwahl während des Gesprächs entwickeln kann und so weitere, aus der persönlichen Wahrnehmung des Probanden stammende, wichtige Aspekte von Security Awareness zur Sprache kommen können. 
\end{sloppypar}

\subparagraph*{Frageblock}\mbox{}

\begin{sloppypar}
Die im Frageblock gestellte Kernfrage entspricht derjenigen aus der Online-Umfrage.
\end{sloppypar}

\paragraph*{Vordefinierte Stichworteliste}\mbox{}

\begin{sloppypar}
Für eine Beschreibung der Stichwortliste  siehe Anhang A, Tabelle \ref{Stichwortliste für Interviewfragen}, Seite \pageref{Stichwortliste für Interviewfragen}.

Bei der Auswertung wird pro Kernfrage die Antwort anhand einer vordefinierten Stichworteliste qualifiziert\footnote{Qualifikationskriterium: das Stichwort wurde (sinngemäss) genannt}.
\end{sloppypar}

\subparagraph*{Zusätzliche Stichworteliste}\mbox{}

\begin{sloppypar}
Um genannte Stichwörter abzudecken die in der vordefinierten Stichworteliste nicht aufgeführt sind, werden diese bei der Auswertung der Frageblöcke extrahiert und in einer zusätzlichen, offenen Stichworteliste dokumentiert.
\end{sloppypar}

\subparagraph*{Ideenspeicher}\mbox{}

\begin{sloppypar}
Da bei einem freien Interview immer die Möglichkeit besteht, dass vom Probanden sehr konkrete Vorschläge oder Ideen formuliert werden, können diese im Ideenspeicher als zusammengefasster Freitext dokumentiert werden.
\end{sloppypar}

\paragraph*{Akustische Aufzeichnung}\mbox{}

\begin{sloppypar}
Die Interviews werden (mit Einverständnis des Probanden) akustisch aufgezeichnet. Die Aufzeichnung wird für die spätere Auswertung nicht transkribiert, sondern sinngemäss vom Interviewer zusammengefasst und elektronisch festgehalten. An persönlichen Daten werden analog der Online-Umfrage die Rolle, Anzahl Jahre in der Firma, Anzahl Jahre in der Branche, höchster Schulabschluss und die Herkunftsregion erhoben. Die akustischen Daten werden nach der Auswertung verschlüsselt und archiviert.
\end{sloppypar}

\end{document}