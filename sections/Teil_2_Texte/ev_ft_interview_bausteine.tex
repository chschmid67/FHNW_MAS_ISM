\documentclass[../../main.tex]{subfiles}

\begin{document}

\paragraph*{Grundstruktur}\mbox{}

\begin{sloppypar}
Ein Experteninterview besteht aus mehreren Frageblöcken. Pro Frageblock gibt es eine vordefinierte Stichworteliste, eine Liste für zusätzliche, dem Interview entnommenen Stichworten und einen "`Ideenspeicher"'. Diese Elemente werden nachstehend beschrieben.

Die Interviews werden von ihrem Ablauf her als halbstrukturierte, "`freie Interviews"' durchgeführt. Der halbstrukturierte Teil besteht darin, die Gesprächsentwicklung an den (ebenfalls in der Online-Umfrage vorkommenden) offen gehaltenen Kernfragen zu orientieren:

\begin{itemize}
  \item Was bedeutet der Begriff "`Security Awareness"' für Dich persönlich?
  \item Was würdest Du bei Dir zu Hause bezüglich Security als die grösste Risikoquelle identifizieren? 
  \item Was würdest Du im Unternehmen bezüglich Security als die grösste Risikoquelle identifizieren?
  \item Was fehlt / was braucht es Deiner Meinung nach bezüglich Security Awareness?
  \item Was erwartest Du (inhaltlich, resultatmässig) von einer Security Awareness Kampagne?
\end{itemize}

Der freie Teil des Interviews soll es ermöglichen, dass sich die Themenwahl während des Gesprächs entwickeln kann und so weitere, aus der persönlichen Wahrnehmung des Probanden stammende, wichtige Aspekte von Security Awareness zur Sprache kommen können. Anlass zu diesem Vorgehen gab der Hinweis von \textsc{Friebertshäuser}:

\begin{quote}
"`Eine Gefahr eines Leitfadens liegt darin, dass das Interview zu einem Frage- und Antwort-Dialog verkürzt wird [\dots] ohne dass dem Befragten Raum für seine (möglicherweise auch zusätzlichen) Themen und die Entfaltung seiner Relevanzstrukturen gelassen wird."' (\textsc{Friebertshäuser} 1997, S. 376, zitiert nach \cite{mayer_interview_2013}, S. 44).
\end{quote}

\end{sloppypar}

\subparagraph*{Frageblock}\mbox{}

\begin{sloppypar}
Die im Frageblock gestellte Kernfrage entspricht derjenigen aus der Online-Umfrage.
\end{sloppypar}

\paragraph*{Vordefinierte Stichworteliste}\mbox{}

\begin{sloppypar}
Für eine Beschreibung der Stichwortliste  siehe Anhang A, Tabelle  \ref{Stichwortliste Leitfragen Experteninterview}, Seite \pageref{Stichwortliste Leitfragen Experteninterview}.

Bei der Auswertung wird pro Kernfrage die Antwort anhand einer vordefinierten Stichworteliste qualifiziert\footnote{Qualifikationskriterium: das Stichwort wurde (sinngemäss) genannt}. Dieses Verfahren wurde in Anlehnung an die Methode der "`Codierung einer offenen Frage"' ausgearbeitet (vgl. \citeauthor{mayer_interview_2013} \citeyear{mayer_interview_2013}, S. 108).
\end{sloppypar}

\subparagraph*{Zusätzliche Stichworteliste}\mbox{}

\begin{sloppypar}
Um genannte Stichwörter abzudecken die in der vordefinierten Stichworteliste nicht aufgeführt sind, werden diese bei der Auswertung der Antworten extrahiert und in einer zusätzlichen, offenen Stichworteliste dokumentiert.
\end{sloppypar}

\subparagraph*{Ideenspeicher}\mbox{}

\begin{sloppypar}
Da bei einem freien Interview immer die Möglichkeit besteht, dass vom Probanden sehr konkrete Vorschläge oder Ideen formuliert werden, können diese im Ideenspeicher als zusammengefasster Freitext dokumentiert werden. Von weiterer Bedeutung ist dabei eine Einschätzung im Hinblick auf die Kosten und den Zeitaufwand für die Realisierung der genannten Idee. Diese Einschätzung wird bei der Auswertung vorgenommen und stellt lediglich einen Richtwert dar. 
\end{sloppypar}

\paragraph*{Akustische Aufzeichnung}\mbox{}

\begin{sloppypar}
Die Interviews werden (mit Einverständnis des Probanden) akustisch aufgezeichnet. Die Aufzeichnung wird für die spätere Auswertung nicht transkribiert, sondern sinngemäss vom Interviewer zusammengefasst (paraphrasiert) und elektronisch nachdokumentiert. An persönlichen Daten werden analog der Online-Umfrage die Rolle, Anzahl Jahre in der Firma, Anzahl Jahre in der Branche, höchster Schulabschluss und die Herkunftsregion erhoben.

Die Aufbewahrung der akustischen Daten soll primär eine zeitversetzte Auswertung der Interviews ermöglichen. Sollte durch die Nachbearbeitung eventuell Unschärfen bei der Stichwortauswahl oder Eintrag in den Ideenspeicher entstanden sein, kann durch die Archivierung zu einem späteren Zeitpunkt auf die Rohdaten des Interviews zurückgegriffen werden, beispielsweise während der Definition einer Security Awareness Kampagne. Zudem stellt die Archivierung die Authentizität der protokollierten Interview-Aussagen sicher.

Die akustischen Aufzeichnungen werden aus Datenschutzgründen in einem nur dem Security-Office zugänglichen Bereich gespeichert. Die Aufzeichnungen können vom CISO jederzeit gelöscht werden.
\end{sloppypar}

\end{document}