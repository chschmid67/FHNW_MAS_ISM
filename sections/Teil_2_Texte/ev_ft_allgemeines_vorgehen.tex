\documentclass[../../main.tex]{subfiles}

\begin{document}

\begin{sloppypar}
Als Grundgedanke für das gewählte empirische Vorgehen dient die von \citeauthor{mayer_interview_2013} (\citeyear{mayer_interview_2013}) beschriebene Zentrale Aufgabe der empirischen Forschung, nach welcher Theorien über die Realität aufgestellt werden und diese mit dem Ziel überprüft werden, ihre Zusammenhänge systematisch zu erfassen (vgl. \cite{mayer_interview_2013}, S. 18). Die gewählte Vorgehensweise ist induktiv, d.h. der Schluss wird vom Speziellen (Einzelbeobachtung) auf das Generelle (Allgemeines) gezogen (vgl. \cite{mayer_interview_2013}, S. 19 und S. 24).

Um dies umzusetzen wird für die Befragung der definitiven Anspruchsgruppen eine Doppelstrategie bestehend aus einer Kombination von \textbf{Online-Umfrage} (quantitativer Ansatz) und \textbf{Leitfadeninterviews} (qualitativer Ansatz) angewandt. Diese Doppelstrategie soll vor allem die gegenseitigen Ergänzungen der beiden Forschungsansätze sinnvoll kombinieren. Laut (\cite{mayer_interview_2013}) ergänzen und vertiefen qualitative Ergebnisse auch die quantitativen Daten und umgekehrt (vgl. \cite{mayer_interview_2013}, S. 27).

Durch die Anwendung der Leitfadeninterviews soll der Zugang zu Ideen und Vorschlägen erschlossen werden, welche in einer Online-Umfrage entweder wegen zeitlichen Gründen oder aus Bequemlichkeit von den Probanden nicht erfasst werden. Die Gespräche ermöglichen es dem Interviewer, agil auf ein während des Gesprächs aufkeimendes Anspruchsgruppenbedürfnis, eine in einem Nebensatz erwähnte Idee oder These einzugehen zu können. Somit kann die Lücke, welche durch die alleinige Anwendung eines standardisierten Fragenkataloges entsteht, geschlossen werden (vgl. \cite{berekoven_marktforschung:_2009}, S. 117).

\end{sloppypar}

\paragraph*{Vorgehensbegründung}\mbox{}

\begin{sloppypar}
Wie in \citeauthor{berekoven_marktforschung:_2009} (\citeyear{berekoven_marktforschung:_2009}, S. 92 ff.) beschrieben, ist die standardisierte Befragung jedoch ein probates Mittel, um mittels Interviews und Online-Befragungen (unter Berücksichtigung der ebenfalls in \citeauthor{berekoven_marktforschung:_2009} aufgezeigten Schwachstellen\footnote{\citeauthor{berekoven_marktforschung:_2009} sagen zu diesem Thema, dass mitunter die häufigste Schwachstelle bei Online-Befragungen die Allgemeinverständlichkeit der Fragen darstellt. Hinzu kommt die Gefahr der "`Überfrachtung"' einzelner Fragen mit der einhergehenden Überforderung des Probanden. Der innere Drang nach einer vollständigen Beantwortung aller Fragen führt dann leicht zu willkürlichen Angaben. Die Qualität der Interviews kann hingegen durch eine zu gross gewählte zeitliche Dauer der Befragung, gepaart mit einem problematischen Befragungsort (z.B. Interview in einem Grossraumbüro, dadurch zwangsläufige Anwesenheit und Mithörerschaft Dritter) zu einem schlechten Ergebnis des Interviews führen.} der jeweiligen Befragungsart) zu repräsentativen Schlussfolgerungen aus den Aussagen entlang der Leitfragen zu gelangen. \citeauthor{berekoven_marktforschung:_2009} (\citeyear{berekoven_marktforschung:_2009}) meinen dazu:

\begin{quote}
"`[\dots] dass persönliche Interviews Vorzüge haben, die andere Erhebungsmethoden nicht aufweisen, weil eben im persönlichen Gegenüber verbale und non-verbale Äu{\ss}erungen in vielen Fällen ergiebiger [\dots] sind. Komplizierte Sachverhalte lassen sich vielfach nur auf diesem Wege eruieren"' (\citeauthor{berekoven_marktforschung:_2009} \citeyear{berekoven_marktforschung:_2009}, S. 100).
\end{quote}
\end{sloppypar}

\end{document}
