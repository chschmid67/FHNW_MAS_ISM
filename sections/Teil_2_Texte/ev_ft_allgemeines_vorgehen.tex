\documentclass[../../main.tex]{subfiles}

\begin{document}

\begin{sloppypar}
Für das empirische Vorgehen wird eine Doppelstrategie bestehend aus einer Kombination von \textbf{Online-Umfrage} und \textbf{Interviews} angewandt. Diese Doppelstrategie soll vor allem über eine systematische Datenerfassung den Grunddatenbestand für statistische Auswertungen bereitstellen.

Zusätzlich soll die Strategie durch die Anwendung der möglichst frei gehaltenen Interviews den Zugang zu Ideen und Vorschlägen erschliessen, die durch eine Online-Umfrage eher schwierig zu erfassen sind. Die Gespräche ermöglichen es dem Interviewer, agil auf ein während des Gesprächs aufkeimendes Anspruchsgruppenbedürfnis, eine Idee oder eine These einzugehen zu können. Somit kann die Lücke, welche durch die alleinige Anwendung eines standardisierten Fragenkataloges entsteht, abgedeckt werden.
\end{sloppypar}

\paragraph*{Online-Umfrage}\mbox{}

\begin{sloppypar}
Um eine standardisierte Erfassung von Antworten zu ermöglichen, wird die Online-Umfrage mit einem auf die zentralen Fragestellungen ausgerichteten Fragenkatalog durchgeführt. Jede Frage ist einem oder mehreren Themenbereichen zugeordnet. Durch die anschliessende Auswertung soll der Abdeckungsgrad für den jeweiligen Themenbereich bestimmt und dargestellt werden. Die Themenbereiche werden im Kapitel \ref{beschreibung_themenbereiche} definiert.

Der Fragenkatalog der im Rahmen dieser Arbeit entwickelt wird, wird für die Durchführung der Umfrage auf einer geeigneten Online-Plattform für die Anspruchsgruppen zur Verfügung gestellt.
\end{sloppypar}

\paragraph*{Interviews / Diskussionen}\mbox{}

\begin{sloppypar}
Als Ergänzung zu der Online-Umfrage finden zusätzliche Interviews mit ausgewählten Probanden aus den Anspruchsgruppen statt. Diese Interviews sind als Ergänzung zu der Online-Umfrage zu verstehen. Als Leitlinie werden diejenigen Fragen aus der Online-Umfrage verwendet, zu denen jeweils eine persönliche Meinung abgefragt wird, Beispielsweise "`Was bedeutet für Dich Security Awareness?"'.

Um die Repräsentativität der Interviews sicherzustellen, werden von jeder der selektierten Anspruchsgruppe mindestens Drei Interviews durchgeführt. Die Interviews werden zusammen mit der Online-Umfrage ausgewertet.

\end{sloppypar}

\end{document}
