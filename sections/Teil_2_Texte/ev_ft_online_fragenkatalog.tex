\documentclass[../../main.tex]{subfiles}


\begin{document}

\paragraph*{Fragebereich allgemeine Angaben}\mbox{}

\begin{sloppypar}
Entegegen der Empfehlungen von \citeauthor{mayer_interview_2013} (\citeyear{mayer_interview_2013}) werden die demografischen Merkmale nicht am Schluss der Umfrage erhoben, sondern bewusst zu Beginn. Diese Daten sollen bei der Auswertung Korrelationen zwischen den Anspruchsgruppen ermöglichen, sie sollen daher möglichst vollständig und korrekt sein. \citeauthor{mayer_interview_2013} (\citeyear{mayer_interview_2013}) sagt dazu:

\begin{quote}
"`Es ist vorteilhaft, Fragen nach demopgraphischen Merkmalen am Ende des Fragebogens zu platzieren, da hier manchmal Ermüdungserscheinungen auftreten können, d.h. das Interesse an den Fragen [\dots] nachlässt."' (\cite{mayer_interview_2013}, S. 96).
\end{quote}

Der Autor teilt diese Meinung nicht. Da den allgemeinen Fragen bei späteren Auswertungen ein relativ hohes Gewicht zukommt, sollten sie auch akkurat beantwortet werden. Deshalb werden sie an den Beginn des Fragebogens patziert. Mittels dieser Fragen soll ein Profil inklusive der kulturellen Herkunft zu den hinterlegten Antworten definiert werden können. Beispielsweise soll damit bei einer späteren Auswertung ersichtlich werden, welche Rollen bereits ein Security Awareness Verständnis haben und welche eher nicht, oder ob sich die Anzahl Jahre in der Branche oder dem Betrieb positiv oder negativ auf das Sicherheitsverhalten auswirkt.

Die Frage nach der Herkunft zielt darauf ab, den kulturellen Hintergrund erfassen zu können. Die Antwort darauf hat bei der Auswertung keinen beeinflussenden Charakter; sie wird bei der Umsetzung der späteren Massnahmen jedoch wichtig für die Art und Weise wie diese interkulturell transportiert werden müssen.\footnote {Dies betrifft Aspekte wie gewählte Kommunikationsart, benutzte Kanäle, Bilder, eingesetzte Farben, etc.}

Die ID's der Fragen zu diesem Bereich werden aus dem Präfix "`P"' für "`Persönlich"' und der fortlaufenden Fragenummer gebildet.

Für eine Beschreibung der Fragen siehe Anhang A, Tabelle \ref{Allgemeine Angaben}, Seite \pageref{Allgemeine Angaben}.
\end{sloppypar}

\paragraph*{Fragebereich eigene Meinung zu Security Awareness}\mbox{}

\begin{sloppypar}
In diesem Umfragebereich soll die befragte Person den Begriff "`Security Awareness"' sowohl aus ihrer persönlichen Perspektive wie auch aus ihrer eigenen Erfahrung heraus beschreiben. Dies ist eine offen formulierter Frage, d.h. die Antwort kann beliebig lang und unstrukturiert sein.

Die Frage-ID zu diesem Bereich wird aus dem Präfix "`F"' für "`Freie Antwort"' und der fortlaufenden Fragenummer gebildet.

Für eine Beschreibung der Fragen siehe Anhang A, Tabelle \ref{Eigene Meinung zu Security Awareness}, Seite \pageref{Eigene Meinung zu Security Awareness}.
\end{sloppypar}


\paragraph*{Fragebereich Security Awareness im privaten Umfeld}\mbox{}

\begin{sloppypar}
Der dritte Umfragebereich dient dazu, das Sicherheitsverhalten der befragten Person in ihrer heimischen / familiären Umgebung sowie in ihrem Verwandten- / Freundeskreis zu erfragen. Die Absicht ist, sicherndes Verhalten zwischen sozialem Umfeld und Unternehmensumgebung zu transportieren.

Die ID's der Fragen zu diesem Bereich werden aus dem Präfix "`S"' für "`Soziales Umfeld"' und der fortlaufenden Fragenummer gebildet.

Für eine Beschreibung der Fragen siehe Anhang A, Tabelle \ref{Security Awareness im privaten Umfeld}, Seite \pageref{Security Awareness im privaten Umfeld}.
\end{sloppypar}


\paragraph*{Fragebereich Security Awareness im Unternehmen}\mbox{}

\begin{sloppypar}
In diesem Umfragebereich geht es darum, die aktuelle Wahrnehmung der sichernden Elemente in der Firma (Sicherheitsbeauftragte Personen, Verfahren bei Sicherheitsvorfällen oder den Stand des Bewusstseins im Umgang mit Firmendaten) abzufragen.

Die ID's der Fragen zu diesem Bereich werden aus dem Präfix "`U"' für "`Unternehmen"' und der fortlaufenden Fragenummer gebildet.

Für eine Beschreibung der Fragen siehe Anhang A, Tabelle \ref{Security Awareness im Unternehmen}, Seite \pageref{Security Awareness im Unternehmen}.
\end{sloppypar}



\paragraph*{Fragebereich Kenntnisse zu Sicherheitsthemen}\mbox{}

\begin{sloppypar}
Dieser Umfragebereich soll der befragten Person die Möglichkeit bieten, zu aktuellen Sicherheits-themen / Bedrohungsformen ein eventuelles Informationsbedürfnis anzuzeigen. Die Fragen sollen so formuliert sein, dass sie nicht als Prüfungsblock für Wissen / nicht Wissen wahrgenommen werden.

Die ID's der Fragen zu diesem Bereich werden aus dem Präfix "`V"' für "`Verständnis"' und der fortlaufenden Fragenummer gebildet (vgl. \citeauthor{kirchhoff_fragebogen:_2010} \citeyear{kirchhoff_fragebogen:_2010}, S. 37)

Für eine Beschreibung der Fragen siehe Anhang A, Tabelle \ref{Kenntnisse zu Sicherheitsthemen}, Seite \pageref{Kenntnisse zu Sicherheitsthemen}.
\end{sloppypar}


\paragraph*{Fragebereich Erwartungshaltung Security Awareness}\mbox{}

\begin{sloppypar}
Dieser letzte Umfragebereich ist als Sammelgefäss zu verstehen, in welches die befragten Personen ihre Anregungen, Ideen, Wünsche oder durch die Tätigkeit in anderen Unternehmen bereits erlebte Security Awareness Massnahmen anregen können. Es handelt sich um offen formulierte Fragen, d.h. die jeweilige Antwort kann beliebig lang und unstrukturiert sein.

Die ID's der Fragen zu diesem Bereich werden aus dem Präfix "`F"' für "`Freie Antwort"' und der fortlaufenden Fragenummer gebildet.

Für eine Beschreibung der Fragen siehe Anhang A, Tabelle \ref{Erwartungshaltung Security Awareness}, Seite \pageref{Erwartungshaltung Security Awareness}.
\end{sloppypar}


\end{document}
