\documentclass[../../main.tex]{subfiles}

\begin{document}
\paragraph*{Struktur der Online-Umfrage}\mbox{}

\begin{sloppypar}
Die Online-Umfrage wird in deutscher und in englischer Sprache zur Verfügung gestellt. Die Grundstruktur setzt sich aus mehreren Bereichen zusammen; jeder dieser Bereiche beinhaltet einen bereichsspezifischen Fragesatz. Die Bereiche bündeln die Fragen zu Kontexten wie Unternehmen, Privat- und Sozialleben, Datenschutz, etc. und sollen es den Probanden ermöglichen, den Fragesatz aus einem jeweiligen Bezug heraus zu beantworten.

Grundsätzlich wird zwischen \textbf{offenen} und \textbf{geschlossenen} Fragen unterschieden. Offene Fragen geben für ihre Beantwortung keine Antwortstruktur vor; geschlossene Fragen haben eine vordefinierte Antwortstruktur (Bereichsangabe, Ja / Nein Entscheidung, etc.)

Die Fragen lassen sich einem (oder in einigen Fällen auch weiteren) der zuvor definierten Themenbereichen zuordnen. Die offen formulierten Fragen stellen hierbei einen Sonderfall dar. Da sich die Antworten auf diese Fragen naturgemäss nicht direkt Auswerten lassen, ist ein Zwischenschritt notwendig, welcher diese Antworten zuerst analysiert. Aus diesem Grund werden die Antworten auf die offenen Fragen zusammen mit den Interviews ausgewertet.\footnotemark
 
Hinter jeder Frage der Umfrage-Bereiche steht eine Absicht, d.h. eine Erklärung dafür, was mit der Frage erreicht / im tieferen Sinne abgefragt werden soll. Diese Absichtserklärung ist bei der jeweils entsprechenden Frage hinterlegt. 

Zwecks besserer Umsetzbarkeit und Lesbarkeit der Umfrage werden nachfolgend die Antworttypen definiert und diese dann jeweils einer Frage zugeordnet.
\end{sloppypar}

\footnotetext {Für eine genauere Beschreibung des Vorgehens siehe Kapitel \ref{Auswertungsmethodik}, Seite \pageref{Auswertungsmethodik}.}

\paragraph*{Definition der Antworttypen}\mbox{}

\begin{sloppypar}
Die Antworttypen dienen der Standardisierung des Fragebogens und sollen die spätere Auswertung erleichtern. Die nachfolgende Tabelle definiert und beschreibt die möglichen Antworttypen, ihre jeweilige Kardinalität, womit eine Antwort ausgewählt werden kann sowie deren Anwendung bei offenen, respektive geschlossenen Fragen. 
\end{sloppypar}

%% - table metadata
%% - --------------------------------------

\sloppy 

\begin{table}[H]
\tablefontsize	
\centering
\caption{Definition der Antworttypen}
\label{Definition der Antworttypen}
\begin{tabular}{ |p{1.0cm}|p{5.0cm}|p{4cm}|p{2.5cm}|p{2.5cm}|}

\hline
\tableheaderbgcolor
\textbf{Typ} & \textbf{Beschreibung} & \textbf{Kardinalität} & \textbf{Antwortauswahl} & \textbf{Anwendung}\\ 

\hline
\textbf{Typ A} &  Einschätzungsantwort mit unscharfer Bereichsangabe. Dieser Antworttyp wird bei Fragen verwendet, die eine gewisse Unsicherheit beim Probanden auslösen können, sei es durch die Fragenthematik selber oder durch die Art und Weise wie die Frage gestellt wird. & "`Trifft überhaupt nicht zu"' \newline bis \newline "`Trifft voll zu"' & Analogskala  & Geschlossene Frage\\

\hline
\textbf{Typ B} & Dieser Antworttyp wird bei Fragen verwendet, die mit einer klaren Auswahl (Beispielsweise Ja oder Nein) beantwortet werden können. & Ja \newline Nein \newline unbekannt oder sagt mir nichts & Auswahlmenü & Geschlossene Frage\\

\hline
\textbf{Typ C} &  Dieser Antworttyp ermöglicht der befragten Person ihre Antwort frei zu formulieren. &  Unstrukturierte Textantwort & Direkte Eingabe & Offene Frage\\
\hline

\end{tabular}
\end{table}

%% - read questionnaire
%% - --------------------------------------

\subfile{sections/Teil_2_Texte/ev_ft_online_fragenkatalog}


\end{document}







