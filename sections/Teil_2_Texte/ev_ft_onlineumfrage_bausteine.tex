\documentclass[../../main.tex]{subfiles}

\begin{document}

\begin{sloppypar}
Die Online-Umfrage wird in deutscher und in englischer Sprache zur Verfügung gestellt. Grundsätzlich wird zwischen \textbf{offenen} und \textbf{geschlossenen} Fragen unterschieden (vgl. \cite{berekoven_marktforschung:_2009}, S. 95). Offene Fragen geben für ihre Beantwortung keine Antwortstruktur vor; geschlossene Fragen haben eine vordefinierte Antwortstruktur (Bereichsangabe, Ja / Nein Entscheidung, etc.)

Die Grundstruktur der Online-Umfrage setzt sich aus mehreren Fragebereichen zusammen; jeder dieser Bereiche beinhaltet einen spezifischen Fragesatz (vgl. \cite{mayer_interview_2013}, S. 96). Die Fragebereiche stellen die Benutzersicht auf die in Kapitel \ref{beschreibung_themenbereiche} definierten Themenkomplexe dar; sie bündeln die Fragen zu Nutzer-Kontexten wie Unternehmen, Privat- und Sozialleben, Sicherheitsthemen, Erwartungshaltungen, etc. Um die Fragen referenzieren zu können, wird pro Fragebereich ein Zeichenliteral als Präfix definiert und dieses den einzelnen Fragen zusammen mit einer streng monoton steigenden Laufnummer als Frage-ID zugewiesen (Beispiel: F35) (vgl. \cite{kirchhoff_fragebogen:_2010}, Seite 37).

Die Fragen lassen sich für die spätere Auswertung einem (oder in einigen Fällen auch weiteren) der zuvor definierten Themenkomplexen zuordnen. Die offen formulierten Fragen stellen hierbei einen Sonderfall dar. Da sich die Antworten auf diese Fragen naturgemäss nicht direkt Auswerten lassen, ist ein Zwischenschritt notwendig, welcher diese Antworten zuerst analysiert. Aus diesem Grund werden die Antworten auf die offenen Fragen zusammen mit den Interviews ausgewertet.\footnotemark
 \end{sloppypar}

\footnotetext {Für eine genauere Beschreibung des Vorgehens siehe Kapitel \ref{auswertungsmethodiken}, Seite \pageref{auswertungsmethodiken}.}

\paragraph*{Definition der Antworttypen}\mbox{}

\begin{sloppypar}
Zwecks besserer Umsetzbarkeit und Lesbarkeit der Umfrage werden nachfolgend die Antworttypen definiert und diese dann jeweils einer Frage zugeordnet.

Die Antworttypen dienen der Standardisierung des Fragebogens und sollen die spätere Auswertung erleichtern. Die nachfolgende Tabelle definiert und beschreibt die möglichen Antworttypen, ihre jeweilige Kardinalität, womit eine Antwort ausgewählt werden kann sowie deren Anwendung bei offenen, respektive geschlossenen Fragen.
\end{sloppypar}

%% - table metadata
%% - --------------------------------------

\sloppy 

\begin{table}[H]
\tablefontsize	
\centering
\caption{Definition der Antworttypen}
\label{Definition der Antworttypen}
\begin{tabular}{ |p{1.0cm}|p{5.5cm}|p{4.3cm}|p{2.3cm}|p{2.5cm}|}

\hline
\tableheaderbgcolor
\textbf{Typ} & \textbf{Beschreibung} & \textbf{Kardinalität} & \textbf{Antwortauswahl} & \textbf{Anwendung}\\ 

\hline
\textbf{Typ A} &  Einschätzungsantwort mit unscharfer Bereichsangabe. Dieser Antworttyp wird bei Fragen verwendet, die eine gewisse Unsicherheit beim Probanden auslösen können, sei es durch die Fragenthematik selber oder durch die Art und Weise wie die Frage gestellt wird. & "`Trifft überhaupt nicht zu"' \newline bis \newline "`Trifft voll zu"' & Analogskala  & Geschlossene Frage\\

\hline
\textbf{Typ B} & Dieser Antworttyp wird bei Fragen verwendet, die mit einer scharfen Auswahl (Beispielsweise Ja oder Nein) beantwortet werden können. & Ja \newline Nein \newline unbekannt oder sagt mir nichts\tablefootnote{Gemäss dem Hinweis zur "`Wei\ss-nicht"' Kategorie von \citeauthor{mayer_interview_2013} (\citeyear{mayer_interview_2013}), S.93 f.} & Auswahlmenü & Geschlossene Frage\\

\hline
\textbf{Typ C} &  Dieser Antworttyp ermöglicht der befragten Person ihre Antwort frei zu formulieren. &  Unstrukturierte Textantwort & Direkte Eingabe & Offene Frage\\
\hline

\end{tabular}
\end{table}

%% - read questionnaire
%% - --------------------------------------

\subfile{sections/Teil_2_Texte/ev_ft_online_fragenkatalog}


\end{document}







