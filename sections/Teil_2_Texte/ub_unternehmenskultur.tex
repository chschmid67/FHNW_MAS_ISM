\begin{sloppypar}
Um dem Begriff "`Kultur"' mehr Präsenz zu verleihen, wurde das interdisziplinäre Kulturteam ins Leben gerufen. Es setzt sich aus freiwilligen Mitarbeitenden aus verschiedenen Unternehmensbereichen zusammen. Die Mission des Kulturteam ist es, die Diskussion um die Firmenkultur und die damit verbundenen Werte (Siehe Kapitel \ref{unternehmenswerte}, Seite \pageref{unternehmenswerte}) in das Unternehmen zu tragen.
\end{sloppypar}

\paragraph*{Kulturplattform}\mbox{}

\begin{sloppypar}
Alle Mitarbeitenden haben unter dem Motto "`Culture \& you"' die Möglichkeit, sich selber auf der Internet-Kulturplattform "`Crowdicity"'\footnote{\url{www.crowdicity.com}} einzubringen und können dort eigene Ideen, Gedanken und Anregungen publik machen und zur Diskussion und Abstimmung (voting) stellen. Damit soll die Diskussion um verschiedene (Kultur-)Themen als solches belebt werden; die Mitarbeitenden sollen dadurch die Unternehmenskultur durch eigene Initiativen selber mitprägen können. Mit der Plattform werden beispielsweise auch sozialisierende interne Anlässe wie die "`\companyshort{} Movie Night"' propagiert.

Solche Anlässe können gruppenweit koordiniert durchgeführt werden, so dass im Beispiel "`\companyshort{} Movie Night"' derselbe Film an den teilnehmenden Standorten ungefähr zur gleichen Zeit vorgeführt wird. Über die Plattform könne anschliessend Kommentare und Fotoreportagen zum stattgefundenen Anlass ausgetauscht werden, um so ein interkulturelles, globales "`Wir sind \companyshort"' Gefühl in den Mitarbeitern wachsen zu lassen.
\end{sloppypar}