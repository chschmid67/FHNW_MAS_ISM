\begin{sloppypar}
Der Wachstumspfad des Unternehmens wurde mit der Umsetzung der Internationalisierungsstrategie im Jahre 2008 eingeschlagen. Zu Beginn wurde vor allem im asiatischen und im ozeanischen Raum in den jeweiligen Finanzzentren \companybranchSEASIA, \companybranchCHINA und \companybranchOCEANIA je eine Niederlassung gegründet. Diese wurden von schweizerischen Mitarbeitern aufgebaut, die operative Führung wurde jedoch bereits nach kurzer Zeit an jeweils einheimische Führungskräfte übergeben. Die lokalen Mitarbeiter rekrutieren sich grösstenteils aus der ansässigen Bevölkerung.

Weitere Niederlassungen wurden im europäischen Raum (\companybranchFRANCE, \companybranchUK, \companybranchBENELUX) gegründet. Die aggressive Wachstumsstrategie gekoppelt mit dem neuen \acrfull{bpo} Geschäftsmodell für Banken führte dazu, dass weitere Firmen vor allem im deutschsprachigen Raum dazugekauft wurden.

Als weiterer Wachstumsfaktor wurde das Outsourcing von ganzen Funktionen wie z.B. das Softwaretesting, Teile der Softwareentwicklung sowie die technische Dokumentation an kostengünstigere Standorte vorangetrieben. In diesem Zuge wurde in \companyopscenterSEASIA ein Testzentrum und in \companydevcenterEU ein Software-Entwicklungszentrum gegründet.

Ausblickend ist der Wachstumstrend des Unternehmens ungebrochen. Einhergehend mit der digitalen Transformation nicht nur des Bankengeschäfts sondern der ganzen Gesellschaft strebt die \companyshort{} strategisch die Eroberung weiterer Finanzmärkte in Übersee durch den Ausbau des \acrlong{bpo} Geschäftsmodelles an. Weitere Firmenübernahmen sind denkbar, ebenso eine Steigerung der Mitarbeiterzahlen an Offshore-Standorten. Eine Expansion in BRIC-Staaten stellt eine erwägenswerte Option dar.
\end{sloppypar}

