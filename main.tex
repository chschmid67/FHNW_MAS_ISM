\documentclass[12pt]{article}

%% - packages definitions
%% - --------------------------------------

\usepackage[utf8]{inputenc}                     % - direct usage of ä,ö,ü possible
\usepackage[T1]{fontenc}                        % - fonts 256 chars
\usepackage[ngerman]{babel}                     % - new greman writing rules since 2006
\usepackage[acronym,nonumberlist]{glossaries}   % - leave out the page numbers
\usepackage{subfiles}                           % - enable usage of subfiles
\usepackage{blindtext}                          % - provide ability to generate blind text
\usepackage{fancyhdr}                           % - use fancy package for fancy pages, header, etc.
\usepackage{graphicx}                           % - provide graphics environment
\usepackage{array}                              % - next package must be [table]{xcolor} 
\usepackage[table]{xcolor}                      % - previous package must be {array}
\usepackage{tocloft}                            % - controlling the design of the TOC
\usepackage{geometry}                           % - specify document margins and size
\usepackage{hyphenat}                           % - global hyphenation rules
\usepackage[font=scriptsize]{caption}           % - set fontsize of figure captions
\usepackage{multirow}                           % - allow merging of fields of a table into one field
\usepackage{longtable}                          % - define long tables span multiple pages
\usepackage{floatrow}                           % - allow formatting of float values
\usepackage{csquotes}                           % - enable quoting style
\usepackage[backend=bibtex,style=alphabetic,sorting=ynt]{biblatex} % - bibliography
\usepackage{tabto}                              % - used in titlepage to align metadata
\usepackage{booktabs}                           % - improve table commands
\usepackage{titlesec}                           % - Select alternative section titles

\usepackage{tikz}
\usetikzlibrary{arrows}
\usetikzlibrary{decorations.markings}
\usetikzlibrary{matrix}
\usetikzlibrary{positioning}
\usetikzlibrary{shadings}


%% - Hyphenation rules
%% - --------------------------------------

%% - put hyphens where you want the word to be hyphenated
%% - if you do not want a word to be hyphenated at all put it i there without any hyphens
%% - --------------------------------------

\hyphenation{In-formations-tech-nik}
\hyphenation{In-for-mations-sicher-heit}
\hyphenation{ent-sprechen-den}
\hyphenation{Aware-ness}
\hyphenation{Out-sour-cing}

%% - Acronyms & Glossaries
%% - --------------------------------------

\makeglossaries

%% - List all acronyms below this block of comments
%% -
%% - example of usage: 
%% -
%% - \newacronym{is}{IS}{Informationssicherheit}
%% - In jedem Betrieb existiert eine \acrlong{is}, abgekürzt \acrshort{is}.
%% -
% - \newacronym{key}{\acrshort}{\acrlong}
%% -              ^
%% -              |
%% -        use "key" for references!!!
%% - --------------------------------------

\newacronym{bsi}{BSI}{Bundesamt für Sicherheit in der Informationstechnik}
\newacronym{bpo}{BPO}{Business Process Outsourcing}

%% - define where to look for images
%% - --------------------------------------

\graphicspath{ {figures/}{../figures/} }

%% - set bibliography file
%% - --------------------------------------

\addbibresource{Securityawareness.bib}

%% - Define the document's geometry and formatting
%% - --------------------------------------


%% - Define the document's geometry
%% - --------------------------------------

\geometry{a4paper, total={170mm,240mm}, headsep=5mm,left=25mm,top=30mm}


%% - Define everything that has to do with arrays
%% - 
%% - Set space between array rows to 1.5 (default 1.0)
%% - --------------------------------------

\renewcommand{\arraystretch}{1.5}

%% - define font (small) for "table" tables
%% - --------------------------------------

\floatsetup[table]{font=scriptsize}

%% - Define Table of Contents Style
%% - add 10pt of space after section entry in toc
%% - --------------------------------------

\renewcommand\cftsecafterpnum{\par\addvspace{10pt}}


%% - Define the Paragraph indent (parindent)
%% - and the Spacing between paragraphs (parskip)
%% - Also set the linespacing
%% - --------------------------------------

\setlength{\parindent}{0em}
\setlength{\parskip}{1em}
\renewcommand{\baselinestretch}{1.0}


%% - 3
%% - --------------------------------------

\titlespacing{\section}{%
  0pt}{%              left margin
  0\baselineskip}{% space before (vertical)
  0em}%               space after (horizontal)


%% - 3.2
%% - --------------------------------------

\titlespacing{\subsection}{%
  0pt}{%              left margin
  0\baselineskip}{% space before (vertical)
  1em}%               space after (horizontal)


%% - 3.2.1
%% - --------------------------------------

\titlespacing{\subsubsection}{%
  0pt}{%              left margin
  0\baselineskip}{% space before (vertical)
  0em}%               space after (horizontal)
  

\titlespacing{\paragraph}{%
  0pt}{%              left margin
  0\baselineskip}{% space before (vertical)
  0em}%               space after (horizontal)
  
\titlespacing{\subparagraph}{%
  0pt}{%              left margin
  0\baselineskip}{% space before (vertical)
  0em}%               space after (horizontal)
  
  

%% - Set document Header and Footer fancystyle
%% - Clear the header and footer
%% - --------------------------------------

\pagestyle{fancy}
\fancyhead{}
\fancyfoot{}

\fancyfoot[R]{\thepage}
\fancyhead[L]{Security Awareness Next Generation}

\renewcommand{\headrulewidth}{0pt} % draws the footer line
\renewcommand{\footrulewidth}{0pt} % draws the footer line

%% - Redefine plain page style, which is used 
%% - for titlepage and chapter beginnings
%% - to match fancy pagestyle
%% - --------------------------------------

\fancypagestyle{plain}{%
    \renewcommand{\headrulewidth}{0pt}%
    \fancyhf{}%
    \fancyfoot[R]{\thepage}%
    \fancyhead[L]{Security Awareness Next Generation}
}

%% - --------------------------------------
%% - Here the document begins
%% - --------------------------------------

\begin{document}

%% - create the title page
%% - --------------------------------------

%% - Define the document title page
%% - --------------------------------------

\begin{titlepage}
\includegraphics[width=0.45\textwidth]{FHNW_HW_10mm.eps}\par
\vspace{7cm}
{\Huge\bfseries Security Awareness \par}
{\large\bfseries Sicherheitsbewusstes Verhalten in einem international tätigen \\Unternehmen: Von OLDSCHOOL zu NEXT GENERATION  \par}
\vspace{4.5cm}
{\large\bfseries Master Thesis MAS Information Systems Management 2016\par}
\vspace{1.0cm}
\TabPositions{4cm}
{\normalsize Auftraggeberschaft:    \tab{\company}\\}
{\normalsize Autor:                 \tab{Christoph Schmid} \\}
{\normalsize Dozent:                \tab{Marco Marchesi, FHNW} \\}
{\normalsize Ort, Datum:            \tab{Pfungen, September 2016}}
\vspace{2.0cm}
\vfill
%% - Bottom of the title page
%% - --------------------------------------
{\small Erstellt mit \LaTeX}
\end{titlepage}


%% - Define simple footer for prolog stuff 
%% - use Roman page numbering (Big roman letters)
%% - --------------------------------------

\pagenumbering{Roman}
\setcounter{page}{0}

%% - Do all the prologue stuff
%% - --------------------------------------

\newpage
\subfile{sections/Prolog/Teil_Prolog_texte}

%% - use arabic page numbering
%% - reset counter of page to 1 
%% - --------------------------------------

\newpage
\vspace*{10cm}

\pagenumbering{arabic}
\setcounter{page}{1}

\setcounter{section}{1}
\setcounter{subsection}{0}
\section*{Erster Teil: Theoretische Grundlagen}
\addcontentsline{toc}{section}{Erster Teil: Theoretische Grundlagen}
\newpage
\subfile{sections/Teil_1_Texte/Teil_1_Theoretische_Grundlagen}

\newpage
\vspace*{10cm}
\setcounter{section}{2}
\setcounter{subsection}{0}
\section*{Zweiter Teil: Beschreibung des Unternehmensumfeldes}
\addcontentsline{toc}{section}{Zweiter Teil: Beschreibung des Unternehmensumfeldes}
\newpage
\subfile{sections/Teil_2_Texte/Teil_2_unternehmensumfeld}

\newpage
\vspace*{10cm}
\setcounter{section}{3}
\setcounter{subsection}{0}
\section*{Dritter Teil: Feststellung Security Awareness Ist-Situation}
\addcontentsline{toc}{section}{Dritter Teil: Feststellung Security Awareness Ist-Situation}
\newpage
\subfile{sections/Teil_3_Texte/Teil_3_SA_ist-situation}

\newpage
\vspace*{10cm}
\setcounter{section}{4}
\setcounter{subsection}{0}
\section*{Vierter Teil: Ziele und Massnahmen}
\addcontentsline{toc}{section}{Vierter Teil: Ziele und Massnahmen}
\newpage
\subfile{sections/Teil_4_Texte/Teil_4_Ziele_Massnahmen}

\newpage
\vspace*{10cm}
\setcounter{section}{5}
\setcounter{subsection}{0}
\section*{Fünfter Teil: Entwurf Security Awareness Programm}
\addcontentsline{toc}{section}{Fünfter Teil: Entwurf Security Awareness Programm}
\newpage
\subfile{sections/Teil_5_Texte/Teil_5_SA_Programm}

\newpage
\vspace*{10cm}
\setcounter{section}{6}
\setcounter{subsection}{0}
\section*{Sechster Teil: Fazit}
\addcontentsline{toc}{section}{Sechster Teil: Fazit}
\newpage
\subfile{sections/Teil_6_Texte/Teil_6_Fazit}

%% - Do all the epilogue stuff
%% - --------------------------------------

\subfile{sections/Epilog/Teil_Epilog_texte}

\end{document}
